\documentclass{grattan}
% Comments are deployed by the % sign; everything after % is ignored by the compiler.
% Please do not put comments before \documentclass as these are reserved for TeX directives.
% add_to_dictionary: SSBs? roman
\addbibresource{bib/put-new-refs-here.bib}

\author{John Daley}
\title{1 Orange Balloon}
\subtitle{Hielten sich fuer schlaue Leute und fühlten sich gleich angemacht}

\GrattanReportNumber{2018-00}

\acknowledgements{%
This report was written by Peter Goss, Julie Sonnemann, and Kate Griffiths.
Carmela Chivers provided extensive research assistance and made substantial contributions to the report.

We would like to thank the members of Grattan Institute's School Education Program Reference Group for their helpful comments, as well as numerous government and industry participants and officials for their input.

The opinions in this report are those of the authors and do not necessarily represent the views of Grattan Institute's founding members, affiliates, individual board members, reference group members or reviewers.
Any remaining errors or omissions are the responsibility of the authors.

Grattan Institute is an independent think-tank focused on Australian public policy.
Our work is independent, practical and rigorous.
We aim to improve policy outcomes by engaging with both decision-makers and the community.

For further information on the Institute's programs, or to join our mailing list, please go to: \textcolor{blue}{\url{http://www.grattan.edu.au/}}.

{\footnotesize
This report may be cited as:
Daley, J., Goss, P., Sonnemann, J., Griffiths, K., and Chivers, C\@. (2018). \emph{\mytitle}. Grattan Institute.

ISBN: 978-1-925015-96-6

All material published or otherwise created by Grattan Institute is licensed under a Creative Commons Attribution-NonCommercial-ShareAlike 3.0 Unported License\par
}
}




\begin{document}

\contentspage
\listoffigures
\listoftables


\chapter{Higher education providers }\label{chap:higher-education-providers}

What is higher education? The answer is surprisingly complex. This opening chapter explores the issue by examining the activities of universities, non-university higher education providers and other organisations in the higher education industry.

%
\section{What is higher education?}\label{sec:what-is-higher-education}

For many people in Australia, `higher education' and `universities' are synonyms. But universities are a particular kind of institution that delivers higher education. While universities educate most higher education students, they are a minority of higher education providers in Australia -- 42 of the 167 operating in mid-2018. This includes 40 universities, one specialist university and one overseas university.\footnote{TEQSA (2018b). Appendix A and appendix B have a full list of higher education providers. Two overseas universities are registered, but only one is operating, Carnegie Mellon University.} The other providers are colleges, institutes, and schools that are authorised to offer higher education qualifications. They are often known as NUHEPs -- non-university higher education providers.

Before offering higher education qualifications, higher education institutions must be registered by the Tertiary Education Quality and Standards Agency (TEQSA -- discussed in more detail in \Cref{subsec:higher-education-standards-panel}). TEQSA ensures that all institutions meet conditions set by government. They are expected to support free intellectual inquiry, offer teaching and learning that engages with advanced knowledge and inquiry, employ academic staff who are active in scholarship, and issue qualifications that comply with the Australian Qualifications Framework (AQF).\footnote{And a range of other requirements around governance and administration. Overseas universities can award their home country degrees: Department of Education and Training (2015c)..}

The power to award specific types of qualifications is the most important defining feature of a higher education provider. Free intellectual inquiry, engagement with advanced knowledge, and scholarship all occur outside as well as within the higher education sector. For these aspects of higher education no government permission is required: the market of ideas assesses value. It is the licence to issue AQF-recognised higher education qualifications, to certify individuals as having acquired knowledge and skills, that makes higher education providers distinctive.

Qualifications are differentiated according to the knowledge and skills required for their successful completion. Table 1 shows the AQF qualifications, ranked from 1 to 10. \footnote{A review of the AQF commenced in 2018. For an overview of the issues see PhillipsKPA (2018).} Generally, certificates I to IV (levels 1 to 4) are classified as vocational, while associate degrees through doctoral degrees (levels 6 to 10) are classified as higher education. Level 5 diplomas and level 6 advanced diplomas can be vocational or higher education, though in practice most are taught in the vocational education sector. Although level 8 graduate certificates and diplomas have been removed from the AQF as vocational qualifications some are still available, but most students at this level are in higher education.\footnote{Department of Education and Training (2017k), table 2.3; NCVER (2018), table 7}

\begin{table} \caption{Australian Qualifications Framework}

%%\begin{longtable}[]{@{}ll@{}}
%%\toprule
%%\textbf{Level} & \textbf{Qualification}\tabularnewline
%%\midrule
%%\endhead
%%1 & Certificate I\tabularnewline
%%2 & Certificate II\tabularnewline
%%3 & Certificate III\tabularnewline
%%4 & Certificate IV\tabularnewline
%%5 & Diploma\tabularnewline
%%6 & Advanced Diploma; Associate Degree\tabularnewline
%%7 & Bachelor Degree\tabularnewline
%%8 & Bachelor Honours Degree; Graduate Certificate; Graduate Diploma\tabularnewline
%%9 & Masters Degree\tabularnewline
%%10 & Doctoral Degree\tabularnewline
%%\bottomrule
%%\end{longtable}

\end{table}

Key differences between the qualifications include the level of theoretical knowledge required, and the student's capacity to analyse information, make independent judgments and devise solutions to problems. Certificate I or II holders are expected to apply technical skills to routine tasks or predictable problems, while doctoral degree graduates are expected to be able to create new knowledge. In the middle classifications there are sometimes subtle distinctions. A certificate IV holder is expected to analyse information to complete a range of activities, while a bachelor degree holder is expected to analyse and evaluate the information. A certificate IV holder is expected to provide solutions to sometimes complex problems, while a bachelor degree holder is expected to generate solutions to problems that are sometimes complex and unpredictable. The AQF encourages pathways between the qualifications, including credit towards bachelor degrees for time spent acquiring diplomas, advanced diplomas, and associate degrees.

Since there is a continuum of knowledge and skills rather than sharp dividing lines between the AQF levels, the distinctions between vocational and higher education are partly a matter of convention. The terminology should not be taken to imply that one sector is concerned with the world of work and the other is not. Most higher education students are seeking vocational outcomes. When the Australian Bureau of Statistics asked people studying qualifications in the past year about their main reason for undertaking learning, 83 per cent of those enrolled in higher education gave a job-related reason. For people in certificate III and IV qualifications, 88 per cent gave a job-related reason.\footnote{Calculated from ABS (2018d)}

Vocational and higher education providers also overlap. Public-sector vocational education providers, the TAFEs, have added higher education qualifications to their course programs; eleven had done so as of mid-2018.\footnote{Not counting dual sector providers with higher education as a principal purpose.} Especially in Victoria, some universities are `dual sector', with a TAFE as well. Other universities offer vocational education courses directly or through subsidiaries. In the private sector, many institutions offer both higher education and vocational education courses. All up, 81 organisations provide both higher and vocational education courses.\footnote{Based on the training.gov.au website and TEQSA's National Register of Higher Education Providers: TEQSA (2018b).}

%
\section{Non-university higher education providers}\label{sec:non-university-higher-education-providers}

Non-university higher education providers (NUHEPs) are a significant part of Australian higher education. In July 2018, 125 NUHEPs (listed in Appendix A and Appendix B) were registered with TEQSA.\footnote{Based on TEQSA's National Register of Higher Education Providers: ibid.} Some are public institutions: the Australian Film, Television and Radio School, the Australian Institute of Police Management, and the various TAFEs now offering degrees, for example. Some are hard to classify on a public-private spectrum, as they are for-profit colleges owned by public universities. But most (103) are clearly in the private sector. The number has fallen slightly in recent years, but is still well up on the 78 private NUHEPs in 1999.\footnote{Watson (2000). There is also significant turnover in NUHEPs, including closures, mergers and takeovers.}

Although NUHEP numbers have expanded over the last 20 years, the provider count is not straightforward. Some providers have multiple trading names, so there are more than 125 brands in the market. But some separately registered providers have common owners, so the number of players in the industry is less than 125. For example, Navitas Limited owns or partly owns 12 NUHEPs.

The private NUHEPs are a mix of not-for-profit and for-profit providers. In 2016, TEQSA identified 54 private NUHEPs as not-for-profit and 65 as for-profit.\footnote{TEQSA (2018c), p 6} As of July 2018, 41 NUHEPs are registered charities with the Australian Charities and Not-for-profits Commission (ACNC).\footnote{ACNC (2018).To be registered, higher education providers must have a charitable purpose in the public benefit. `Advancing education' is a legislated charitable purpose: \emph{Charities Act 2013}, division 2.} NUHEP finances are discussed in \Cref{sec:overall-financial-position}.

We cannot say exactly how many students are taught in NUHEPs. Where public universities outsource teaching (\Cref{sec:higher-education-service-providers}) the students are counted in the university rather than the teaching institution. With this caveat, in 2016 NUHEPs enrolled nearly 81,000 full-time equivalent students.\footnote{TEQSA (2018c), p 12} It is a big increase on slightly less than 15,000 full-time equivalent students in 1999, but only an 8 per cent market share (see \Cref{sec:selecting-students} for more detail on enrolments).

Most NUHEPs are specialised compared to universities (discussed in \Cref{sec:what-is-distinctive-about-universities}). Usually, teaching is their only major education function, without about half (63) offering vocational as well as higher education courses. Only one NUHEP offers the full range of higher education AQF qualifications from a diploma to a PhD, and many offer only one or two of the six higher education qualification levels.

Fifteen pathway colleges specialise in diploma-level courses. Their purpose is to prepare students for entry into the second year of a university course. Typically, they have a relationship with a specific university, and the diploma curriculum matches that taught in the target university's first year. For example, students who successfully complete a Diploma of Business at Deakin College can enter the second year of a Deakin University Bachelor of Business.

Other institutions offer only postgraduate courses, often serving specific occupations with professional admission or development courses. Fourteen NUHEPs are in this category. For example, the College of Law offers entirely postgraduate courses as it prepares law graduates for practice or gives lawyers additional specialist skills.

Many other NUHEPs include a specific field of study, industry or occupation in their title, for example: Kaplan Business School, International College of Hotel Management, and the Australian College of Nursing. Subject specialisation can build provider reputations in niche areas.

NUHEPs offer more courses in business than any other field. These include some delivered by professional associations such as Chartered Accountants Australia and New Zealand. There are also a significant number (19) of institutions with a religious affiliation. Some are theological colleges, but others offer a wider range of courses.

Health, and particularly alternative health, is also a common field in the non-university higher education sector. Nineteen providers have a health-related speciality. Another 16 colleges specialise in various kinds of creative arts.

Some NUHEPs are primarily focused on the international student market. Twenty-six NUHEPs with published enrolment data report that two-thirds or more of their students are from overseas, and another ten primarily or exclusively market to international students. But 37 NUHEPs only take domestic students.\footnote{To take international students, higher education providers must be registered on the Commonwealth Register of Institutions and Courses for Overseas Students (CRICOS). The 37 are not registered.}

In most cases, TEQSA accredits NUHEP courses.\footnote{TEQSA (2017c)} The accreditation process includes examining course content, student admission criteria, assessment methods, and staff qualifications. NUHEPs with appropriate quality assurance systems and a track record of re-accreditation can become self-accrediting -- a legal right to approve their own courses. However, most NUHEPs are not self-accrediting.\footnote{The twelve self-accrediting NUHEPs are noted in Appendix A.}

On top of these licence-to-operate requirements, some NUHEPs seek other third-party approval or endorsement of their courses. For example, NUHEPs offering accounting courses have them recognised by CPA Australia, so their graduates can become members of that accounting professional association. Some courses at the Australian College of Applied Psychology are approved by the Australian Psychology Accreditation Council, which also registers psychologists practising in Australia.

%
\section{What is distinctive about universities?}\label{sec:what-is-distinctive-about-universities}

`University' is a regulated term in Australia. No educational organisation can operate as an Australian university without meeting criteria set out in law. Commonwealth Government Provider Category Standards enforced by TEQSA determine which institutions can operate as universities.\footnote{Department of Education and Training (2015c). A list of universities is in Appendix A. Most universities also have their own founding legislation, usually from a state government.}

%
\subsection{Research}\label{subsec:research}

The most important distinctive aspect of universities as higher education institutions is their combination of research and teaching. Research is defined as original work conducted to produce new knowledge. To be a full Australian university, a higher education provider must be active in research across at least three broad fields of study: disciplines such as health, engineering, education, or science.\footnote{A detailed categorisation of disciplines can be found in ABS (2001).} Higher education institutions with research activity in only one or two fields can apply to be a specialist university. Under this provision, the Melbourne College of Divinity became the University of Divinity.

While the idea that universities must be active in research is widely accepted in Australia today, the original Australian universities established in the mid-19\textsuperscript{th} century were to be places of scholarship -- expertise in existing knowledge rather than original research. Though universities were conducting some research by the latter part of the 19\textsuperscript{th} century, PhD degrees were not offered until the 1940s.\footnote{Starting with the University of Melbourne in 1945: Forsyth (2014), p 27.}

In the late 1980s and early 1990s, predominantly teaching-focused colleges of advanced education and other government-funded higher education institutions were turned into or merged with universities, substantially diluting the university sector's research orientation. The universities that were created as a result are still sometimes referred to as `Dawkins universities' (after the minister behind the policy, John Dawkins).\footnote{The universities created during the Dawkins years are noted in the list of universities in Appendix A. For a more detailed history see Macintyre\emph{, et al.} (2017).} Yet only 10 years later, research became a defining legal feature of a university.\footnote{Through agreements between education ministers: MCEETYA (2000), later replaced by MCEETYA (2007).}

The research requirement has made it difficult for new universities to start. University research typically is not self-financing. Public research funding is primarily awarded according to past research performance, which makes it hard for new universities to build research output. So after a period in the 1980s and 1990s, when many new universities were created, only one has opened since the three fields of study rule came into effect in 2000, Torrens University Australia.\footnote{Torrens University Australia was established under now-abolished provisions that allowed greenfield universities to have a plan to meet the three fields rule: MCEETYA (2007), \Cref{sec:the-rise-of-commonwealth-authority}.1.}

Universities aspire to a teaching-research nexus: the idea that teaching and research are mutually beneficial, not just two separate functions of the same institution. Both student satisfaction with teaching and research output have improved in the last 15 years (\Cref{sec:student-satisfaction-with-teaching} and 5.3) which might suggest a synergy. Profits on teaching finance research (\Cref{subsec:other-sources-of-research-funding}). But teaching performance is not closely linked with research quality.\footnote{See Cherastidtham\emph{, et al.} (2013) and the summary and references at Norton and Cherastidtham (2015a), p 31-33.}

%
\subsection{Comprehensiveness}\label{subsec:comprehensiveness}

While many NUHEPs are specialised in what they teach (\Cref{sec:non-university-higher-education-providers}), full Australian universities must offer courses in at least three broad fields of study, as classified by the Australian Bureau of Statistics.\footnote{ABS (2001)} Most offer more. They are often described as being `comprehensive' in the range of courses they offer. Two-thirds of universities have students in all ten major broad fields of study, and all but two have at least eight major fields of study.\footnote{Calculated from Department of Education and Training (2018h).}

While many students specialise in their university studies, the comprehensive nature of Australian universities creates opportunities for studying more than one field. Australian universities offer many combined qualifications, such as arts/law or commerce/science, so that students graduate with two degrees. Eight per cent of domestic bachelor enrolments in 2016 were in combined courses.\footnote{Department of Education and Training (2017e)} Many students also take units from faculties other than the one they are principally enrolled in. For example, an arts student may do a mathematics unit taught by a science faculty.

Comprehensiveness also extends to the range of qualifications offered. All full universities offer courses from bachelor through to PhD (\Cref{sec:what-is-higher-education}). Some also offer diploma, associate degree and vocational education qualifications.

%
\subsection{Self-accreditation}\label{subsec:self-accreditation}

Unlike other higher education institutions, Australian universities automatically have the right to accredit their own courses. University academic boards approve their university's courses, within a framework established by government regulation.\footnote{At universities, the academic board itself is usually established or required under the university's founding legislation. The role of academic boards is discussed in Dooley\emph{, et al.} (2013). Courses must meet the requirements of the Higher Education Standards Framework: Department of Education and Training (2015c).} Self-accreditation is an aspect of academic freedom (\Cref{subsec:academic-freedom}). In developing courses, academics in self-accrediting universities can include material without seeking a government agency's approval. They are instead regulated by their fellow academics.\footnote{Although in practice the role and power of academic boards varies between universities, with central administrations also playing a role: see Rowlands (2017).}

Though universities self-accredit, they also seek external accreditation or recognition. Often this is necessary for their graduates to be admitted to professional practice. Universities sometimes also secure international recognition for their qualifications. In total, about 100 accrediting agencies or professional bodies set requirements for university courses.\footnote{PhillipsKPA (2017), appendix 1} Although external influence on course content is common in courses linked to the professions, in other areas it can cause controversy, especially when it comes from politically sensitive sources.\footnote{For example, a proposal by the Ramsay Centre to fund a course on Western civilisation and Chinese Government influence via Confucius Institutes: Bongiorno (2018); Haines (2018); Hamilton and Joske (2018), \Cref{chap:benefits-for-graduates}.}

%
\subsection{Academic freedom}\label{subsec:academic-freedom}

The institutional freedom of self-accreditation has its individual equivalent in the idea of academic freedom. As one American study put it, `academic freedom establishes the liberty necessary to advance knowledge, which is the liberty to practise the scholarly profession.'\footnote{Finkin and Post (2009), p 39} This includes freedom from government, from external funders and, in more complex ways, from the university administration. Surveys of academics show that freedom to pursue their own research interests is a major part of what attracts them to universities.\footnote{Bexley\emph{, et al.} (2011), p 66}

Academic freedom is a narrower concept than freedom of speech. It protects academics in their professional capacity, as members of the university and scholars in their field. It does not guarantee a broad personal freedom to comment on any matter of interest. Academic freedom's legislative protections, in the licence to operate rules enforced by TEQSA and the funding laws implemented by the Department of Education and Training, refer to `free intellectual inquiry', not free speech.\footnote{\emph{Higher Education Support Act 2003}, section 19-115; Department of Education and Training (2015c), A6.1.4 and B1.1.2. These provisions appear to apply to students as well as academics.}

University policies and enterprise agreements set out more detail on how academic freedom works in practice.\footnote{Stobbs (2015); Gelber (2018)} Typically, universities allow academics a public role outside as well as within their scholarly expertise, but sometimes discourage use of their university affiliation. Occasionally university administrations dismiss or discipline academics who make controversial or embarrassing public statements.\footnote{For examples and some background, see Jackson (2005) and O'Brien (2015) p. 223-229.} Such actions almost always attract strong criticism, as academics do not see this as a legitimate role for managers (see more in \Cref{subsec:self-governing-communities} below). With much of the academic workforce on casual or fixed-term contracts (\Cref{sec:casual-and-fixed-term-academic-employment}), self-censorship may affect academic freedom more than actual administrative action.

The funding system also influences how academic freedom works in practice. The major government research project funding programs (\Cref{subsec:public-research-funding-programs}) respect academic freedom, as they are assessed by academics from applications by other academics. But research expenditure has biases towards medical and scientific disciplines, and towards research with potential practical outcomes (\Cref{sec:research-topics-and-types}). Academics with other research priorities receive less financial support. In 2010, six in ten academics agreed that they had freedom to pursue their research interests, but were much less likely to say they had the time or funding to do so.\footnote{Bexley\emph{, et al.} (2013), p 66}

%
\subsection{Self-governing communities}\label{subsec:self-governing-communities}

Universities see themselves as self-governing communities. Both public and private universities are legally distinct from government.\footnote{For discussions of the corporate status of universities see Corcoran (2000) and Orr (2012).} At public universities, government appointments to university governing bodies, commonly called councils or senates, are never a majority. Private universities have no such appointments. Education ministers have no direct operational control. Partly for historical constitutional reasons, much Commonwealth government regulation is via conditions on grants (\Cref{sec:the-rise-of-commonwealth-authority}), which universities can decline.

Within universities, academics see themselves as members of the university community and not just as employees.\footnote{See the discussion in Forsyth (2014), especially \Cref{chap:higher-education-finance-the-micro-picture} \& 8.} The legal force of this distinction was explicitly acknowledged in a Federal Court judgment.\footnote{In \emph{University of Western Australia v Gray} the court held that academic staff were, by virtue of the definition of `university' in the UWA Act, members of a university, `linked historically by that definition to the idea of the university as a community of teachers and scholars', see Stobbs (2015).} Academics expect inclusion in collective decisions, a decision-making process known as collegiality. Traditionally academics elect members to university senates and councils. Academic critiques of university administrators often complain about what they call managerialism -- managers directing academics or steering their behaviour through targets and incentives. Managerialism is seen as an ideological rival to collegiality.\footnote{University staff surveys show low confidence in university management: NTEU (2018).}

Despite complaints that university managers are too powerful, university organisational structures are highly decentralised compared to for-profit corporations, with large amounts of consultation and decision-by-committee.

Unlike at universities in some other countries, Australian students do not usually live on campus.\footnote{In 2017 there were 56,176 on-campus student accommodation beds and a further 32,198 off-campus student accommodation beds, usually within a short walking distance of campus: Urbis (2018), p iii. Together they could accommodate less than 6 per cent of enrolments.} But the idea of the university as a community helps explain why they often provide a wide range of health and welfare services. Student groups are represented in university decision making, often through student associations officially recognised by the university. Traditionally this recognition was granted; it is now required by regulation.\footnote{DIICCSRTE (2013); Department of Education and Training (2015c), A6.3} The role and funding of official student organisations has been the subject of a long-running political dispute between the Liberal Party on one side, and official student organisations, universities and the Labor Party on the other.\footnote{Norton (2005). A compulsory fee for student services and amenities, a long-standing aspect of Australian higher education, was abolished by the Liberal Howard government from 2006. A more limited fee, capped in total and with restrictions on how it can be used, was restored by the Labor Gillard government from 2012.} This was in part a debate about the scope of the university. Is it just a provider of education and research services, or is it something more to its staff and students?

%
\subsection{Broad social responsibilities }\label{subsec:broad-social-responsibilities}

As well as being a community in themselves, universities are expected to contribute to the broader community. Community engagement is sometimes referred to as the third stream of university activity, after teaching and research. It can include universities working with or for local communities, government, industry, not-for-profits, and the media. The standards for registration as a university elevate some of these activities from desirable to necessary, requiring demonstrated engagement with local and regional communities, and a commitment to `social responsibility' in their activities.\footnote{Department of Education and Training (2015c), B1.2} Research policy has encouraged universities to focus on practical problems or commercial opportunities (\Cref{sec:research-topics-and-types}, 6.4.1, 9.3). It is also common for university founding statutes to include community engagement objectives.

Community engagement is so diverse that it is hard to measure. One input indicator comes from academic time-use surveys, although the published surveys include other activities. The latest, from 2015, found that academics spent on average 5.3 hours a week on community and university service, out of an average 50.7 hours of work.\footnote{NTEU (2015b). A 2007 survey, which excluded internal university service but included services to clients or patients as well as community service, reported 4.4 hours out of 50.6 hours a week: Coates\emph{, et al.} (2009).} An earlier survey of academics found that more than half believed that community service should be rewarded in promotions, though only 15 per cent said that it was rewarded.\footnote{Bexley\emph{, et al.} (2011)} Although a distant third priority after research and teaching, community service is an important part of university culture and practice.

%
\subsection{Multiple missions}\label{subsec:multiple-missions}

Though the term `university' has a formal legal definition, no single feature makes universities distinct as higher education providers. NUHEPs conduct research, self-accredit, give their academic staff freedom, operate as a community of scholars, and engage with broader social responsibilities. But few NUHEPs do all of these things, and most have limited functions beyond teaching. Contemporary Australian universities are characterised by their combination of activities more than by any one feature.

%
\section{Higher education service providers }\label{sec:higher-education-service-providers}

Although only TEQSA-registered higher education providers can award higher education qualifications, other organisations support higher education providers or deliver related higher education services.

While universities do their own marketing, intermediary organisations help co-ordinate the matching of students with courses and institutions. The most important intermediaries are the state-based tertiary admissions centres, which handle most school-leaver applications for university (\Cref{sec:selecting-students}). In the international student market, IDP Education helps match international students with universities in Australia and elsewhere.

Open Universities Australia (OUA) does not deliver education or award degrees. It sells online units and courses offered by its seven shareholder universities and other higher education providers. It is unusual in promoting not-for-degree units; selling just knowledge without a credential (though students may apply to individual universities for credit towards a degree for OUA units).

Organisations such as Blackboard, Canvas and Moodle help universities co-ordinate teaching-related activities through software known as learning management systems. These store course content and are used to submit work, run student forums, record assessment results, and do other administrative tasks. Other technology firms offer `adaptive learning' software, where online course materials adjust to the student. This includes an Australian company, Smart Sparrow.

Companies offering one-to-one online tutorial services for university students operate in the Australian market. An Australian tutorial service provider, Studiosity, works with sixteen universities. Another firm in this market in Smarthinking, which is owned by the world's largest international educational services company, Pearson Education.

Universities also outsource course delivery. For example, Swinburne Online courses are delivered by Online Education Services, a subsidiary of the SEEK job search company. Navitas operates La Trobe University's `La Trobe University Sydney' campus. Students study a La Trobe University curriculum and are awarded a La Trobe University degree. Queensland TAFE offers University of Canberra degrees. There are other similar arrangements around Australia and at offshore campuses.

.

%
\chapter{Higher education students }\label{chap:higher-education-students}

This chapter examines how people enter higher education course, how many students are enrolled, what they are studying, and some of their social characteristics.

%
\section{Selecting students}\label{sec:selecting-students}

Higher education is popular in Australia. For decades most upper-year school students in Australia have expressed an interest in going to university.\footnote{James (2002), p. 15; Mission Australia (2016), p. 15-16; Roy Morgan Research (2009), p. 48-50; ANOP/DEET (1994), p. 7-9, ACER (2018), p 10.} In 2017, 57 per cent of the students who completed Year 12 the previous year applied for university.\footnote{Department of Education and Training (2017m), p 39} More of the students completing school in 2016 will apply in later years.

Most undergraduate applicants, 243,000 in 2017, use state-based tertiary admissions centres.\footnote{Ibid., p 11} Tertiary admission centre applicants list the courses they would like to take in order of their preferences. In effect, applicants simultaneously apply to multiple higher education providers and/or for multiple courses at the same provider. If they miss out on their first preference course they can still receive an offer for their second or a lower preference course.

Although the admissions centres handle most applications, increasing numbers of people, more than 124,000 in 2017, apply directly to higher education providers.\footnote{Ibid., p 49. In practice, universities can outsource processing these applications to a tertiary admission centre. The student is just not using the multi-university preference system.} Mature age students especially apply directly. Reasons include only having one preference, early admission opportunities that bypass the tertiary admission centres, universities using additional selection criteria such as non-academic personal attributes, and prospective students thinking it is easier or more appropriate to them.\footnote{Harvey\emph{, et al.} (2016), p 55-56, 96-97} Direct applications are the most common way of entering postgraduate courses.

Universities cannot take anyone who wants to attend. Legally, no higher education provider should accept prospective students without the academic preparation and English proficiency needed for the course.\footnote{Department of Education and Training (2015c), Part A, 1.1} In practice, however, most people who want to attend university can do so. Some universities offer places to more than 90 per cent of the people who apply, and overall 82 per cent of applicants received an offer in 2017.\footnote{Department of Education and Training (2017m), appendix table A10}

Although the higher education system can accommodate most demand, entry to particular universities and courses is rationed. The reflects capacity constraints, course academic requirements, and prestige factors. Typically, places in over-subscribed courses are allocated based on prior academic performance at school, in another tertiary course, or on an admission test. Universities relax their entry requirements for students from disadvantaged backgrounds. But the better an applicant's past academic results, the better their chance of being offered a place.

The most frequently used source of information on past academic performance is school results. In 2016, 42 per cent of bachelor degree students were admitted based on their secondary education. In most cases their Australian Tertiary Admission Rank (ATAR) plays an important role in the admission decision.\footnote{Formerly called ENTER in Victoria, UAI in NSW, and TER in other jurisdictions except Queensland, which kept its OP system. The OP system will be replaced by ATAR in 2020. In 2017, 94 per cent of school leaver applicants had a valid ATAR: ibid., p 39. Universities do not always use ATAR as the basis of admission for school leavers, but in 2016 three-quarters of those admitted based on secondary education had their ATAR recorded in the enrolment data: Department of Education and Training (2017e).} The ATAR ranks school leavers in their age cohort between 0 and 99.95. For example, an ATAR of 80 means that the student did better in Year 12 than 80 per cent of their age cohort, including people who did not finish school. ATARs below 30 are just reported to students as `less than 30'.\footnote{Strictly speaking, ATAR is a rank of students who did Year 7 in each state in the same year. For more detail see UAC (2018); VTAC (2017).}

More low-ATAR students are admitted to university now than in the past, but ATAR and university attendance are still closely linked. As ATAR declines, school leavers become less likely to apply for a university course, receive an offer, or accept an offer.\footnote{Norton\emph{, et al.} (2018), p 8} In 2017, nearly 22,000 school leavers with an ATAR of 90 or more accepted an offer of a university place, compared to less than 5000 school leavers with an ATAR of 50 or less.\footnote{Department of Education and Training (2017m), appendix table A8,1}

At some universities, an undergraduate general admission test for school leavers, UniTest, supplements school result-based admission systems. Specialised admissions tests are used in some fields. An example is the UMAT (Undergraduate Medicine and Health Sciences Admission Test) used by students applying for medicine at some universities. Auditions are often used for courses in the performing arts.\footnote{There are no national statistics on how widely used these tests are. In 2016, 20 per cent of the total commencing undergraduate students who completed school in 2015 were admitted based on something other than their secondary school results: Department of Education and Training (2015d).} Mature-age applicants can sit the Specialised Tertiary Admissions Test (STAT).\footnote{For a study of STAT's predictive value see Coates and Friedman (2010).}

Nearly thirty percent of domestic commencing bachelor-degree students are admitted based on previous higher education study. These include students who attended pathway colleges that award undergraduate diplomas, students switching courses or universities, or students returning for a second degree. Twelve per cent of applicants are admitted based on their vocational education qualifications or experience.

For bachelor-degree international students, universities set admission requirements based on home country school systems or international qualifications such as the International Baccalaureate.\footnote{International students also enter university from Australian schools or after other preparatory study.} International students must also demonstrate English language proficiency, which is discussed further in \Cref{sec:passing-and-failing}.\footnote{Arkoudis\emph{, et al.} (2012), \Cref{chap:higher-education-students}; Muller (2017)}

University offers to bachelor applicants are not the end of the selection process. About ten per cent of people who receive an offer do not accept it, and another 13 per cent accept but defer commencing.\footnote{The annual statistics on applications exclude those who defer from acceptances: Department of Education and Training (2017m), p 22. Exclusion helps estimate how many students will enrol that year. Of those who deferred in 2014, 32 per cent of school leavers and 53 per cent of older applicants did not reach a census date in the next two years. This analysis uses applicants for the 2014 academic year who are then tracked for the next two years: Department of Education and Training (various years-c).} Domestic applicants who accept their offer and enrol typically have until about a month after teaching starts to decide whether to proceed with a subject or course. The decision time is called the `census date', which must be at last 20 per cent of the way through a subject. Students who end their enrolment before the census date don't pay their student contributions or fees. About 7 per cent of the applicants who accept an offer leave prior to the first census date.\footnote{Norton\emph{, et al.} (2018), \Cref{sec:public-spending-on-higher-education-overview}. Some universities have separate processes for accepting an offer and enrolling, so some of this 7 per cent may never have enrolled.}

Intense competition between applicants for places in some universities and courses draws attention to how universities choose students. But in the higher education system overall universities compete for students who make choices in their initial applications, in whether to accept offers, in whether to stay past the first census date, and then in whether to keep going with their original course, transfer to another one, or leave higher education entirely. It is an ongoing process of mutual selection, with students and universities each making important decisions about the other.\footnote{Ibid., \Cref{sec:what-is-higher-education}}

%
\section{Overall enrolment trends}\label{sec:overall-enrolment-trends}

Australian higher education student numbers have grown rapidly since the 1960s, as \Cref{fig:higher-education-students-19502016} shows. In 2016, total domestic and international student enrolments were 1.48 million.\footnote{TEQSA (2018c). This is a slightly higher number than reported in \Cref{fig:higher-education-students-19502016}, as TEQSA includes enrolments from providers that receive no government financial support (listed in Appendix B). However, there is limited historical data available from TEQSA.}

Although undergraduate numbers increased by the largest absolute number over the last 30 years, postgraduate coursework grew from 14 to 25 per cent of all enrolments. While the number of research students expanded (\Cref{sec:the-research-workforce}), their share of all students has been stable at around 5 per cent for many years. \Cref{fig:enrolment-share-by-level-of-study-19892016} shows the trends.


    \begin{figure} %% original Figure 1
    \caption{Higher education students, 1950--2016}\label{fig:higher-education-students-19502016}
    \units{Millions of students}
    \includegraphics[page= 1, width=1\columnwidth]{atlas/mycharts.pdf}
    \noteswithsources{{Figures from 2001 onwards are based on enrolments at any time throughout the year; prior years are based on enrolments as at 31st March.}{} Department of Education and Training (various years-c); Department of Education and Training (2014); Department of Education and Training (2018h)}
    \end{figure}



    \begin{figure} %% original Figure 2
    \caption{Enrolment share by level of study, 1989--2016}\label{fig:enrolment-share-by-level-of-study-19892016}
    \units{Proportion of enrolments}
    \includegraphics[page= 2, width=1\columnwidth]{atlas/mycharts.pdf}
    \noteswithsources{{Doctorate by coursework and extended masters are classified as postgraduate coursework. }}{{} {Department of Education and Training (2014(2018h); )}}
    \end{figure}



%
\section{Domestic students }\label{sec:domestic-students}

Just under three-quarters of students enrolled in Australian higher education institutions are Australian citizens or permanent residents. Occasional years of slow growth or small declines in student numbers only interrupt the long-term trend towards more students (\Cref{fig:domestic-higher-education-students-19892016}). In 2016, there were 1.06 million domestic students.


    \begin{figure} %% original Figure 3
    \caption{Domestic higher education students, 1989--2016}\label{fig:domestic-higher-education-students-19892016}
    \units{Millions of students}
    \includegraphics[page= 3, width=1\columnwidth]{atlas/mycharts.pdf}
    \notewithsources{{Figures from 2001 onwards are based on enrolments at any time throughout the year; prior years are based on enrolments as at 31st March.}}{Department of Education and Training (2014(2018h); )}
    \end{figure}



Australia's population has also increased in this period. Participation rates adjust for this by showing the proportion of people enrolled in higher education. \Cref{fig:domestic-higher-education-participation-rates-19-year-olds-19892016} reports higher education participation rates for 19 year-olds. Forty-one per cent of 19 year-olds were enrolled in higher education in 2016, more than double the rate in 1989.


    \begin{figure} %% original Figure 4
    \caption{Domestic higher education participation rates, 19 year olds, 1989--2016}\label{fig:domestic-higher-education-participation-rates-19-year-olds-19892016}
    \units{Proportion of population enrolled}
    \includegraphics[page= 4, width=1\columnwidth]{atlas/mycharts.pdf}
    \noteswithsources{The 19 year-old participation rate is the number of Australian citizen and permanent resident, and NZ citizen, 19 year-olds who are enrolled in higher education as a proportion of total 19 year-old residents in a given year less onshore international students (temporary visa-holders). Offshore international students are excluded from the calculation.}{{Department of Education and Training (various years-c);} Department of Education and Training (2018h); ABS (2018a)}
    \end{figure}



Domestic student numbers are not expected to expand significantly in the next few years. The late teenage population, the main source of new undergraduate students, is not growing significantly.\footnote{ABS (2018a)} Funding policy changes will discourage universities from taking more undergraduate students (\Cref{sec:spending-on-and-by-students}). The number of domestic students starting postgraduate coursework degrees fell 9 per cent between 2014 and 2016.\footnote{Department of Education and Training (2018h)}

%
\subsection{Courses taken by domestic students}\label{subsec:courses-taken-by-domestic-students}

Australian universities have mixed general and professional education from their earliest days, and this remains true today, although professional courses dominate (\Cref{fig:domestic-enrolment-share-time-series-by-selected-fields-of-education-19892016}).\footnote{Davis (2017), p 42-44} Since 2001, commerce and IT have lost enrolment share, while health-related fields have gained. Because overall enrolments expanded significantly in this period (\Cref{fig:domestic-higher-education-students-19892016}), all disciplines except IT have increased their total student numbers.

    \begin{figure} %% original Figure 5
    \caption{Domestic enrolment share time series by selected fields of education, 1989-2016}\label{fig:domestic-enrolment-share-time-series-by-selected-fields-of-education-19892016}
    \units{}
    \includegraphics[page= 5, width=1\columnwidth]{atlas/mycharts.pdf}
    \noteswithsource{Shows domestic enrolments in undergraduate and postgraduate non-research courses. Society and culture is an ABS category. It includes humanities, social sciences, psychology and other fields listed in Appendix C. We have removed law from the count. Remaining fields are classified according to the table in Appendix C. In 2001, the ABS moved from Field of Study classifications to the more detailed Field of Education classifications. Correspondence between the two classifications is imperfect, but acceptable for broad fields.}{Department of Education and Training (2018h(various years-b); )}
    \end{figure}


%
\subsection{Full and part-time enrolment}\label{subsec:full-and-part-time-enrolment}

Most undergraduate students are enrolled full-time. Since 2001 the proportion of undergraduates studying full-time has increased, but since 2010 has tapered off slightly to 76 per cent (\Cref{fig:proportion-of-domestic-students-enrolled-fulltime-19892016}). Postgraduate students are much less likely to study full-time, but an upward trend is apparent for them, reaching 41 per cent in 2016.


    \begin{figure} %% original Figure 6
    \caption{Proportion of domestic students enrolled full-time, 1989--2016}\label{fig:proportion-of-domestic-students-enrolled-fulltime-19892016}
    \units{Per cent}
    \includegraphics[page= 6, width=1\columnwidth]{atlas/mycharts.pdf}
    \notewithsource{Full-time enrolment is defined as 75 per cent or more of a normal one year study load. Figures from 2001 onwards are based on enrolments at any time throughout the year; prior years are based on enrolments as at 31st March.}{{Department of Education and Training (various years-c);} Department of Education and Training (2018h)}
    \end{figure}



%
\subsection{The rise of off-campus study}\label{subsec:the-rise-of-off-campus-study}

Studying off-campus is not new in Australia. Originally carried out by correspondence, it is now mostly online. Since 2011, off-campus study has increased significantly, with one-in five domestic students studying off-campus. It is most popular with mature age students and postgraduates. School leaver undergraduates still overwhelmingly prefer on-campus study.\footnote{In 2016, 87 per cent of domestic bachelor-degree students who completed school in 2015 were enrolled on-campus. By contrast, 44 per cent of postgraduate coursework students aged 31-60 were enrolled on-campus: Department of Education and Training (various years-c).} `Multi-modal study' -- which mixes online and on-campus -- has also increased its share of enrolments (\Cref{fig:proportion-of-domestic-students-studying-off-campus-19892016}). By 2016, 13 per cent of students were enrolled on a multi-modal basis. Combining multi-modal and off-campus enrolments, a third of students have a substantial share of their instruction away from a university campus.

Although more students are enrolled off-campus, on- and off-campus are not always distinct forms of study. In 2016, about 45 per cent of students enrolled on campus reported doing half or more of their study online.\footnote{Social Research Centre/Department of Education and Training (2018b)} Some universities provide study centres for their online students, which offer computers, study rooms, and other services to assist their education.

Several factors are likely to explain these changes. Improved educational technology via the internet has made studying at home easier. This technological change coincided with increased demand for postgraduate study, often from people with significant work and family responsibilities. Not having to travel to campus makes study easier for this group. Funding policy changes let public universities create many more undergraduate student places, including in online courses (\Cref{subsec:teaching-grants-for-higher-education-institutions}).

    \begin{figure} %% original Figure 7
    \caption{Proportion of domestic students studying off campus, 1989--2016}\label{fig:proportion-of-domestic-students-studying-off-campus-19892016}
    \units{Per cent of students studying off campus}
    \includegraphics[page= 7, width=1\columnwidth]{atlas/mycharts.pdf}
    \notewithsources{{Open Universities Australia not included. }}{{Department of Education and Training (various years-c);} Department of Education and Training (2018h)}
    \end{figure}


%
\subsection{Other student characteristics }\label{subsec:other-student-characteristics}

Universities used to be places mainly for men. In the 1950s, only about one in five university students was female. But in the late 1950s women started a 50-year run of consistent annual gains in enrolment share, which has now stabilised at about 58 per cent. Women have been a majority of university students since 1987 (\Cref{fig:proportion-of-domestic-enrolments-by-gender-19492016}).

There are many reasons why this has happened: the overall social position of women has improved; entry into occupations dominated by women (teaching and nursing) now requires higher education qualifications; girls outperform boys at school; and young men have better-paying vocational education options than young women.

Despite their long-standing overall majority in higher education enrolments, women are still an official `equity' group in disciplines where they are a minority of students, such as engineering and information technology. Other equity groups include students with disabilities, Indigenous students, regional and remote students, non-English speaking background students who arrived in the last decade, and low socio-economic status (SES) students.


    \begin{figure} %% original Figure 8
    \caption{Proportion of domestic enrolments by gender, 1949--2016}\label{fig:proportion-of-domestic-enrolments-by-gender-19492016}
    \units{Per cent}
    \includegraphics[page= 8, width=1\columnwidth]{atlas/mycharts.pdf}
    \source{{} {Department of Education and Training (2014);} {Department of Education and Training (2018h)}}
    \end{figure}

Over the long term, higher education attainment has increased across all SES groups, high and low. For example, by 2001 children born in the 1970s who had parents in manual occupations had nearly five times the higher education attainment of the children born in the 1950s to manual workers. The higher education attainment level of children of `upper service' workers increased by about two-thirds in the same period.\footnote{Marks and Macmillan (2007)}

Although attainment increases continue across the socioeconomic spectrum, SES differences in university participation remain large. \Cref{fig:level-of-highest-education-enrolment-or-attainment-for-2024-year-olds-by-parents-occupation-2006-and-2016} reports on higher education participation or qualification attainment of people aged 20-24 in 2006 and 2016, classified according to their parent's occupation. In 2016, 25 per cent of the children of machinery operators, drivers and labourers were in higher education or had a degree. By contrast, 61 per cent of the children of managers and professionals were enrolled in or had completed higher education.

School outcomes limit options for young people from low SES backgrounds. They are less likely to finish school than high SES students, and much less likely to receive a high ATAR.\footnote{Norton (2016), p 186-188} Although school results vary significantly by SES, university participation rates differ little across different socio-economic backgrounds once ATAR is taken into account (\Cref{fig:atar-socioeconomic-status-and-university-participation-2012}).

The vast majority of domestic students speak English at home (84 per cent). The most common other home languages are Cantonese, Mandarin, Vietnamese and Arabic.\footnote{Department of Education and Training (2017k), table 9.2} Young people who speak languages other than English at home typically have much higher rates of university attendance than other young people (\Cref{fig:higher-education-participation-aged-1820-by-language-spoken-at-home-2016}). In communities speaking Eastern Asian languages, such as Cantonese or Mandarin, or Southern Asian languages, such as Hindi or Bengali, participation rates are well over twice that of their contemporaries who speak English at home.

    \begin{figure} %% original Figure 9
    \caption{Level of highest education enrolment or attainment for 20--24 year olds, by parent's occupation, 2006 and 2016}\label{fig:level-of-highest-education-enrolment-or-attainment-for-2024-year-olds-by-parents-occupation-2006-and-2016}
    \units{}
    \includegraphics[page= 9, width=1\columnwidth]{atlas/mycharts.pdf}
    \notewithsource{{Where parents had different occupations, the occupation requiring the highest skill level was used.}}{HILDA (2016)}
    \end{figure}


    \begin{figure} %% original Figure 10
    \caption{ATAR, socio-economic status and university participation, 2012}\label{fig:atar-socioeconomic-status-and-university-participation-2012}
    \units{}
    \includegraphics[page= 10, width=1\columnwidth]{atlas/mycharts.pdf}
    \notewithsource{The chart shows university participation rates by 2012 for young people who were in Year 9 in 2006. SES is defined by the father's occupation when the student was aged 15.}{Grattan analysis of Longitudinal Survey of Australian Youth: NCVER (2014).}
    \end{figure}



    \begin{figure} %% original Figure 11
    \caption{Higher education participation aged 18-20, by language spoken at home, 2016}\label{fig:higher-education-participation-aged-1820-by-language-spoken-at-home-2016}
    \units{Per cent}
    \includegraphics[page= 11, width=1\columnwidth]{atlas/mycharts.pdf}
    \notewithsource{Australian citizens only. For Pacific Island languages, ancestry was also used to better identify the group of interest.}{ABS (2017a)}
    \end{figure}



%
\section{International students }\label{sec:international-students}

International students have long studied at Australian universities, but in small numbers until the 1990s. Before then, international enrolments were usually wholly or partly subsidised by the Australian Government.\footnote{Meadows (2011); Megarrity (2007). A limited number of international students from developing countries still receive scholarships to study in Australia.} From 1986, universities were allowed to take international students at fees they set and kept. Double-digit growth rates quickly became the norm, promoted at times by migration policies favouring former international students.\footnote{Spinks and Koleth (2016) has an overview of migration policy changes.} Australian universities have also established branch campuses overseas, or partnered with education providers in other countries to deliver Australian courses.\footnote{See Ziguras and McBurnie (2015), especially \Cref{chap:higher-education-finance-the-macro-picture} and 7.}

In 2016, 391,500 international students were enrolled with Australian higher education providers (\Cref{fig:international-students-enrolled-in-australian-higher-education-19892016}). After a slowdown between 2010 and 2013, international student numbers recovered to reach a new peak in 2016. Data derived from immigration statistics, which only includes international students studying in Australia, shows strong growth continuing into 2018.\footnote{Department of Education and Training (2018e); Department of Home Affairs (2018d)}

A substantial minority of international students, 86,180 in 2016, are enrolled offshore, with about two-thirds in Singapore (30\%), Malaysia (21\%) or China (14\%). Counting only onshore students, about one in five students in Australian universities is an international student. Just under half of all international students are enrolled in commerce-related courses. Other major fields include engineering (12 per cent), information technology (10 per cent), health (8 per cent) and `society and culture' a broad field that includes humanities, social sciences and law (8 per cent).\footnote{See the notes to \Cref{fig:international-students-by-main-source-countries}.}

Australian universities enrol students from most countries, but the largest numbers come from Asian nations (\Cref{fig:international-students-by-main-source-countries}). Since 2000, enrolments from China have grown twenty-fold. They now make up over one-third of all international students in Australian higher education.


    \begin{figure} %% original Figure 12
    \caption{International students enrolled in Australian higher education, 1989--2016}\label{fig:international-students-enrolled-in-australian-higher-education-19892016}
    \units{Thousands}
    \includegraphics[page= 12, width=1\columnwidth]{atlas/mycharts.pdf}
    \notewithsources{{Figures from 2001 onwards are based on full year enrolments, prior years are based on enrolments as at 31st March.}}{{Department of Education and Training (2014);} {Department of Education and Training (2018h)}}
    \end{figure}




    \begin{figure} %% original Figure 13
    \caption{International students by main source countries}\label{fig:international-students-by-main-source-countries}
    \units{Thousands of onshore students in 1990, 2003 and 2016}
    \includegraphics[page= 13, width=1\columnwidth]{atlas/mycharts.pdf}
    \noteswithsource{Number of international enrolments in 1990, 2003 and 2016 by country of permanent home residence.}{{Department of Education and Training (various years-c). Annual figures are available from} {Department of Education and Training (2017k), and preceding years. }}
    \end{figure}



%
\chapter{The student experience and academic integrity}\label{chap:the-student-experience-and-academic-integrity}

%


This chapter examines the student experience. How likely are students to pass their subjects, and what does that say about academic standards? Are students satisfied with the quality of teaching? What proportion of students complete a course?

%
\section{Passing and failing }\label{sec:passing-and-failing}

For students, subjects passed is a key measure of success. For the higher education system's credibility, subjects failed matter too. Failed subjects are a sign that academic standards are being enforced, and that employers and others can trust that graduates have the expected knowledge and skills.

In some cases, that trust is not fully in place. Doubts about the quality of teacher education graduates prompted the NSW and Victorian governments to set minimum academic entry requirements for school leavers applying for teaching courses.\footnote{Department of Education and Training (Victoria) (2018); NSW Education Standards Authority (2018).} While admission processes are regulated for all courses (\Cref{sec:selecting-students}), government-set minimum admission requirements for particular courses are unusual. Nationally, teaching graduates also now need to pass a literacy and numeracy test.\footnote{AITSL (2015). To be successful, they need a score that would put them in the top 30 per cent of the population. The pass rate in 2017 was 92 per cent, but about 75 per cent at the worst performing university: Urban (2018).} No other field has such a test mandated by government.

In the international education market, government agency reports have argued that some universities put student recruitment and retention ahead of academic standards, with issues around inadequate English language requirements, cheating, and soft marking.\footnote{ICAC (2015); Victorian Ombudsman (2011)}

All universities set minimum English language requirements, but not necessarily at the level needed for successful study. One major language testing service, IELTS, recommends that for academic courses students should have a `band' of 7 or more on their 1 (lowest) to 9 (highest) scale.\footnote{IELTS (2018), p 12-13} Despite this, most universities set a minimum of band 6, and none has a general minimum above 6.5, although specific courses can require greater proficiency.\footnote{Desktop research of university websites conducted in July 2018.} One study found that students with band 6 English averaged over 200 errors per 1000 words of written text, compared to 35 for students with band 7 English.\footnote{Müller (2015), p 1211. Many of the errors for band 7 students were errors also made by native speakers of English.}

With international students paying high fees, both they and their universities have a strong need for them to succeed. This is likely to affect behaviour.

A survey of 14,000 students in eight Australian universities found that six per cent self-reported cheating on an assignment and/or an exam. International and domestic students have similar views on what constitutes cheating, but international students were twice as likely as domestic students to admit cheating.\footnote{Bretag\emph{, et al.} (2018)} Many more students reported sharing or trading notes or assignments. In a parallel survey of 1150 academic staff, more than two-thirds had suspected that an assessment task was not written by the student who submitted it. In more than 60 per cent of cases, the academic's knowledge of the student's language ability was a reason for their suspicion. Seven per cent of staff were aware of their students cheating in an exam.\footnote{Harper\emph{, et al.} Ibid.}

TEQSA has issued advice to universities on how to deal with `contract cheating', prompted by well-publicised cases of commercial websites offering cheating services.\footnote{TEQSA (2017a)} However, most cheats use work by other students or friends.\footnote{Bretag\emph{, et al.} (2018)} While cheating is sometimes sophisticated, it can also be amateurish, with document metadata (10 per cent) and off-topic content (36 per cent) being common grounds for academics suspecting malpractice.\footnote{Harper\emph{, et al.} Ibid.} In the 2018 Budget, TEQSA received additional funding to work on academic integrity issues.\footnote{TEQSA (2018a)}

While cheating is punished when detected, academics claim that international students benefit from `soft marking'. In a union survey, 28 per cent agreed with the proposition that they felt pressure to pass full-fee paying students whose work is not good enough.\footnote{NTEU (2018), p 20} Due to high levels of academic autonomy (\Cref{sec:what-is-distinctive-about-universities}), the same organisation recruits the students, designs the course, approves the curriculum, teaches the subjects, conducts the assessment and awards the qualification. With weak third-party overview, the university system relies on the integrity of academics and internal processes to maintain academic standards.\footnote{Courses must be periodically reviewed by independent people within the organisation, and student performance compared with that of students in comparable courses: Department of Education and Training (2015c). However, this does not fully deal with the potential conflicts of interest built into university internal processes.}

Whether soft marking exists or not, international students receive lower marks than domestic students. International students are substantially more likely to self-report average marks below 70 per cent (\Cref{fig:average-selfreported-marks-bachelor-degree-domestic-and-international-students-2016}). Academic performance between international and domestic students diverges over time. Between first and third year, domestic students become slightly less likely to report average marks below 70 per cent, while international students became more likely to do so.\footnote{Social Research Centre/Department of Education and Training (2018b)} We cannot say whether the results recorded in \Cref{fig:average-selfreported-marks-bachelor-degree-domestic-and-international-students-2016} fairly assess student work, or whether there are long-term trends in marks received.

    \begin{figure} %% original Figure 14
    \caption{Average self-reported marks, bachelor degree domestic and international students, 2016}\label{fig:average-selfreported-marks-bachelor-degree-domestic-and-international-students-2016}
    \units{Per cent}
    \includegraphics[page= 14, width=1\columnwidth]{atlas/mycharts.pdf}
    \noteswithsource{Bachelor pass degree students only. Survey conducted August-October 2016. A small number of students with no reported results or missing data omitted.}{Student Experience Survey, Social Research Centre/Department of Education and Training (2018b)}
    \end{figure}


International students are always more likely to fail subjects than domestic students (\Cref{fig:subject-fail-rates-for-bachelor-domestic-and-international-students-20062016}), which is consistent with their English-language difficulties and their reported marks. However, fail rates decrease after first year despite reduced average marks. International student fail rates went down in the years to 2010. These improved results were largely driven by universities that previously had high fail rates.

For domestic commencing bachelor-degree students fail rates went up as enrolments increased between 2009 and 2013. Since then, rates have declined slightly, with 14 per cent of subjects attempted failed in 2016 (\Cref{fig:subject-fail-rates-for-bachelor-domestic-and-international-students-20062016}). For continuing domestic students, fail rates have varied less. From 2013 to 2016, their subject fail rate was about 9.5 per cent.


    \begin{figure} %% original Figure 15
    \caption{Subject fail rates for bachelor domestic and international students, 2006--2016}\label{fig:subject-fail-rates-for-bachelor-domestic-and-international-students-20062016}
    \units{Fail rate of onshore international and domestic bachelor-degree students}
    \includegraphics[page= 15, width=1\columnwidth]{atlas/mycharts.pdf}
    \notewithsource{The calculation is subjects passed as a percentage of all subjects passed or failed. Withdrawn, yet-to-be-determined and missing subject results are ignored. Subjects dropped before the census date that triggers payment for the course, usually around a month after it starts, are not recorded in the data. This may bias the results in favour of domestic students, as for visa reasons international students must maintain their enrolment at least 75 per cent of the full-time level.}{{} {Department of Education and Training (various years-c)}}
    \end{figure}



%
\section{Student satisfaction with teaching}\label{sec:student-satisfaction-with-teaching}

Since the early 1990s, completing students at Australian universities have received a course experience questionnaire (CEQ). Core questions cover teaching, generic skills and overall satisfaction. As the CEQ is conducted after the course is finished it is an overview that combines views of many different subjects. Universities have their own surveys of individual subjects.

The initial CEQ surveys revealed low levels of satisfaction with teaching, but by the mid-1990s a positive trend had started. In a slow but steady way, each year more completing students indicated satisfaction with elements of university teaching (defined as choosing one of the top two points on a five-point scale). The surveyed elements included the level and helpfulness of feedback, teaching staff effort and effectiveness, whether students were motivated by teaching staff, and whether teaching staff made an effort to understand difficulties students were having.

\Cref{fig:average-student-satisfaction-with-teaching-19952017} shows average responses to these questions from completing bachelor-degree students. Though the trend until 2016 was consistently towards more satisfaction, it was not until 2007 that majority satisfaction was achieved. Due to survey methodology changes in 2010 and 2016 there are breaks in the time series.\footnote{In 2010 a mid-point in a five-point scale, which had previously been unlabelled, was described as `neither agree nor disagree' with the proposition being offered (for example, `the staff put a lot of time into commenting on my work'.) Possibly this means that satisfaction using the top two point definition was understated for previous years. However, CEQ respondents may have interpreted `neither agree nor disagree' as meaning `I have no opinion', while they could have interpreted the unmarked mid-point as representing a view, such as `middling' or `mediocre' but not unsatisfactory. In 2016, the survey was taken over by another organisation with different sampling and data collection methods.} Flat results in 2016 and 2017 suggest satisfaction levels are stabilising.

Possible reasons for long-term improvement in student satisfaction with teaching include research into teaching methods, teacher training, better information from student surveys, linking academic promotion to teaching performance, improved technology, increased regulation of standards, occasional government financial incentives, and more market competition.\footnote{For discussion of possible mechanisms for teaching improvement see Norton\emph{, et al.} (2013), \Cref{chap:higher-education-finance-the-macro-picture}; Probert (2015).}

Despite progress in improving student satisfaction with teaching, Australian students seem less satisfied than their British or American counterparts.\footnote{Social Research Centre/Department of Education and Training (2018a), p 25; Department of Education and Training (2018a), p 61}


    \begin{figure} %% original Figure 16
    \caption{Average student satisfaction with teaching, 1995--2017}\label{fig:average-student-satisfaction-with-teaching-19952017}
    \units{Per cent}
    \includegraphics[page= 16, width=1\columnwidth]{atlas/mycharts.pdf}
    \notewithsource{{Uses the good teaching scale in the CEQ.}}{{} {GCA (1995-2016Department of Education and Training (2018a); )}}
    \end{figure}



Since 2012, Australia has also had a national survey of current students, now called the Student Experience Survey. Its specific questions on curriculum, teaching and assessment differ from those in the CEQ, and there are other important differences in how the results are calculated and reported. These produce higher satisfaction scores than reported in \Cref{fig:average-student-satisfaction-with-teaching-19952017}.\footnote{The SES allows students to give a clear `mediocre' or `middling' response in options such as `some' and `fair'. All responses including these are coded on a 0 to 100 scale, and all students averaging 55 or more are classified as satisfied. See Social Research Centre/Department of Education and Training (2018a), p 81-82.} Between 2012 and 2017, current student satisfaction with teaching quality has varied in a narrow range between 79 and 82 per cent, with 80 per cent recorded in 2017.\footnote{Ibid., p 4} As with the CEQ, SES results suggest that student satisfaction with teaching has stopped improving.

In both the Course Experience Questionnaire and the Student Experience Survey, international students are less satisfied with teaching than are domestic students.\footnote{GCA (2015), table H; Social Research Centre/Department of Education and Training (2018a), p 6}

The 2017 Student Experience Survey included 58 NUHEPs. Average overall satisfaction with teaching in NUHEPs matched that of universities, but with a wide range of institutional results, from a low of 62 per cent to a high of 98 per cent of students satisfied with teaching quality.\footnote{Social Research Centre/Department of Education and Training (2018a), p 19-23} Teaching satisfaction results by university, NUHEP and field of education can be found at the Quality Indicators for Learning and Teaching (QIILT) website.\footnote{\href{http://www.qilt.edu.au}{www.qilt.edu.au}}

%
\section{Course attrition and completion }\label{sec:course-attrition-and-completion}

Not everyone who begins a degree finishes one. Of the students who started a bachelor degree in 2007, 22 per cent left without completing in the following nine years. Four per cent were still enrolled after nine years, and the remaining 74 per cent had completed a degree (not necessarily the one they started).\footnote{Department of Education and Training (2017b). The international student completion rate over 9 years was 78 per cent for those starting in 2007.} Analysis over shorter time periods suggests that completion rates are trending slightly down. After six years, 63 per cent of the students who commenced in 2011 had finished a course, down from 65 per cent who commenced in 2008 (\Cref{fig:completion-rates-for-the-20052013-commencing-bachelor-degree-cohorts}). The likely reason is that universities took more students with characteristics that put them at elevated risk of not completing, such as a low ATAR or part-time study.\footnote{Norton\emph{, et al.} (2018), p 24-29, 33-36} Short-term attrition rates -- students not enrolled in what would have been their second year -- stabilised for those commencing in 2014 and 2015.\footnote{Department of Education and Training (2017k), appendix 4.1. International student attrition is increasing but remains much lower than for domestic students: 9 per cent compared to 15 per cent.}

    \begin{figure} %% original Figure 17
    \caption{Completion rates for the 2005--2013 commencing bachelor degree cohorts}\label{fig:completion-rates-for-the-20052013-commencing-bachelor-degree-cohorts}
    \units{}
    \includegraphics[page= 17, width=1\columnwidth]{atlas/mycharts.pdf}
    \noteswithsource{Commencing domestic bachelor-degree students at public universities. Based on the first enrolment between 2005 to 2012 of each student. A common student identification number is used to track students over time, including moves between universities. Students did not necessarily complete their original course, but did complete a bachelor degree. }}{{} {Cherastidtham et al. (2018)}}
    \end{figure}
Attrition is not necessarily bad. Some students are just trying university without having clear goals. Many students who don't complete leave in first year, limiting their time and money costs.\footnote{Norton\emph{, et al.} (2018), p 8-12, 17-18} Students often also benefit from their time at university, even if they do not complete a degree. In a Grattan Institute survey, 40 per cent of people who had dropped would still begin their degree despite knowing what they know now, suggesting that the benefits outweighed the costs. But while attrition does not preclude positive effects, 10 per cent of students who drop out have studied for three years or more, accumulating substantial HELP debts.\footnote{Ibid., p 16-22}

A report from the Higher Education Standards Panel (\Cref{subsec:higher-education-standards-panel}) has recommended a range of measures to increase completion rates, which the Government has supported.\footnote{HESP (2018). These recommendations have significant overlap with Norton\emph{, et al.} (2018), \Cref{chap:research-in-higher-education-institutions} to 8.}

%


%
\chapter{The higher education workforce}\label{chap:the-higher-education-workforce}

%


Although employment in higher education remains attractive to many people, finding long-term secure work can be difficult. Most people doing academic work are on fixed-term or casual contracts.

%
\section{People employed in higher education }\label{sec:people-employed-in-higher-education}

Australia's universities employed 123,000 people on a permanent or fixed-term contract basis in 2017. The total number of university employees has increased in most years since the late 1990s, as \Cref{fig:number-of-permanent-and-fixedterm-staff-in-universities-19892017} shows. These statistics omit casually employed staff. In mid-2018 an estimated 94,500 people were employed on a casual basis, predominantly in teaching-only academic roles.\footnote{Data supplied by UniSuper, as of 30 June 2018. This is in imperfect count as it is based on patterns of superannuation contributions rather than direct information about employment status. It includes people with UniSuper accounts who no longer work in higher education, and omits others who work in higher education but use another superannuation fund. In 2016, 55 per cent of full-time equivalent casual staff were in teaching-only roles: Department of Education and Training (2017j), appendix 1.7.} In the non-university higher education sector staff numbers are reported on a full-time equivalent basis only.\footnote{For example, two part-time staff each working half the hours normally expected of a full-time staff member would be counted as one full-time equivalent.} Non-university higher education providers (NUHEPs) employed 3,200 full-time equivalent academics in 2016, with 38 per cent engaged on a casual basis.\footnote{TEQSA (2018c), p 42. There is no information on non-academic staff.}


    \begin{figure} %% original Figure 18
    \caption{Number of permanent and fixed-term staff in universities, 1989--2017}\label{fig:number-of-permanent-and-fixedterm-staff-in-universities-19892017}
    \units{Thousands of university staff}
    \includegraphics[page= 18, width=1\columnwidth]{atlas/mycharts.pdf}
    \source{{} {Department of Education and Training (2017j) and predecessor publications.}}
    \end{figure}
In 2017, 55,600 permanent or fixed-term contract staff had academic job classifications. Most academics are employed to teach and research, or to research only (the research workforce is discussed in \Cref{sec:the-research-workforce}). Teaching-only staff are the smallest (11 per cent) but fastest growing part of the academic workforce. Sometimes this growth reflects a genuine university commitment to teaching. But it also explained by reclassifying teaching and research staff whose research output is too low, and university enterprise agreement provisions allowing some casually-employed teaching staff to convert to on-going employment.\footnote{Probert (2013); Andrews\emph{, et al.} (2016)}

Academic staff are outnumbered by the 67,400 permanent or fixed-term contract employees with non-academic job classifications. There is a common belief that non-academic, who are often called `general' or `professional' staff, are growing as a share of the university workforce.\footnote{Forsyth (2014), \Cref{chap:higher-education-finance-the-micro-picture}.} For permanent and fixed-term contract employees, the non-academic share of the total workforce has been stable at around 57 per cent for at least the last 35 years, on a full-time equivalent basis.\footnote{Department of Education and Training (2017j), table 1.2; DEET (1993), p 137. For full-time staff only, the 57 per cent was also recorded in 1967: Commonwealth Bureau of Census and Statistics (1967).}

Although staff with academic titles are a minority of university employees, no data source gives a fully satisfactory account of higher education work. Both academic and non-academic work is outsourced, casual employees need to be counted (explored further in \Cref{sec:casual-and-fixed-term-academic-employment}), and job classifications do not necessarily describe daily duties.\footnote{For example, some people with academic titles are primarily administrators, while some non-academic staff are classified as `research only'.} Official university statistics describe non-academic staff according to where they work in the university, not their role.

Within these constraints, \Cref{fig:staff-by-area-of-university-2016} estimates the distribution of staff responsibilities in 2016. On a full-time equivalent basis, including casual staff, 47 per cent of the university workforce is in teaching or research. Twenty per cent of university employees are faculty support staff, 19 per cent work in central administration (which includes building and grounds maintenance), nine per cent are in learning support services (such as libraries), and five per cent work in student welfare services (such as health and counselling). In percentage terms, student welfare services staff are increasing at the fastest rate. Universities may be responding to significant minorities of students reporting mental health issues.\footnote{Cvetkovski\emph{, et al.} (2012). Health or stress issues are the most commonly-cited reason for considering leaving university without completing a course: Social Research Centre/Department of Education and Training (2018a), p 29.}

The 2016 Census provides detailed occupational data for people working in higher education, which can clarify some aspects of non-academic employment. Seventeen per cent of higher education staff are in clerical or administrative roles, 10 per cent are managers, and 5 per cent are in technical or trade occupations.\footnote{ABS (2017a). 55 per cent of higher education jobs in the ABS `professional' category are clearly academic, mainly `university lecturers and tutors', but apart from some occupations clearly labelled as `scientist' there is no clear category for research-only staff. Many higher education professional jobs are too ambiguous to allocate within universities to academic or non-academic functions. A statistician, doctor, lawyer, engineer or one of many other highly-educated professionals could be either.}


    \begin{figure} %% original Figure 19
    \caption{Staff by area of university, 2016}\label{fig:staff-by-area-of-university-2016}
    \units{Thousands of full-time equivalent staff}
    \includegraphics[page= 19, width=1\columnwidth]{atlas/mycharts.pdf}
    \noteswithsources{2016 data used because it includes a count of casual staff. Most data is based on staff employed by area of the university. However using data on staff functions some staff with academic ranks not engaged in teaching or research in a department or faculty have been reclassified as `faculty support staff', while some staff who are not in a department of faculty who are engaged in teaching or research are classified as such. Depending on university organisational structures, roles performed by `faculty support staff' could be the same as those in other non-academic categories. The figures are approximate due to data limitations.}{Department of Education and Training (2017j(2018h); )}
    \end{figure}



%
\section{Entry into the academic workforce}\label{sec:entry-into-the-academic-workforce}

Unsurprisingly, the main motivations for seeking academic work are intellectual. In a 2010 survey of Australian academics, more than 90 per cent agreed that opportunities for intellectually stimulating work, genuine passion for a field of study, and the opportunity to contribute to developing new knowledge drew them to academia.\footnote{Bexley\emph{, et al.} (2011), p 13} A survey of research students in the same year had similar findings. Developing knowledge and the interest and challenge of academic work were rated most highly as reasons to choose academic over other types of work.\footnote{Edwards\emph{, et al.} (2011), p 39}

Over time, the PhD has become the expected qualification for an academic. In 1991, fewer than half of all academics had a PhD; by 2017 nearly 70 per cent had one.\footnote{Department of Education and Training (2017j), table 4.2; DEET (1993), p 149} Some academic staff are enrolled in, but yet to complete, research qualifications.\footnote{Bexley\emph{, et al.} (2011), p 41} Most research students aspire to an academic job, although fewer see this as a realistic goal.\footnote{Edwards\emph{, et al.} (2011), p 22; Bentley\emph{, et al.} (2017), p 29-30}

The legal standards universities must meet support the practice of preferring higher qualifications. Teaching staff must have a PhD or a qualification level above the course they are teaching, or equivalent professional experience.\footnote{Department of Education and Training (2015c), A3.2} The latter exception recognises the subject matter expertise of people working outside universities, along with the insights professional practice can bring to teaching.

%
\section{Casual and fixed-term academic employment }\label{sec:casual-and-fixed-term-academic-employment}

Temporary academic jobs have become more common over time (\Cref{fig:casual-employment-as-a-share-of-the-fulltime-equivalent-academic-workforce-19882016}).\footnote{For more detail on employment conditions for casuals, see Andrews\emph{, et al.} (2016).} On a full-time equivalent basis, casual staff are 23 per cent of the university academic workforce. On a headcount basis, casually-employed academics are probably a majority of the academic workforce.\footnote{See May (2011) for an analysis based on 2010 data. The more recent figures cited earlier in this chapter are consistent with this conclusion, but due to data limitations certainty is not possible.} Most academic casuals are employed at the most junior academic rank.

Casual academic employment has benefits. For students, casual teaching staff can offer expertise -- often from professional practice -- that full-time academics lack. About a quarter of casual academic staff primarily work outside the university sector.\footnote{May\emph{, et al.} (2013), p 264} For aspiring academics studying for a PhD, casual teaching work helps them financially and gives them experience relevant to their future careers. About half of casually employed academics are also students, mostly in PhD programs.\footnote{Bexley\emph{, et al.} (2011), p 38; Strachan\emph{, et al.} (2012), p 59}

Yet while casual academic employment has benefits, for aspiring academics low pay and job insecurity can produce frustration. Some academics have been employed casually for long periods of time. Of the casual staff who responded to a 2017 union survey, 60 per cent reported having worked on a casual or sessional basis for six years or more.\footnote{Evans (2017)} An earlier survey found that one-in-five casual teaching staff taught at more than one higher education institution.\footnote{NTEU (2015a) p 22} Less than 20 per cent of respondents to the 2017 survey were satisfied with their work contract, although most did not want a full-time job.

Although casual employment causes difficulties for affected staff, it will remain common. Casual staff, who can be hired and fired more easily than continuing staff, help universities manage volatility in student numbers.\footnote{Andrews\emph{, et al.} (2016), p 1} As universities must compete for domestic and international students (\Cref{sec:distributing-student-places}), enrolment fluctuations will continue.

Casual employment also reflects the unusual schedule of universities. While some use trimester systems to teach for most of the year, two main semesters are more common. With two semesters the undergraduate teaching period runs for six months a year, with about two more months for exams. As funding policy does not support giving all academic staff research time in non-teaching periods, it is cheaper to hire staff only for teaching periods, rather than all year round.


    \begin{figure} %% original Figure 20
    \caption{Casual employment as a share of the full-time equivalent academic workforce, 1988--2016}\label{fig:casual-employment-as-a-share-of-the-fulltime-equivalent-academic-workforce-19882016}
    \units{Per cent}
    \includegraphics[page= 20, width=1\columnwidth]{atlas/mycharts.pdf}
    \sources{May et al. (2013), Department of Education and Training (2017j), appendix 1.17, DEET (1993)}
    \end{figure}

In 2017, 46 per cent of non-casual academic staff were on fixed term contracts -- slightly down on recent years because permanent staff numbers increased more quickly between 2013 and 2016 (\Cref{fig:permanent-and-fixedterm-academic-staff-19922017}). In 2017, growth in fixed-term staff numbers outpaced increase in permanent staff.

Most fixed-term academic staff, 52 per cent in 2017, are in research-only positions. This reflects the time-limited nature of much research funding. The major research agencies -- the Australian Research Council and the National Health and Medical Research Council (\Cref{subsec:public-research-funding-programs}) -- award project funding of only up to five years. Academic employment can be built on successive fixed-term contracts. In 2015, 40 per cent of these employees had been on fixed-term contracts for six years or more.\footnote{Grattan calculations from NTEU (2016), p 4.}

Within universities, permanent academic appointments on a teaching and research or research-only basis are seen as the ideal. But the way universities are organised and funded does not support this ideal. \Cref{chap:higher-education-finance-the-macro-picture} explores funding issues in detail.

    \begin{figure} %% original Figure 21
    \caption{Permanent and fixed-term academic staff, 1992--2017}\label{fig:permanent-and-fixedterm-academic-staff-19922017}
    \units{Thousands of staff}
    \includegraphics[page= 21, width=1\columnwidth]{atlas/mycharts.pdf}
    \notewithsource{Research-only staff holding jobs without academic classifications are excluded from this chart. Their numbers are declining, but in 2017 300 had permanent positions and 1916 had fixed-term contracts.}{Data special request from the Department of Education and Training}
    \end{figure}


%
\section{Pay and job satisfaction}\label{sec:pay-and-job-satisfaction}

For research students, pay is one of the few aspects of work life that they believe will be worse in academia compared to alternative careers.\footnote{Edwards\emph{, et al.} (2011), p 39} Academic salary ranges in 2016 are reported in Table 2. Some universities have higher base pay rates than others, creating an annual salary range of \$63,000 for an associate lecturer to \$194,000 for a professor. In practice, loadings are sometimes paid on top of these rates to make universities more competitive in the labour market. In a 2011 survey, 15 per cent of female and 22 percent of male academics reported receiving a loading.\footnote{Strachan\emph{, et al.} (2012), p 56}

Since 2001, the proportion of permanent and fixed-term academic staff at professor or associate professor level has increased from 21 per cent to 29 per cent. The other levels have fallen as a share of all academic staff (although the most junior has increased in full-time equivalent terms if casual staff are included).\footnote{Department of Education and Training (2018h)} With more senior staff, average academic salaries are increasing.

\begin{table} \caption{Academic pay ranges, 2018}

%%\begin{longtable}[]{@{}lll@{}}
%%\toprule
%%\textbf{Rank} & \textbf{Minimum} & \textbf{Maximum}\tabularnewline
%%\midrule
%%\endhead
%%Professor & \$168,000 & \$194,000\tabularnewline
%%Reader/Associate Professor & \$134,000 & \$166,000\tabularnewline
%%Senior Lecturer & \$111,000 & \$144,000\tabularnewline
%%Lecturer & \$91,000 & \$121,000\tabularnewline
%%Associate Lecturer & \$63,000 & \$98,000\tabularnewline
%%\bottomrule
%%\end{longtable}

\end{table}

Employer superannuation contributions of 17 per cent are common. Many universities are negotiating with the NTEU on new enterprise agreements in 2018, which are likely to change these figures.

Source: University websites and enterprise agreements

Various surveys of academics since the early 1990s have shown issues with academic job satisfaction.\footnote{Bentley\emph{, et al.} (2013b), p 30} In some surveys as few as half of academics are satisfied with their jobs. Australian academics also appeared less satisfied with their jobs than their peers in other countries.\footnote{Bentley\emph{, et al.} (2013a), p 247} In 2011, 69 per cent of academic staff were satisfied with their job overall.\footnote{Strachan\emph{, et al.} (2012), p 39} The most recent academic staff survey, conducted by the National Tertiary Education Union (NTEU) in 2017, found 77 per cent agreement with the proposition that `my work gives me satisfaction'. But this may be satisfaction with their core academic work, rather than their overall employment. The same survey showed dissatisfaction with workloads, promotion and senior management.\footnote{NTEU (2018), p. 16-17}

%


%
\chapter{Research in higher education institutions }\label{chap:research-in-higher-education-institutions}

Research is a central activity of universities. Without it, they could not use the `university' title (\Cref{subsec:research}). The research workforce and research output have both increased significantly over the last 20 years. Research can advance knowledge as an end in itself, address particular problems or practical goals, or be a mix of both.

%
\section{The research workforce}\label{sec:the-research-workforce}

The higher education research workforce overlaps with, but is not the same as, the academic workforce described in \Cref{chap:the-higher-education-workforce}. This is mainly due to research students, but also because academics with teaching-only positions are excluded, while research-only staff with non-academic classifications are included.

In 2017, 48,000 academics with permanent or fixed-term positions had a research or teaching and research function (figure 22). With other research staff added, the university research workforce was 50,250. With teaching and research roles under pressure from largely separate teaching and research funding (\Cref{subsec:public-research-funding-programs}), specialised research only academic staff have increased from 12 per cent of academic research staff in 1992 to 33 per cent in 2017. In recent years, this growth has been offset by declining numbers of non-academic staff with research roles. Their combined total of 17,950 is slightly below what it was in 2013.


    \begin{figure} %% original Figure 22
    \caption{Numbers of teaching and research, and research only staff, 1992--2017}\label{fig:numbers-of-teaching-and-research-and-research-only-staff-19922017}
    \units{Thousands of staff by responsibilities}
    \includegraphics[page= 22, width=1\columnwidth]{atlas/mycharts.pdf}
    \source{{Special data request from the Department of Education and Training. }}
    \end{figure}

On a full-time equivalent basis, including casual staff, the research or research and teaching workforce was 43,435 in 2016, more than a thousand below its 2013 peak. Casuals play a minor role in research compared to teaching, at 8 per cent of full-time equivalent research only or teaching and research positions in 2016.\footnote{Department of Education and Training (2018h)}

Not all of the university research workforce is funded by universities. A quarter of research outputs in 2015 included a contribution from a person who volunteered, such as an emeritus professor or an honorary fellow, or who was employed by another organisation, such as a medical research institute.\footnote{ARC (2015b), p 70. Academics on exchange from other universities are paid by a university, but not necessarily by the university they are currently working at.}

Research students are another source of unpaid or lowly-paid research labour. Their numbers have increased every year since 1990 (\Cref{fig:average-student-satisfaction-with-teaching-19952017}). Including overseas students, who make up 32 per cent of enrolments, there were 66,405 research students in 2016. High rates of part-time enrolment bring full-time equivalent numbers down to 45,470, which is less than in 2014 or 2015.\footnote{Department of Education and Training (2018h)} In 2016, 8903 PhDs were completed, along with 1638 masters by research degrees.\footnote{Department of Education and Training (2017k)}

On both a headcount and full-time equivalent basis, postgraduate research students outnumber university staff with research responsibilities. On ABS figures, 57 per cent of all research and development `person years of effort' in higher education institutions comes from postgraduate students.\footnote{ABS (2018e)}

International collaboration supplements Australia's research capacity. In 2016, Australian universities had nearly 5700 academic or research collaboration agreements with higher education institutions overseas, compared to just over 3000 in 2003.\footnote{Universities Australia (2016), p 5} In the 2008--2014 period, over half of Australian scientific publications had an international co-author.\footnote{UNESCO (2016), p 790}


    \begin{figure} %% original Figure 23
    \caption{Enrolments in research degrees, 1979--2016}\label{fig:enrolments-in-research-degrees-19792016}
    \units{Thousands of students}
    \includegraphics[page= 23, width=1\columnwidth]{atlas/mycharts.pdf}
    \sources{{} {Department of Education and Training (2014(2018h); )}}
    \end{figure}

%
\section{Research topics and types}\label{sec:research-topics-and-types}

Research spending is strongly skewed towards scientific and technology disciplines, and medical science in particular. Medical and health research accounted for 28 per cent of higher education research spending in 2016, with other sciences, engineering and IT together responsible for 44 per cent of expenditure. About 13 per cent of research spending is on the humanities and social sciences.\footnote{ABS (2018e). Humanities and social sciences includes economics and creative arts.} Within health research, cancer, infectious diseases and cardiovascular diseases get the greatest financial support.\footnote{NHMRC (2018). This funding goes to medical research institutes as well as universities.} In 2015, the government set research priorities. These are organised around themes rather than disciplines -- food, health, soil and water, transport, cybersecurity, energy and resources -- but are unlikely to change the historic discipline emphases of Australian university research.\footnote{Australian Government (2015). To date, a research topic being a priority theme has had little effect on success rates in competitive grant projects: ARC (2018b); ARC (2018a).}

Research is classified using OECD categories according to its approach to knowledge as well as its field. As \Cref{fig:research-journal-articles-with-one-or-more-australian-university-affiliated-author-19602017} shows, `pure basic research', which is the pursuit of knowledge without looking for long-term benefits other than advancing knowledge, has declined as a proportion of all research spending since 1994. In twenty years it went from 36 to 23 per cent of all research expenditure. With total university research spending increasing substantially (\Cref{subsec:other-sources-of-research-funding}), however, basic research spending increased significantly in real terms until 2012.\footnote{Basic research declined slightly in real terms between 2012 and 2014 and then increased to 2016, but still below 2012 levels: calculated from ABS (various years).} The shift has been to applied research, a category covering research aimed at finding possible uses for basic research or new ways of achieving specific and predetermined objectives. If government policy is successful, the trend towards applied research will continue, as research funding policy favours collaboration with industry (\Cref{subsec:public-research-funding-programs}).


    \begin{figure} %% original Figure 24
    \caption{Research spending by type, 1994--2016}\label{fig:research-spending-by-type-19942016}
    \units{Per cent}
    \includegraphics[page= 24, width=1\columnwidth]{atlas/mycharts.pdf}
    \source{ABS (various years)}
    \end{figure}

%
\section{Research outputs}\label{sec:research-outputs}

Publications are the main academic research output. These include books, book chapters, papers and articles. Until 2014, data was published on all these outputs, with articles the most common form of publishing.\footnote{Department of Education and Training (2015e). This was discontinued due to a change in research block grant funding policy (\Cref{subsec:public-research-funding-programs}). A 1997--2014 time series is available in Norton and Cakitaki (2016), p 40.} A full count of research publications is not available, but international research databases provide numbers of articles with authors affiliated to Australian universities. While these sources cover large numbers of academic journals, they are oriented to science more than the social sciences and humanities, and so omit some publications . The Scopus database, used in \Cref{fig:research-journal-articles-with-one-or-more-australian-university-affiliated-author-19602017}, identified nearly 67,000 research journal articles in 2017 with Australian connections. International co-authors mean that Australian universities cannot claim full credit for all these publications.

Like the enrolment figures in \Cref{chap:higher-education-students}, \Cref{fig:research-journal-articles-with-one-or-more-australian-university-affiliated-author-19602017} shows how universities have transformed themselves in recent decades. From 2004 to 2014, the annual increase in journal articles never fell below 6 per cent a year. In 2016 and 2017, growth slowed to below 2 per cent per year. Because it takes time to conduct research, and because academic journals are often slow to publish articles, research outputs as an indicator lag behind staff and funding numbers. Slow growth in 2016 and 2017 is likely to reflect the trends in staff numbers shown in \Cref{fig:numbers-of-teaching-and-research-and-research-only-staff-19922017}, which in turn are affected by funding issues explored in \Cref{subsec:public-research-funding-programs}.

    \begin{figure} %% original Figure 25
    \caption{Research journal articles with one or more Australian university affiliated author, 1960--2017}\label{fig:research-journal-articles-with-one-or-more-australian-university-affiliated-author-19602017}
    \units{Thousands of publications}
    \includegraphics[page= 25, width=1\columnwidth]{atlas/mycharts.pdf}
    \source{Elsevier (2018)}
    \end{figure}
Other research outputs have also increased. `Invention disclosures' -- a notification of a novel and useful device, material or method to a university's technology transfer office -- more than doubled, to 982, between 2000 and 2015.\footnote{Larkins (2011), p 218; DIIS (2017), table 1} The number of licences or options earning or expected to earn income increased more than fourfold over the same time period, to 2073.\footnote{ARC\emph{, et al.} (2002), p 97; DIIS (2017), table 1} The number of firms getting ideas or information for innovation for business increased between 2007 and 2015, before falling back (\Cref{sec:research-impact}). Academics also share their research expertise for free as part of their community engagement activities (\Cref{subsec:broad-social-responsibilities}).

%
\chapter{Higher education finance -- the macro picture}\label{chap:higher-education-finance-the-macro-picture}

%


This section discusses the various sources of finance in the higher education sector, and the relationships between them. These include funding for teaching (both from government and from students), for research (competitive and performance-based), and income support for students. It also looks at the overall financial position of higher education providers.

%
\section{Higher education as an industry}\label{sec:higher-education-as-an-industry}

As participation in higher education has increased, it has become a more economically significant industry. In 2016-17, the total revenue of organisations offering higher education qualifications was \$37.9 billion.\footnote{TEQSA (2018c), p 45} This includes income from teaching, research and other sources. Public universities dominate the industry, with revenues of \$30 billion in 2016.\footnote{Department of Education and Training (2017c)}

Over the last twenty years, higher education has become a significant export industry. The ABS estimates that international student fee revenue earned by all Australian higher education providers totalled \$7.6 billion in 2016 and \$9.3 billion in 2017.\footnote{ABS (2018c), table 9. For financial year, see ABS (2017c), table 9.} International students also contribute to other industries, through spending on living and other expenses while in Australia.

%
\section{Public spending on higher education -- overview}\label{sec:public-spending-on-higher-education-overview}

Public spending on higher education takes four main forms:

\begin{itemize}
\item
  Direct grants to higher education institutions, primarily for teaching (\Cref{sec:spending-on-and-by-students});
\item
  Student loans that are taken out by students but paid to higher education institutions on students' behalf (\Cref{subsec:lending-to-students});
\item
  Student income support payments, which are paid directly to students (6.3.3); and
\item
  Direct grants to higher education institutions primarily for research (\Cref{subsec:public-research-funding-programs}).
\end{itemize}

\begin{table} \caption{Overview of public higher education subsidies, 2017-18}

%%\begin{longtable}[]{@{}llll@{}}
%%\toprule
%%\textbf{Category} & \textbf{Sub-category} & \textbf{Description} & \textbf{Millions}\tabularnewline
%%\midrule
%%\endhead
%%\textbf{Teaching grants\\
%%(\textasciitilde{}\$7bn)} & Commonwealth Grant Scheme & Funding based on the number of supported domestic student places. See \Cref{subsec:teaching-grants-for-higher-education-institutions}. & \$7,000\tabularnewline
%%\begin{minipage}[t]{0.22\columnwidth}\raggedright
%%\textbf{Loan costs\\
%%(\textasciitilde{}\$1.2bn)}
%%
%%\textbf{(Distinct from new lending of \textasciitilde{} \$6.5 bn)}\strut
%%\end{minipage} & \begin{minipage}[t]{0.22\columnwidth}\raggedright
%%HECS-HELP, FEE-HELP,\\
%%OS-HELP,\\
%%SA-HELP \& Student Start-up Loans\strut
%%\end{minipage} & \begin{minipage}[t]{0.22\columnwidth}\raggedright
%%Costs are interest subsidies and doubtful debt. See \Cref{subsec:lending-to-students}.\strut
%%\end{minipage} & \begin{minipage}[t]{0.22\columnwidth}\raggedright
%%\$1,232\strut
%%\end{minipage}\tabularnewline
%%\textbf{Income support for students\\
%%(\textasciitilde{}\$1.8 bn)} & \vtop{\hbox{\strut Youth}\hbox{\strut Allowance}} & \vtop{\hbox{\strut Living expense support for students aged 16-24.}\hbox{\strut See \Cref{subsec:private-spending-by-students}}} & \$1,453\tabularnewline
%%& Austudy & \vtop{\hbox{\strut Living expense support for students aged 25 or more.}\hbox{\strut see \Cref{subsec:private-spending-by-students}.}} & \$356\tabularnewline
%%& Abstudy & Support for living expenses for Indigenous students. See \Cref{subsec:private-spending-by-students} & \$53\tabularnewline
%%\begin{minipage}[t]{0.22\columnwidth}\raggedright
%%\textbf{Research grants\\
%%(\textasciitilde{}\$3.3 bn),\\
%%}\strut
%%\end{minipage} & \begin{minipage}[t]{0.22\columnwidth}\raggedright
%%Competitive research grants
%%
%%Performance-based block research grants\strut
%%\end{minipage} & \begin{minipage}[t]{0.22\columnwidth}\raggedright
%%NHMRC -- see section \textbf{Error! Reference source not found.}\strut
%%\end{minipage} & \begin{minipage}[t]{0.22\columnwidth}\raggedright
%%\$627\strut
%%\end{minipage}\tabularnewline
%%& & Research training and general research funding. See \Cref{subsec:public-research-funding-programs}. & \$1,953\tabularnewline
%%& & ARC -- see \Cref{subsec:public-research-funding-programs}. & \$758\tabularnewline
%%\textbf{Other grants (\textasciitilde{}\$.4 bn)} & Other recurrent grants & For example: equity, national institutes, TEQSA. & \$429\tabularnewline
%%\textbf{Total} & & & \textbf{\$13,851}\tabularnewline
%%\bottomrule
%%\end{longtable}

\end{table}

\emph{Main sources:} \emph{Department of Industry (2017bDepartment of Education and Training (2018fDepartment of Social Services (2018b); ); ).}

\emph{\\
}Table 3 provides an overview of these funding streams. It omits minor grants from government agencies other than the Department of Education, short-term programs and legacy superannuation costs. In total, higher education-related government expenditure for 2017-18 was \$13.85 billion.

Eligibility for public funding depends in the first instance on the legal status of each higher education institution. Institutions that meet basic criteria can offer their students FEE-HELP loans (discussed in 6.3.2), which also makes their students eligible for income support (discussed in 6.3.3). But eligibility for other funding categories is largely restricted to institutions specifically listed in the \emph{Higher Education Support Act 2003}.

The `Table A' list contains all universities to which governments appoint council or senate members, plus the Australian Catholic University and the Batchelor Institute of Indigenous Tertiary Education. Though `public university' is not a legal concept, in common usage the term refers to Table A universities. They are entitled to receive money from all teaching and research funding schemes, as well as money under programs for disadvantaged students.\footnote{The main programs are the Higher Education Participation and Partnerships Program (HEPPP), which spent \$138.3 million in 2017-18, and the Disability Support Program, which spent \$7.4 million in 2017-18: Department of Education and Training (2018f), p 48.}

Table B contains Bond University, the University of Notre Dame, MCD University of Divinity, and Torrens University. This listing entitles them to FEE-HELP and research funding, and makes them eligible for teaching subsidies in courses deemed to be national priorities. In practice, only Notre Dame receives these subsidies.

Table C gives FEE-HELP to students in higher education providers operating in Australia but controlled from overseas. Carnegie Mellon University students benefit from Table C listing.

An overview of different entitlements to public support is in Table 4.

In contrast to the process for higher education providers accessing FEE-HELP, there are no rules determining which institutions are on Tables A, B or C, and no application process. Entitlements are largely a matter of history and politics.

\begin{table} \caption{Overview of funding eligibility}

%%\begin{longtable}[]{@{}llllll@{}}
%%\toprule
%%\endhead
%%\begin{minipage}[t]{0.14\columnwidth}\raggedright
%%\textbf{Funding Type}\strut
%%\end{minipage} & \begin{minipage}[t]{0.14\columnwidth}\raggedright
%%\textbf{Table A}\strut
%%\end{minipage} & \begin{minipage}[t]{0.14\columnwidth}\raggedright
%%\textbf{Table B}\strut
%%\end{minipage} & \begin{minipage}[t]{0.14\columnwidth}\raggedright
%%\textbf{Table C}\strut
%%\end{minipage} & \begin{minipage}[t]{0.14\columnwidth}\raggedright
%%\begin{quote}
%%\textbf{Other HE providers}
%%\end{quote}\strut
%%\end{minipage} & \begin{minipage}[t]{0.14\columnwidth}\raggedright
%%\textbf{OUA\^{}}\strut
%%\end{minipage}\tabularnewline
%%FEE-HELP loans & & & & &\tabularnewline
%%Commonwealth supported places and HECS-HELP loans & & (provided the place is in a `national priority' category')* & (provided the place is in a `national priority' category')* {[}none in 2018{]} & (provided the place is in a `national priority' category')* & \vtop{\hbox{\strut \textbf{\textasciitilde{}}}\hbox{\strut Indirectly via universities delivering award programs}}\tabularnewline
%%Research block grants & & & & &\tabularnewline
%%Research training places & & & & &\tabularnewline
%%ARC competitive grants & & & & &\tabularnewline
%%NHMRC grants & & & None in practice & &\tabularnewline
%%Student income support & & & & If provider offers HELP &\tabularnewline
%%\bottomrule
%%\end{longtable}

\end{table}

%
\section{Spending on and by students}\label{sec:spending-on-and-by-students}

Spending on students, for both their tuition costs and living expenses, comes from a mix of public and private sources.

%
\subsection{Teaching grants for higher education institutions}\label{subsec:teaching-grants-for-higher-education-institutions}

The single largest source of public subsidy for higher education is the Commonwealth Grant Scheme (CGS). Public universities and their students have the main entitlements to CGS funding, as Table 4 shows. Although the funding legislation does not specifically say that the CGS is for teaching, its Constitutional foundation as a `benefit to students' (\Cref{sec:the-rise-of-commonwealth-authority}) and its link with student numbers strongly imply that supporting teaching is its main purpose.

CGS funding for each higher education provider is principally calculated according to its number of Commonwealth-supported places. One `place' is equivalent to the number of subjects normally taken by a full-time student (equivalent full-time student load, or EFTSL, has the same meaning as a place). The CGS payment per place depends on its discipline. All disciplines are allocated to one of eight funding `clusters', each of which has its own Commonwealth funding rate, called a Commonwealth contribution (these rates and the separate student contribution rates are discussed in \Cref{sec:funding-per-student}). These rates are indexed annually to CPI inflation.

For each cluster, the number of Commonwealth-supported student places is multiplied by its Commonwealth contribution funding rate. These cluster sub-totals are added together to calculate the core CGS funding for each higher education provider. Extra payments for regional locations, medical students and preparatory courses paid out of the CGS add to the total, but these are a small part of overall spending. Per student funding rates and the number of places are the key drivers of total funding. \Cref{subsec:distributing-government-supported-places} outlines how the number of places is set.

For 2018 and 2019, total CGS funding for bachelor-degree places at most universities has been set at no more than 2017 levels. Each university's CGS funding will therefore be the amount payable for student places delivered or the maximum amount set by government, whichever is the lower. In practice, universities will not be paid their CPI indexation, or Commonwealth contributions for additional students beyond 2017 numbers.\footnote{Based on a freedom of information request, the ABC reported university-level funding implications: Conifer (2018). For the amounts allocated to each university, see Department of Education and Training (2018d). For detail on how the limits were imposed see Norton (2017b).} The Government says it will increase funding in line with population growth from 2020, but only for universities that meet yet-to-be-determined performance criteria.\footnote{Birmingham (2017). Population growth for people aged 18-64 years.} The number of postgraduate Commonwealth supported places has also been cut.

Due to these policy changes, total CGS spending in 2018 will fall in real terms for the first time since 2003 (\Cref{fig:core-teaching-grant-funding-19892018}). However, spending grew by 80 per cent in real terms in the intervening 15 years. The single most important reason for this growth was the easing, and then abolition, of controls on the number of bachelor-degree Commonwealth supported places (\Cref{subsec:distributing-government-supported-places}). Effectively, capping total CGS funding put these controls back in place.

The CGS is the main direct grant for teaching, but Commonwealth cash flow to universities for teaching also includes HELP loans, which are discussed in the next section.


    \begin{figure} %% original Figure 26
    \caption{Core teaching grant funding, 1989--2018}\label{fig:core-teaching-grant-funding-19892018}
    \units{\$2018 billion}
    \includegraphics[page= 26, width=1\columnwidth]{atlas/mycharts.pdf}
    \noteswithsource{{Operating grant figures are used prior to 2005, less HECS charges and research funding subsequently distributed separately. Adjusted using CPI.}}{{Data from the Department of Education and Training. }}
    \end{figure}



%
\subsection{Lending to students}\label{subsec:lending-to-students}

Since 1989, the Commonwealth Government has lent higher education students money to pay for their courses. The loans are called income contingent because repayments depend on the debtor's income. Students or former students pay a share of their income through the tax system each year until the debt is fully paid off. In 2018-19, they begin repaying 2 per cent of all their income of \$51,957 or more. The share of income repaid increases with earnings, up to a maximum of 8 per cent.\footnote{ATO (2018). Threshold information from 1989 to 2017 is available in Norton and Cherastidtham (2016), appendix A.} The repayment system will change from 2019-20, with a new initial repayment threshold of \$45,881, and repayments of between 1 and 10 per cent of income, depending on earnings.\footnote{\emph{Higher Education Support Legislation Amendment (Student Loan Sustainability) Act 2018}.} HELP debtors living overseas were exempt from repayment until the 2016-17 tax year. They must now report their worldwide income to the Australian Taxation Office (ATO) and repay using the same thresholds and rates as debtors in Australia.

%
\subsubsection{Student loan schemes}\label{subsubsec:student-loan-schemes}

Australia's income-contingent loan scheme, initially known as HECS (Higher Education Contribution Scheme), was renamed HELP (Higher Education Loan Program) in 2005. Since the scheme's inception, other income-contingent loan schemes have proliferated. The most direct descendant of the original scheme, HECS-HELP, lends money to pay student contributions -- the student share of the funding rate for a Commonwealth-supported place (see \Cref{subsec:commonwealth-supported-students}). An estimated \$4.7 billion will be lent through HECS-HELP in 2018-19.\footnote{Estimate provided by the Department of Education and Training.}

Currently students can borrow unlimited amounts through HECS-HELP, although annual course charges are capped (\Cref{subsec:commonwealth-supported-students}). From 2020, there will be a maximum amount of outstanding debt for tuition expenses. This will be \$150,000 for medicine, science and dentistry students, and \$104,440 for other students.\footnote{As indexed to inflation. The cap will include HECS-HELP, FEE-HELP, VET FEE-HELP and VET Student Loans. It will not include loan fees, indexation, OS-HELP, SA-HELP or Student Start-up Loans.} HECS-HELP borrowing prior to 2020 will not be counted towards the cap.

The FEE-HELP scheme lends money to domestic full-fee students -- mainly postgraduate coursework students and students outside the public universities. An estimated \$1.6 billion will be lent through FEE-HELP in 2018-19, of which \$1 billion will be borrowed by postgraduate students.

FEE-HELP borrowers have always had a lifetime borrowing limit for tuition expenses (for 2018, \$127,992 for medicine, dentistry and veterinary science; \$102,392 for all other courses).\footnote{Borrowing under the VET Student Loans scheme, and its predecessor VET FEE-HELP, is also included in this cap.} From 2020, FEE-HELP loans will be incorporated into the overall borrowing cap, which could include HECS-HELP debt incurred in a previous course. However, there will be no lifetime limit on borrowing. HELP debtors who repay previous debt will be able to borrow again, up to the limit.

OS-HELP helps finance overseas study by Commonwealth supported students. How much students can borrow under OS-HELP depends on circumstances, but it is up to \$9063 for a six-month period.\footnote{The standard OS-HELP cap per six-month period is \$6665 or \$7998 for students going to an Asian country. An additional \$1065 is available for language study in preparation for going to an Asian country.} Students can borrow twice under OS-HELP. An estimated \$124 million will be lent under OS-HELP in 2018-19. SA-HELP supports a separate charge for student services and amenities. Its maximum annual loan is \$298 in 2018 (the price limit on the student amenities fee). An estimated \$117 million will be lent under SA-HELP in 2018-19.

In 2016 the Government converted a previous student start-up scholarship -- a lump sum grant for students receiving student income support -- into an income contingent loan, the Student Start-up Loan.\footnote{Department of Human Services (2016)} The loan is available to students on education-related income support (\Cref{subsec:private-spending-by-students}). Eligible students can receive lump sums of \$1055 up to twice a year. The money is intended to assist them with textbooks, relocation expenses and other education-related costs. Although the Student Start-up Loan is separate from HELP, its repayment provisions mirror those applying to HELP debt. Debtors begin repaying start-up loan debt once they finish repaying HELP debt.


\subsubsection{
HELP borrowing and repayment trends }\label{help-borrowing-and-repayment-trends}

Annual lending through HELP for higher education reached \$6.4 billion in 2017-18, having doubled in real terms since 2008 (\Cref{fig:help-lending-and-repayment-20052018}). Total HELP lending accelerated more quickly until 2016, due to the vocational education VET FEE-HELP scheme. VET FEE-HELP lending was separate to the other schemes, but it shared borrowing caps and repayment systems with the higher education HELP loans. VET FEE-HELP was replaced with VET Student Loans in 2017, with annual lending now much lower than before. All the money borrowed is consolidated into a single HELP debt managed by the ATO.\footnote{As of August 2018, there was a bill before the Parliament to create a separate loan scheme for vocational education.}

From 2012 to 2015, HELP repayments stalled, as \Cref{fig:help-lending-and-repayment-20052018} shows. This was due to weak graduate employment and wages (see \Cref{chap:benefits-for-graduates}), and increasing numbers of HELP debtors falling below the initial income threshold below which no repayment is required.\footnote{Norton and Cherastidtham (2016), \Cref{chap:the-student-experience-and-academic-integrity}} Since 2016 repayments have started growing again. In 2016-17, the most recent year available, the ATO collected an estimated \$2.4 billion in HELP repayments. It is not clear whether the lower initial threshold applying from 2019-20 will have much effect on total annual HELP repayment revenue. While additional lower-income debtors will make repayments, these will all be under \$550 a year. Because of changes to the higher income repayment thresholds, most debtors will repay less each year.\footnote{Norton and Cherastidtham (2018)}


    \begin{figure} %% original Figure 27
    \caption{HELP lending and repayment 2005--2018}\label{fig:help-lending-and-repayment-20052018}
    \units{\$2017-18 billion}
    \includegraphics[page= 27, width=1\columnwidth]{atlas/mycharts.pdf}
    \noteswithsources{HELP lending by financial year is calculated by taking the average of the relevant two calendar years. Because the government does not routinely publish full HELP repayment data, the total repayment for 2014-15 to 2016-17 is projected based on 2013-14 repayment and the growth in compulsory repayment reported by the ATO. The ATO's figures are usually revised upward as late tax returns are submitted.}{{} {Department of Education and Training (2015baCommissioner of Taxation (various years); ); ); data supplied by the Department of Education and Training}}
    \end{figure}



%
\subsubsection{HELP's costs}\label{subsubsec:helps-costs}

Government financial statements do not present a clear account of HELP's annual costs.\footnote{The main annual reporting is in the Department of Education and Training's Portfolio Budget Statements, for example Department of Education and Training (2018f), p 49-50. It is an un-itemised estimate of the cost of that financial year's lending. While reporting of HELP's finances could be significantly improved, the international accounting standards used in the Commonwealth Budget are an obstacle to clarity. Some of the complexities are discussed in Parliamentary Budget Office (2016).} \Cref{fig:annual-cost-of-help-19942017} provides Grattan Institute estimates of HELP's component costs, along with offsetting revenue.

HELP's largest cost is debt not expected to be repaid, commonly called doubtful debt. Debt becomes doubtful when debtors are not expected to make sufficient repayments during their life to clear what they owe. Eventually, on death, remaining debt is written off. To date, only a very small percentage of HELP debtors have died without repaying.\footnote{As of 30 June 2015, 0.35 per cent of those who have ever taken out a HELP debt.} Doubtful debt costs are therefore estimates, given what we know about current HELP debtors and their repayment prospects. The Government estimates that 17 per cent of new HELP debt issued will not be repaid.\footnote{Department of Education and Training (2018f), p. 50} \Cref{fig:annual-cost-of-help-19942017}'s \$4 billion doubtful debt cost for 2016-17 includes further write downs of money previously lent, as well as lending during 2016-17.

HELP's other major cost is an interest subsidy. This occurs because the Government borrows money in the bond markets, which it re-lends to students at the typically lower CPI inflation rate. Taxpayers pay the cost of the difference between the two numbers. For 2016-17 this net interest bill is an estimated \$280 million.\footnote{This figure is an estimate because the government does not specifically borrow for HELP. The notes to \Cref{fig:annual-cost-of-help-19942017} explain the assumptions behind this estimate.} The Grattan Institute has estimated the annual net interest bill on the HELP debt each year from 1994 (\Cref{fig:annual-cost-of-help-19942017}). Bond rates that are well below their long-term average kept down the interest cost for 2016-17, despite the increased total debt discussed below.


    \begin{figure} %% original Figure 28
    \caption{Annual cost of HELP, 1994--2017}\label{fig:annual-cost-of-help-19942017}
    \units{\$2017 billion}
    \includegraphics[page= 28, width=1\columnwidth]{atlas/mycharts.pdf}
    \noteswithsources{{This chart cannot be compared directly to the Department of Education and Training's portfolio Budget papers (}Table 3{) due to a different methodology. The most important difference is that this chart examines the cost of the historical stock of HELP debt, while the Budget papers incorporate estimates of the future cost of each year's lending. In this figure, addition to doubtful debt is the increase in total doubtful debt since the previous year. The interest cost is calculated as the difference between the ten-year Commonwealth bond rate and the CPI indexation rate, multiplied by the level of outstanding debt. Loan fees include FEE-HELP, VET-FEE-HELP and VET Student Loans, based on estimates of loan fee-liable lending. Indexed using CPI.}}{{Based on} {Department of Education and Training (2015b), annual reports for portfolios responsible for higher education, information supplied by the Department of Education and Training. }}
    \end{figure}

The `early payment bonus' reduced HELP debt by more than the amount repaid. Before it was abolished at the end of 2016, HELP debts were reduced by 5 per cent more than the payment. This is counted as a cost to the government. The `upfront discount' applied to student contributions paid directly to universities. Before it was abolished at the end of 2016 upfront payments received a 10 per cent discount. The Government compensated universities for the discount.



Offsetting these costs are revenues from loan fees. Full-fee undergraduates at NUHEPs must pay a 25 per cent loan fee if they take out a FEE-HELP loan.\footnote{Prior to 2019, full-fee students at private universities also paid the 25 per cent loan fee.} For example, if a full-fee undergraduate student borrows \$10,000 the Government records a debt of \$12,500. The loan fee is source of revenue for HELP (\Cref{fig:annual-cost-of-help-19942017}), although much of this is in recent years was due to vocational education lending.\footnote{The loan fee for vocational education courses is 20 per cent.}

%
\subsubsection{Total HELP debt}\label{subsubsec:total-help-debt}

At 30 June 2017, HELP debtors owed the Commonwealth Government \$55.4 billion, including money lent to vocational education students. Since 2000, the Government has published the HELP debt's `fair' value (shown in \Cref{fig:help-debt-including-fair-value-19892017}). This is an estimate of how much the HELP debt is worth to the Government. At 30 June 2017, the HELP debt's fair value was \$35.9 billion, \$19.5 billion less than its nominal value.\footnote{Department of Education and Training (2017a), p. 128. This large write-down is affected by substantial predicted losses on the VET FEE-HELP scheme.} The main cause of the lower fair value is doubtful debt.


    \begin{figure} %% original Figure 29
    \caption{HELP debt (including fair value), 1989--2017}\label{fig:help-debt-including-fair-value-19892017}
    \units{\$2017 billion}
    \includegraphics[page= 29, width=1\columnwidth]{atlas/mycharts.pdf}
    \notewithsources{{Indexed using CPI}}{{} {Department of Education and Training (2017a) and preceding publications. }}
    \end{figure}



%
\subsection{Private spending by students}\label{subsec:private-spending-by-students}

Private higher education spending by students has increased, reaching \$13.2 billion in 2016, although 40 per cent of this revenue still comes from the government through HELP loans (\Cref{fig:teaching-revenue-from-students-19972016}). In 2016, domestic students contributed slightly more revenue than international students. As enrolment trends show weak domestic but strong international student growth (\Cref{chap:higher-education-students}), and t the ABS reports rapid increases in education exports (\Cref{sec:higher-education-as-an-industry}), international students are likely now be the largest source of revenue from students. When the CGS is added, total income for domestic students would still exceed international student revenue. The fees paid by different types of student are discussed in \Cref{chap:higher-education-finance-the-micro-picture}.


    \begin{figure} %% original Figure 30
    \caption{Teaching revenue from students, 1997--2016}\label{fig:teaching-revenue-from-students-19972016}
    \units{\$2018 billion}
    \includegraphics[page= 30, width=1\columnwidth]{atlas/mycharts.pdf}
    \notewithsource{{Does not include fees or charges paid by students for non-teaching services such as student amenities or accommodation. Indexed using CPI.}}{{} {Department of Education and Training (various years-a)}}
    \end{figure}



%
\subsection{Student income support}\label{subsec:student-income-support}

Although most domestic students finance their tuition expenses through the CGS and/or HELP, living costs while studying are mainly privately financed. In a 2017 survey of student finances, a third of undergraduates reported receiving a student income support payment, and 6 per cent reported another Centrelink payment. Nearly 80 per cent of undergraduates were in paid work, and 62 per cent of full-time undergraduates reported financial support from their family or partner.\footnote{Universities Australia (2018a), p 19, 24}

In 2017, full-time domestic undergraduates reported a median annual income of \$18,300, and expenditure of \$14,200, of which \$1,300 was study-related. Part-time undergraduate students had more finely-balanced finances, with median annual income and expenditure both at \$29,000.\footnote{Ibid., p 18, 27}

Government student income payments are generally available to full-time undergraduates who are Australian citizens or permanent residents. Students in some postgraduate coursework programs are also eligible for student income support.\footnote{For historical student income support programs and statistics see Daniels (2017).} Eligibility was restricted in 2018 to students at higher education providers that also offer HELP loans.\footnote{See changes to the \emph{Student Assistance (Education Institutions and Courses) Determination 2009 (No. 2)}, Schedule 2. Institutions whose students can receive student income support are listed in Appendix A. Institutions whose students cannot receive student income support are listed in Appendix B.} Although this will only affect a small number of students, it is a significant change. Since the current needs-based student income support system was established in 1974, it had been the only government payment that treated students equally, regardless of their choice of higher education provider.

The biggest student income support scheme is Youth Allowance. In June 2016, 177,378 higher education students received Youth Allowance.\footnote{Department of Social Services (2017b). These are the latest figures in which higher education students are specifically reported.} The estimated cost of Youth Allowance for higher education students in 2017-18 was \$1.45 billion.\footnote{Expenditure on Youth Allowance for higher education students is not specifically reported. This estimate assumes that the share of Youth Allowance recipients who are in higher education is the same as in June 2016, and that they receive payments proportionate to their share of the Youth Allowance population. School and vocational education students also receive Youth Allowance. Funding aggregates from Department of Social Services (2018b), p. 47.}

Students whose parents earn \$52,706 a year (2016-17) or less are entitled to the full at-home Youth Allowance rate of \$293.60 a fortnight. There are higher rates for students who live independently. The payment reduces if parents earn more than \$52,706, or if the student earns more than \$437 a fortnight.\footnote{Department of Human Services (2018)}

Youth Allowance recipients are not subject to the parental income test if they meet various criteria indicating independence from their parents or if they turn 22. This makes students in high-income households eligible for Youth Allowance, so long as their personal income is low.

There are two other smaller income support programs. Austudy is for students aged 25 or older, and supported 32,392 students in June 2016.\footnote{Department of Social Services (2017a)} Its estimated cost for 2017-18 was \$356 million.\footnote{Expenditure on Austudy for higher education students is not specifically reported by the Government. See the explanation for Youth Allowance.} ABSTUDY is for Indigenous students, and in 2015-16 supported 4815 higher education students.\footnote{ANAO (2017), p. 16. Some Indigenous students receive Youth Allowance or Austudy instead: Department of Social Services (2018a).} Its estimated cost for 2017-18 is \$52.7 million.\footnote{Expenditure on ABSTUDY for higher education students is not specifically reported by the Government. See the explanation for Youth Allowance.}

As noted, students receiving Youth Allowance, Austudy and Abstudy can all apply for up to \$2110 a year in loans, on top of their benefits (\Cref{subsec:lending-to-students}). \$159.5 million was lent in 2017-18.\footnote{Department of Social Services (2018b), p. 47}

As well as these generally needs-based income support schemes, merit-based stipends are available for research students through the Research Training Program (RTP). These are funded by the Commonwealth but allocated by universities, which decide on a stipend amount between a 2018 base rate of \$27,082 and a maximum of \$42,307.\footnote{Department of Education and Training (2018g)} This new program gives universities flexibility on student numbers and funding rates. Statistics on how they spent RTP money should be available in the second half of 2018. In 2016, 12,142 students received funding under the predecessor Australian Postgraduate Awards.\footnote{Department of Education and Training (2018c)}

%
\section{Spending on research}\label{sec:spending-on-research}

Like teaching, research is funded from a mix of public and private sources.

%
\subsection{Public research funding programs}\label{subsec:public-research-funding-programs}

Universities receive two broad types of research grant. Competitive grants provide money for specific projects, centres or individuals through fellowships. Performance-based block research grants are determined by formulas that are primarily based on success in attracting other funding. `Block' funding means that universities have discretion on its precise use, within the broad parameters of the funding scheme. Though all universities can apply for research grants, the Group of Eight or sandstone universities (listed in appendix A) receive most research funding.

%
\subsubsection{Competitive grants}\label{subsubsec:competitive-grants}

The Australian Research Council (ARC) and the National Health and Medical Research Council (NHMRC) are the main sources of competitive grant funding.\footnote{Department of Education and Training (2018b) providers a full list of competitive grant sources recognised for performance grant funding.} Eligibility for ARC grants is largely restricted to universities. Eligibility for NHMRC grants is broader and includes medical research institutes and hospitals, but universities are the main recipients. \Cref{fig:arc-nhmrc-and-block-research-grants-to-universities-20012018} shows trends in ARC and NHMRC university funding. In real terms, the ARC's funding has declined since 2013 after a period of significant growth. It spent \$758.1 million on research grants in 2017-18.\footnote{Department of Education and Training (2018f), p 121} Estimated NHMRC university grants for 2017-18 were \$626.9 million.\footnote{Sourced from Department of Industry (2017b). NHMRC figures not available at the time of writing.}

For universities, the significance of these competitive grants goes beyond the money they receive. Their level of grant income contributes to their performance-based block research funding (see next section). For academics and their institutions, winning competitive grants brings prestige as well as money.

Winning an ARC grant is difficult. Projects are assessed by academic experts in the relevant field, so that only the highest quality projects are supported. For Discovery Project grants, aimed at supporting excellent basic and applied research, 18.9 per cent of the 3136 applications for funding in 2018 were approved. Funded projects receive between \$30,000 and \$500,000 a year for up to five years. A researcher applying for a Discovery grant must show a track record in research publications and evidence of research quality, including whether the proposal addresses a significant problem and will advance knowledge.\footnote{ARC (2017b)} Group of Eight universities won more than 60 per cent of new Discovery Project money for 2018.\footnote{ARC (2018a)}

Linkage Projects encourage collaboration between higher education providers and other organisations, including industry. Partner organisations are required to contribute to the project. Linkage grants reflect a government emphasis on useful knowledge and innovation. These grants are one reason why research activity has shifted towards applied research (\Cref{sec:research-topics-and-types}). Because they involve external partners, Linkage grant proposals are more difficult to organise and many fewer applications are made (only 417 in 2017) than for Discovery grants, despite their higher success rate -- 32 per cent in 2017. Group of Eight universities also dominate this pool, securing 63 per cent of new funding for 2017.\footnote{ARC (2018b)}


    \begin{figure} %% original Figure 31
    \caption{ARC, NHMRC and block research grants to universities, 2001--2018}\label{fig:arc-nhmrc-and-block-research-grants-to-universities-20012018}
    \units{\$2018 billion}
    \includegraphics[page= 31, width=1\columnwidth]{atlas/mycharts.pdf}
    \noteswithsources{ARC and block grants are for the financial year ending 30 June, NHMRC is calendar year. Non-university funding has been excluded from the NHMRC figure. Indexed using CPI.}{Department of Industry (2017bDepartment of Education and Training (2018f); )}
    \end{figure}



For NHMRC project grants, application success rates declined in from 23 per cent in 2010 to 16 per cent in 2017.\footnote{NHMRC (2017)} The main criteria for assessing projects are scientific quality, significance and/or innovation, and the researchers' track record in research output and impact. There is no maximum amount of project funding, and projects can be funded for up to five years. The NHMRC also offers program funding for broad areas of health research expected to `contribute new knowledge at a leading international level'. Once again, the Group of Eight universities dominate. They secured more than 80 per cent of grant payments in 2015.\footnote{NHMRC (2015)}

%
\subsubsection{Performance-based block grants}\label{subsubsec:performance-based-block-grants}

Block grants help sustain systemic research capacity. They help fund general research infrastructure such as laboratories and libraries that can be used in many different research projects. Project grants do not cover 100 per cent of project costs, on the assumption that block grants cover part of the total cost. Block grants are, however, widely regarded as too low to cover all the indirect costs associated with competitive grants.\footnote{Watt (2015), p 13-14} In 2001, block grants programs paid \$2.37 for every \$1 of ARC and NHMRC funding. In 2017, they paid \$1.37.

Historically, several block grant programs have operated with different purposes and funding formulas.\footnote{Larkins (2011). See also Norton and Cherastidtham (2014), p 50-51.} These have now been simplified into two main programs.

Research Support Program funding is distributed to universities according to their success in attracting research income. It is slightly weighted to industry and other engagement income (52.8 per cent) compared to competitive grant income (47.2 per cent).\footnote{Department of Education and Training (2017f)} The RSP allocated \$894 million for 2018, with Group of Eight universities receiving two-thirds of the total.\footnote{Department of Education and Training (2017g)}

Research Training Program funding is distributed to universities according to their numbers of research degree completions (50 per cent weighting), and their share of competitive grants and industry and other research income (25 per cent weighting each).\footnote{Department of Education and Training (2017f)} The RTP provides student income support as well as offsetting other university research training costs (\Cref{subsec:private-spending-by-students}). Its allocation is \$1.03 billion for 2018, with 60 per cent going to Group of Eight universities.\footnote{Department of Education and Training (2017g)}

Most previous research block grant programs were weighted more towards competitive grant income. They also included numbers of academic publications. The current formulas are intended to encourage university engagement with industry.

The grants described in this section are the largest recurrent sources of specific research funding. There are also other smaller research funding programs, including access to the new Medical Research Future Fund; contract research from government agencies; once-off capital grants for research infrastructure and various other funding sources from all levels of government.\footnote{Department of Education and Training (2017c); Department of Industry (2017b); Department of Health (2018)}

%
\subsection{Other sources of research funding}\label{subsec:other-sources-of-research-funding}

Government funding specifically for research finances only part of total university research activity (\Cref{fig:total-university-research-expenditure-19922016}). In 2016, total university research spending was \$11.3 billion (in \$2018). Commonwealth research-specific spending financed 34 per cent of university research expenditure. Universities also draw on international and private sources of research funding, including industry contracts and donations. Together, these sources finance another 15 per cent of research.\footnote{Department of Education and Training (2017h) provides total and by-university data.} About half of all research expenditure is therefore financed from sources that are not specifically for research.

    \begin{figure} %% original Figure 32
    \caption{Total university research expenditure, 1992--2016}\label{fig:total-university-research-expenditure-19922016}
    \units{\$2018 billion}
    \includegraphics[page= 32, width=1\columnwidth]{atlas/mycharts.pdf}
    \notewithsources{Indexed using CPI. Commonwealth research-specific funding includes research block grants, competitive grants, and other Commonwealth income recorded in the Higher Education Research Data Collection.}{{} {Department of Education and Training (2017iABS (2018eUniversities Australia (2018b); ); )}}
    \end{figure}


The Commonwealth Grant Scheme discussed in \Cref{sec:spending-on-and-by-students} is one of these sources. Its predecessor funding program, the operating grant, was explicitly for teaching and research, and this combination established university practices. The most recent analysis of CGS spending, using a sample of universities, concluded that 85 per cent of it was spent on teaching.\footnote{Deloitte Access Economics (2017), p xxii} The remaining 15 per cent could have provided approximately \$1 billion for research in 2016.

Along with CGS money, universities make profits on full-fee students (\Cref{subsec:private-spending-by-students} and 7.1.2). Grattan Institute analysis for 2013 estimated that overall at least one dollar in every five spent on research came from teaching-driven funding. Financial transfers occur between disciplines as well as between teaching and research. Surpluses from teaching business-related courses support research in other faculties.\footnote{Norton and Cherastidtham (2015a)}

While using student-derived revenue for research is sometimes questioned, it is unavoidable given the current structure of Australian higher education. Most academic staff with permanent or fixed-term contracts have both teaching and research roles (\Cref{sec:the-research-workforce}), but this staffing model is not well-supported by the public funding system. Teaching staff and funding reflect student enrolments by institution and field of study. Yet the main research funding schemes distribute money using criteria that are unrelated to coursework student numbers. Government funding schemes tend to drive teaching and research resources in divergent directions. Research spending funded by surpluses on teaching preserves teaching-research academic employment.

Teaching-funded research can also protect university autonomy. Although governments have successfully steered university research towards more applied research (\Cref{sec:research-topics-and-types}), the Commonwealth's dwindling share of all research funding expenditure suggests that universities can finance their own priorities.

%
\section{Public and private spending over the long run}\label{sec:public-and-private-spending-over-the-long-run}

Over the long run, total public spending on higher education has increased in most years. From the perspective of universities, it has two distinct phases over the last 80 years, as \Cref{fig:public-and-private-revenue-shares-of-universities-19392016} shows. Until the late 1980s public funding complemented and then replaced income from students, pushing up the government share of all university revenue. From the late 1980s private funding usually grew more quickly than public funding, due to HECS/student contributions and full-fee courses, pushing down government spending as a share of all university revenue.\footnote{See Norton (2017a) for more detail on the history of public and private funding.} Despite the growth of private funding, universities remain reliant on government. In 2016, 58 per cent of university cash flow -- counting both grants and HELP revenue -- came from government.\footnote{Commonwealth Government grants, 38.6 per cent; HELP loans 17.5 per cent; State Government grants, 2.2 per cent.}

    \begin{figure} %% original Figure 33
    \caption{Public and private revenue shares of universities, 1939--2016}\label{fig:public-and-private-revenue-shares-of-universities-19392016}
    \units{}
    \includegraphics[page= 33, width=1\columnwidth]{atlas/mycharts.pdf}
    \notewithsources{Upfront student payments include fees and HECS or student contribution payments.}{DEET (1993Department of Education and Training (various years-a); )}
    \end{figure}


%
\section{Overall financial position }\label{sec:overall-financial-position}

From the mid-1990s to the mid-2000s public universities often experienced financial difficulties, but their position improved subsequently (\Cref{fig:public-university-revenue-and-expenses-19962016}). For a number of years, additional government grant income (\Cref{fig:core-teaching-grant-funding-19892018}, \Cref{fig:arc-nhmrc-and-block-research-grants-to-universities-20012018}) and private revenue (\Cref{fig:teaching-revenue-from-students-19972016}) each contributed to larger annual surpluses despite rising expenses. In recent years, surpluses have started to shrink again. In 2016, they were 5 per cent of revenue.

In the near future, booming international student numbers (\Cref{subsec:full-and-part-time-enrolment}) will deliver large profits to some universities. But limits on total Commonwealth Grant Scheme spending (\Cref{sec:spending-on-and-by-students}) mean that costs for Commonwealth supported students will grow more quickly than revenue for 2018 and 2019 at least. Universities that were expanding Commonwealth supported enrolments before their funding was frozen may be locked into new spending without fully offsetting new revenue (they will still receive student contributions).\footnote{The universities that were expected to expand their enrolments and income can be seen in Conifer (2018).} A downturn in international student numbers, as has happened before (\Cref{sec:international-students}), would put further pressure on university finances.


    \begin{figure} %% original Figure 34
    \caption{Public university revenue and expenses, 1996--2016}\label{fig:public-university-revenue-and-expenses-19962016}
    \units{\$2018 billion}
    \includegraphics[page= 34, width=1\columnwidth]{atlas/mycharts.pdf}
    \notewithsource{Indexed using CPI. The 2008 result was due to investment losses in the global financial crisis.}{Department of Education and Training (various years-a)}
    \end{figure}



In recent years, TEQSA has published limited information about NUHEP finances. Not including TAFEs, their revenues were \$3.9 billion in 2016-17, including from their non-higher education activities.\footnote{TEQSA (2018c), p 45} Many not-for-profit NUHEPs are financially fragile, with a third making losses in 2015-16 and overall surpluses at 1 per cent of revenue. The median for-profit higher education provider earned profits of 10 per cent of their revenue, with one-in-ten making a loss.\footnote{TEQSA (2017b), p 8, 21} Navitas Ltd is the largest for-profit in the Australian market. In 2016--17 it had university partnerships revenues of \$574 million, with profits of \$125 million, from operations in many countries.\footnote{Navitas (2017), p 67. Laureate International, which operates Torrens University, is smaller than Navitas in Australia but has global revenues of \$US4.4 billion in 2017, making it the world's largest education company: Laureate Education (2018).} It is bigger than many of the smaller public universities.

%
\chapter{Higher education finance -- the micro picture}\label{chap:higher-education-finance-the-micro-picture}

This chapter investigates the financing arrangements at the micro level of how resources are allocated to students. It discusses how policy and history influence funding levels for Commonwealth supported student places. It explains how student places are distributed among higher education providers.

%
\section{Funding per student}\label{sec:funding-per-student}

%
\subsection{Commonwealth supported students}\label{subsec:commonwealth-supported-students}

A `Commonwealth supported student' is somebody in a place funded by the Commonwealth Grant Scheme (\Cref{sec:spending-on-and-by-students}) or required to pay a student contribution.\footnote{Most Commonwealth supported students are supported by the CGS and pay a student contribution. However, students in enabling places aimed at preparing them for a higher education course do not pay student contributions, and universities do not receive CGS payments for enrolments in excess of agreed numbers in courses with allocated places or when the total funding amount would exceed their maximum CGS payment (\Cref{subsec:teaching-grants-for-higher-education-institutions} and 7.2.1).} Except in limited circumstances, every domestic undergraduate student in a public university is Commonwealth supported. Postgraduate coursework Commonwealth supported places are initially allocated to universities, which then distribute them to students. In recent years about one-third of domestic postgraduate coursework students have been Commonwealth supported.\footnote{Department of Education and Training (various years-c). Publicly-funded postgraduate research places are distributed under the Research Training Program, see \Cref{subsec:public-research-funding-programs}.}

Commonwealth supported students can pay their student contribution directly to their university or, if they are citizens, borrow it under the HECS-HELP scheme (\Cref{subsec:lending-to-students}). If the student borrows under HECS-HELP, the Commonwealth Government pays the university on the student's behalf.\footnote{Australian citizens, permanent residents and New Zealand citizens resident in Australia are domestic students for CGS funding. However, only Australian citizens and New Zealand citizens who have lived in Australia for ten or more years and since they were a child have access to HELP. Other New Zealand citizens and permanent residents must pay their student contributions upfront.} Nearly 90 per cent of student contribution liabilities are deferred using HECS-HELP.

Commonwealth and student contributions are both based on the unit of study, or subject. They are the same for undergraduates and postgraduates, but differ according to field of study. There are eight Commonwealth contribution amounts and three student contribution amounts.\footnote{Disciplines with the same Commonwealth contributions are described as being in a `funding cluster', and the disciplines with the same student contributions are described as being in the same `student contribution band'.} For each field, adding together the Commonwealth and student contribution gives the total funding rate. Table 5 lists fields of study and their funding levels, expressed as the rate for a full year of study.

These rates reflect history and political compromises. A study of higher education expenditure from the late 1980s is the single biggest influence on the total per student amount.\footnote{For the background, see DEEWR (2010) p 24-26.} The biggest change since then happened in 2005, when universities were given the power to set student contributions, up to a legislated maximum. They could keep the money (previously, HECS went to the government). For most disciplines, the maximum was 25 per cent more than the previous HECS rates. There was no science to this particular percentage; it was a political compromise to get the higher education reform bills through the Senate. Maximum student contributions quickly became standard prices charged by all universities.

\begin{table} \caption{Contributions for a 2018 Commonwealth-supported place}

%%\begin{longtable}[]{@{}llll@{}}
%%\toprule
%%\textbf{Discipline} & \textbf{Commonwealth contribution} & \textbf{Maximum student contribution } & \textbf{Total funding rate }\tabularnewline
%%\midrule
%%\endhead
%%Law, business, economics & \$2,120 & \$10,754 & \$12,874\tabularnewline
%%Humanities & \$5,896 & \$6,444 & \$12,340\tabularnewline
%%Mathematics, statistics & \$10,432 & \$9,185 & \$19,617\tabularnewline
%%Computing, other health & \$10,432 & \$9,185 & \$19,617\tabularnewline
%%Behavioural sciences & \$10,432 & \$6,444 & \$16,876\tabularnewline
%%Journalism & \$12,830 & \$6,444 & \$19,274\tabularnewline
%%Social studies & \$10,432 & \$6,444 & \$16,876\tabularnewline
%%Architecture & \$10,432 & \$9,185 & \$19,617\tabularnewline
%%Education & \$10,855 & \$6,444 & \$17,299\tabularnewline
%%Clinical psychology & \$12,830 & \$6,444 & \$19,274\tabularnewline
%%Visual and performing arts & \$12,830 & \$6,444 & \$19,274\tabularnewline
%%Allied health & \$12,380 & \$9,185 & \$22,015\tabularnewline
%%Nursing & \$14,324 & \$6,444 & \$20,768\tabularnewline
%%Engineering & \$18,240 & \$9,185 & \$27,425\tabularnewline
%%Science & \$18,240 & \$9,185 & \$27,425\tabularnewline
%%Dentistry, medicine, veterinary science & \$23,151 & \$10,754 & \$33,905\tabularnewline
%%Agriculture & \$23,151 & \$9,185 & \$32,336\tabularnewline
%%\bottomrule
%%\end{longtable}

\end{table}

\emph{Source:} \emph{Department of Education and Training (2017l)}

%
\subsection{Full-fee paying students}\label{subsec:full-fee-paying-students}

Full-fee paying students do not receive CGS funding. The fees they pay are lightly regulated. There is a floor price for international students, but no legal ceiling on the fees universities can charge international students or domestic students in full-fee markets.\footnote{The floor prices are set out in the \emph{Higher Education Provider Guidelines 2012.} Their purpose is to prevent universities spending public money on international students.} Only market forces regulate maximum fees.

\Cref{fig:annual-international-student-bachelor-degree-fees-2018} shows median fees charged to international students taking bachelor degrees in 2018, along with the maximum and minimum fee charged. The median fee ranges from \$27,500 to \$34,000 a year depending on discipline. Fees vary widely around these mid-points. Students can pay twice as much to attend the most expensive university as the cheapest university offering a similar course. International students often prefer high-fee over low-fee universities.\footnote{Norton and Cherastidtham (2015b), \Cref{chap:higher-education-students}} Generally, universities earn more from an international than a domestic student. However, some universities set fees for international students in agriculture, science and engineering that are below the combined Commonwealth and student contributions reported in Table 5.


    \begin{figure} %% original Figure 35
    \caption{Annual international student bachelor degree fees, 2018}\label{fig:annual-international-student-bachelor-degree-fees-2018}
    \units{\$2018}
    \includegraphics[page= 35, width=1\columnwidth]{atlas/mycharts.pdf}
    \noteswithsource{Course fees were based on comparing similar courses at different universities. Fees are indicative.}{University websites}
    \end{figure}



Although domestic postgraduates are sometimes charged high fees, these are never more than and usually significantly less than the fees charged to international students in the same course. In disciplines such as nursing and teaching, it is common for domestic postgraduate fees to be less than the funding rates for a Commonwealth-supported place.\footnote{In 2018, 7 of 27 universities offering a full-fee master of teaching or education course to domestic students, and 12 of 23 universities offering a full-fee master of nursing course charged fees below the Commonwealth supported rates shown in Table 5. Source: University websites.}

%
\subsection{Teaching spending per student}\label{subsec:teaching-spending-per-student}

Although we can identify most revenue coming to public universities from teaching (\Cref{sec:spending-on-and-by-students}), spending on teaching is not currently reported, although the Government is going to collect this information from universities receiving CGS funding.\footnote{Spending on teaching and scholarship as a condition of funding: Department of Education and Training (2018d).}

There are inherent difficulties in calculating teaching spending. The same staff and facilities are used to produce teaching, research and community engagement. Time and facility use surveys can allocate some costs among activities, but not all expenditures can be neatly classified in this way. Assumptions need to be made, which may inflate or deflate teaching costs.

Using 2015 cost data, a 17-university study found that median undergraduate teaching costs were \emph{below} funding rates in nine of ten broad fields of study (though at least one university had costs above funding in each of the ten). The average cost on this basis was about \$16,000 per EFTSL.\footnote{Deloitte Access Economics (2017) ,p xxii, 17.}

The observed behaviour of public universities suggests that funding for Commonwealth-supported places is usually sufficient to cover teaching costs. Public universities voluntarily increased their annual domestic bachelor-degree intake by more than 40 per between 2008 and 2015, when numbers stabilised.\footnote{Department of Education and Training (2018h)} The universities enjoyed health financial surpluses in this period (\Cref{sec:overall-financial-position}).

Universities often claim to be under-funded, but it is difficult to evaluate whether this is true, and, if so, by how much. The problems are partly conceptual -- to what extent should research be funded through teaching, and what standard of course delivery is acceptable? And they are partly evidential -- how should costs be calculated, and what assumptions should be made about reasonable costs?

%
\subsection{Internal allocation of funding}\label{subsec:internal-allocation-of-funding}

Universities are not obliged to spend teaching revenues in the disciplines or departments that earned them. The funding rates reported in Table 5 above are not recommended internal funding rates. They were essentially used -- at least until the inception of the demand-driven funding system discussed in \Cref{subsec:distributing-government-supported-places} below -- to calculate a total sum of money paid to each university. Within their overall funding allocation, universities can design internal funding systems reflecting their own costs and priorities. The federal funding system does not adjust per-student rates to institutional differences, but it does permit universities to make those adjustments in how they spend their money.

In practice, revenue from Commonwealth supported students tends to be allocated to the faculties or departments where the students are enrolled. If spending on these students exceeds revenues, the faculties or departments are typically described as losing money or receiving cross-subsidies from profitable parts of the university. If costs cannot be contained or other revenues found, `loss-making' areas risk closure. So in practice Commonwealth funding rates shape university behaviour more than policymakers originally intended.

%
\section{Distributing student places}\label{sec:distributing-student-places}

A higher education system requires a system of distributing student places. Places need to be allocated to higher education providers, disciplines and students. The three broad theoretical models are central allocation, block grant and market distribution.

In a central allocation model, the government determines detailed priorities and allocates the student places it funds accordingly. Priorities could be for specific courses or disciplines, higher education providers, or types of students. While nobody is forced to take student places created under government-priority setting, the system limits options. People who want a university place must take what is available. Priority setting can be supported by student incentives, such as scholarships or lower fees.

In a block grant model, the government allocates funding to higher providers, with broad guidelines about its use, such as minimum numbers of students. Within these broad rules, universities decide how to spend the money.

In a market distribution model, the government does not set priorities. Higher education providers decide what courses to offer students, and students decide whether to enrol in the courses at the fees charged. This is the model that largely applies for international students, for much of the domestic postgraduate market, and among the non-university higher education providers (NUHEPs -- see \Cref{sec:non-university-higher-education-providers}).

Compared to central allocation of student places, a market system gives students more power. Higher education institutions have stronger incentives to respond to student preferences, and to concentrate on the student experience. Yet market systems depend on students paying full fees, which may reduce total demand for higher education, especially without a student loan scheme.

A higher education `voucher' scheme combines market mechanisms and public subsidies. Under this model, the government broadly steers the higher education market, using subsidies to make higher education generally or particular disciplines more financially attractive. The number of vouchers can be limited or unlimited, and rationed using academic results or other eligibility criteria. The key point is that higher education providers must compete for students, rather being allocated student places. Voucher schemes may have literal vouchers -- documents sent to prospective students that they can redeem at higher education providers. Usually this is not necessary. Prospective students can provide higher education providers with evidence of their eligibility.

%
\subsection{Distributing government-supported places}\label{subsec:distributing-government-supported-places}

Historically, Australia has mostly used block grant systems to distribute government-supported student places. Not using detailed centralised allocation acknowledged universities as autonomous institutions (see also \Cref{subsec:self-accreditation} to 1.3.5). Until the 1980s, this distance from political intervention was reinforced by using semi-independent bureaucratic bodies to distribute funding.\footnote{Coaldrake and Stedman (2016), p 232-239} Centralised allocation was used at the margins, mostly through funding new higher education places. This was sometimes very prescriptive, allocating precise numbers of places to specific courses and campuses. But new places were only ever a small percentage of total Commonwealth-supported places.

Block grants let universities plan around predictable public funding levels. This gave the system stability, but weakened competitive pressures. Universities had few financial incentives to attract additional students or to change what they offered to reflect student preferences. For a few years in the mid-2000s, universities were penalised if they exceeded enrolment targets set out in funding agreements with the Government by more than 5 per cent. With demand exceeding the supply of student places, each publicly-funded university had a virtually guaranteed share of total enrolments.\footnote{A summary of allocative systems since 1989 can be found in Grattan Institute (2018).}

In 2009, the Government announced that it would introduce a `demand-driven' funding system.\footnote{DEEWR (2009), p 17-19} After a phase-in period of increased maximum block grants, in 2012 funding caps on Commonwealth-supported bachelor-degree places at public universities, except in medicine, were lifted. Medical places, Commonwealth-supported postgraduate places and sub-bachelor places (diploma, advanced diploma, associate degree -- see \Cref{sec:what-is-higher-education}) were still allocated centrally, using funding agreements between the government and universities. Despite these exceptions, demand driven funding represented a major shift away from block grants to the voucher model. Bachelor-degree enrolments in each public university, along with the system as a whole, could now move up and down in line with university decisions and student demand.

Demand driven funding achieved its original goals of increasing participation in higher education (see \Cref{sec:domestic-students}) and making universities more responsive to student and labour market demand.\footnote{Kemp and Norton (2014); Norton (2017c)} But it was criticised for letting in under-prepared students, producing too many graduates, and costing too much. This last criticism finally brought demand driven funding to an end, with the funding freeze discussed in \Cref{subsec:teaching-grants-for-higher-education-institutions}.\footnote{Birmingham (2017)} Effectively, block grant funding was restored. The opposition Labor Party promises to bring back demand driven funding.\footnote{Shorten (2018)}

%
\chapter{Higher education policymaking }\label{chap:higher-education-policymaking}

Higher education policymaking has become increasingly centralised in Canberra. This chapter reviews the major higher education policymakers and the interest groups that try to influence policy.

%
\section{The rise of Commonwealth authority}\label{sec:the-rise-of-commonwealth-authority}

Australian higher education began as a state responsibility. Except in its territories, the Commonwealth Government lacked clear constitutional power to establish or regulate a higher education institution. The Canberra-based Australian National University, legislated in 1946, is the only university with Commonwealth founding legislation.\footnote{Other universities are established under state or territory legislation or company law.} There was no federal department or minister for education until 1966.\footnote{Previously education was managed by the Department of Prime Minister and Cabinet: Tracey (2001). See Parliamentary Library (2017) for the names of the first education minister and his successors.}

While the states had full responsibility for education in Australia's early decades, after World War Two the Commonwealth slowly increased its policy involvement in higher education.\footnote{See Tracey (2001); Forsyth (2014), especially \Cref{chap:the-student-experience-and-academic-integrity}; Department of Prime Minister and Cabinet (2014), \Cref{chap:the-higher-education-workforce}.} A 1946 amendment to the Australian Constitution authorised the Federal Government to make laws with respect to `benefits to students'. This remains the only reference in the Australian Constitution to education, albeit an indirect one. The main constitutional vehicle for funding higher education was through conditional grants to the states. This was replaced in 1993 with direct grants to universities.

The Commonwealth's control of money gave it significant power in higher education, but in law it was a limited power. The rules it imposed were conditions of grants, not laws that had to be followed. The public universities could, in theory, have declined a Commonwealth grant and its associated conditions. In practice, universities have generally accepted whatever funding conditions were set. This let the Commonwealth leverage its limited legal position into extensive control.

Commonwealth funding power reached its peak in the decade after 1974. State governments stopped funding teaching and research on a regular basis, and universities were not allowed to charge tuition fees until these were reintroduced from the mid-1980s.\footnote{The period of free education and its gradual replacement is described in more detail in Norton (2017a).} Although the governments of this time respected university academic autonomy, the universities had limited non-Commonwealth government income to fund their own priorities (\Cref{fig:public-and-private-revenue-shares-of-universities-19392016}, page 74).

Until FEE-HELP began in 2005, most private higher education institutions received no money from the Commonwealth, and so were free of Commonwealth control, beyond general laws applying to all. Private higher education institutions were regulated by State accreditation agencies, although with national coordination of rules from 2000.\footnote{MCEETYA (2000), later replaced by MCEETYA (2007).}

High Court decisions have altered the legal basis of higher education policy. In the 2006 \emph{WorkChoices} case the High Court took an expansive view of the Australian Constitution's corporations power. Since higher education is largely delivered by organisations, including universities, that are legally corporations (as opposed to partnerships or state government departments), the Federal Government now uses the corporations power to regulate higher education accreditation and quality control. The Tertiary Education Quality and Standards Agency (TEQSA) replaced the state higher education accreditation bodies in 2012.\footnote{For more detail on the legal issues see Williams and Pillai (2011).}

TEQSA is a sign of how higher education policymaking is changing. Using the corporations power, the Commonwealth can now mandate rather than buy compliance. The funding legislation now imposes civil penalties (fines) on higher education providers breaching government requirements, alongside the older conditions on funding.\footnote{See the provisions introduced by the \emph{Education Legislation Amendment (Provider Integrity and Other Measures) Act 2017.}} The corporations power brings all higher education institutions, not just those receiving public funds, under Commonwealth Government control.

While the \emph{WorkChoices} case increased Commonwealth power, other High Court cases have complicated it. In a 2014 case on Commonwealth funding of school chaplains, the High Court took a narrow view of the `benefits to students' power. The benefit needs to be closely related to being a student, and for specific students.\footnote{Chordia\emph{, et al.} (2015)} On this reading, the `benefits to students' power almost certainly could not be used to fund research unless it directly involved students, such as the Research Training Program (\Cref{subsec:public-research-funding-programs}).

Complicating matters further, direct Commonwealth research funding lacks an explicit constitutional basis. There is some High Court authority for using an implied `nationhood' power to support research spending.\footnote{See Twomey (2010) for an analysis and critique.} In 2015, the government strengthened the legal basis of research block grants and some other university programs. They did this by mentioning in higher education funding legislation a list of potential constitutional foundations.\footnote{Now section 41-95 of the \emph{Higher Education Support Act 2003}.} If direct Commonwealth research funding were successfully challenged in the High Court, it could be restored through conditional grants to the states.

The more likely outcome is that the states will continue with their current limited role in higher education policy.\footnote{Charles Darwin University and the University of Canberra have legislation from their respective territories. Although the territories have a lesser constitutional status than the states, the following paragraph applies to them.} They still have university establishment acts on their statute books, and impose various reporting and accountability requirements on universities. They can affect university admissions through their school systems and by their accreditation of teacher education courses (\Cref{sec:passing-and-failing}). TEQSA must consult them about some matters, including new universities in their jurisdictions. They are still expected to fund special projects at universities within their borders. Yet on key higher education policy matters the states have little influence.

When the Commonwealth sets all important aspects of higher education policy, the relevant departments and agencies matter more than ever to the success of Australian higher education.

%
\section{Commonwealth departments and agencies}\label{sec:commonwealth-departments-and-agencies}

%
\subsection{The Department of Education and Training}\label{subsec:the-department-of-education-and-training}

Higher education is primarily the responsibility of the Department of Education and Training. It manages the major teaching and research block grant funding schemes described in \Cref{chap:higher-education-finance-the-macro-picture} and 7. These are authorised by the \emph{Higher Education Support Act 2003.} It also has over-arching policy responsibility for tertiary education standards (discussed below). These are authorised by the \emph{Tertiary Education Quality and Standards Agency Act 2011.} Other important pieces of legislation overseen by the Department are the \emph{Education Services for Overseas Students Act 2000} and the \emph{Australian Research Council Act 2001}.

%
\subsection{Higher Education Standards Panel}\label{subsec:higher-education-standards-panel}

Under the TEQSA legislation the higher education minister performs the key policy making function, setting standards applying to higher education providers under the Higher Education Standards Framework. The standards cover higher education provider registration, course accreditation, and qualifications. Higher education providers need to meet the standards to offer courses leading to higher education awards.

The education minister appoints a Higher Education Standards Panel to develop and advise on the standards. Before making a standard, the minister consults state education ministers and TEQSA. The current standards took effect in 2017.\footnote{Department of Education and Training (2015c)} In practice, the minister uses the Panel to investigate and report on matters broadly related to the standards. These have included how admissions requirements are publicised, students not completing their courses, and professional accreditation requirements.

%
\subsection{Tertiary Education Quality and Standards Agency}\label{subsec:tertiary-education-quality-and-standards-agency}

TEQSA began operations in early 2012. Its main task is to apply and enforce the TEQSA legislation and the Higher Education Standards Framework. It is also responsible for several regulatory functions under the \emph{Education Services for Overseas Students Act 2000}.

TEQSA registers higher education providers and approves courses offered by non-self-accrediting institutions (\Cref{chap:higher-education-providers}). It carries out this task independently of the minister, who can only give TEQSA directions of a general nature (not about a specific provider). These directions can be disallowed by a majority vote of the House of Representatives or the Senate.

The legislation establishing TEQSA sought to minimise the bureaucratic burden it places on higher education providers. It uses a range of risk indicators to monitor higher education providers, concentrating its attention on the institutions at most risk of not complying with the standards.\footnote{TEQSA (2018e)}

%
\subsection{The research grant agencies}\label{subsec:the-research-grant-agencies}

The two main competitive grant research agencies are the Australian Research Council (ARC) and the National Health and Medical Research Council (NHMRC) (\Cref{subsec:public-research-funding-programs}). They both have their own statutes and report respectively to the education minister and the health minister.

The ARC and NHMRC work within broad policy frameworks established by the Government, with priorities set by the relevant ministers. Both organisations use systems of peer review to determine which applications are successful. This approach respects the culture of universities (\Cref{sec:what-is-distinctive-about-universities}). Each organisation's legislation prevents its minister interfering in favour of a project. They can only approve or not approve the funding recommendations made by the agencies. Approval is usually a formality. Rare rejections of ARC grant recommendations are always controversial.\footnote{Haigh (2006)}

The media and politicians sometimes question ARC-funded projects with seemingly obscure, trivial or politicised topics. Academics sometimes claim that the peer view process leads to favouritism (to the detriment of the complainant's application). Yet overall the ARC and NHMRC enjoy high esteem. The most widespread criticism is that given low application success rates (\Cref{subsec:public-research-funding-programs}), resources are wasted preparing and assessing applications that are rejected.

%
\subsection{The Chief Scientist}\label{subsec:the-chief-scientist}

Australia's Chief Scientist advises the Prime Minister and other ministers on science, technology and innovation. Chief Scientists have been advocates for increased enrolments in science and technology courses, and advised governments on research policy.

%
\subsection{Department of Home Affairs}\label{subsec:department-of-home-affairs}

The Department of Home Affairs, which includes the former Department of Immigration, has a major influence on Australian higher education. It controls eligibility for student visas, and the post-study temporary and permanent migration programs that attract international students to Australia.

Onshore international students need a student visa. Visa applicants must show that they have confirmed enrolment for their course, have health insurance, meet rules on English language ability, and can support themselves financially. The amount of evidence needed depends on the applicant's home country and education provider.\footnote{Department of Home Affairs (2018b). Providers and countries have different risk levels, based on histories of applications with fraudulent information, rates of overstaying in Australia after a student visa expires, and other visa issues: Department of Home Affairs (2018a).} The Department tries to exclude applicants with fraudulent documents or claims in their applications, or who might breach visa conditions around attending classes and work hours, or who might overstay their visa.\footnote{Department of Home Affairs (2018a)}

International students graduating with a bachelor degree or higher qualification can apply for another visa to remain in Australia and work full-time. They can stay for two to four years depending on their qualification.\footnote{Department of Home Affairs (2018e)} Former student visa holders also remain in Australia through a range of other temporary visas.\footnote{Department of Home Affairs (2018c), p 69} Former students with work rights can apply for an independent skilled migration visa or employer sponsorship to continue their employment in Australia beyond this time period.\footnote{Examples of sponsored visas are the Employer Nomination Scheme (subclass 186) or Regional Sponsored Migration Scheme visa (subclass 187). Skilled Independent (subclass 189) or Skilled Nominated (subclass 190) visas are available to some former international students.}

%
\subsection{Austrade}\label{subsec:austrade}

The Australian Trade and Investment Commission, known as Austrade, promotes Australian education to international students. It is a statutory agency in the Foreign Affairs and Trade portfolio.

%
\subsection{Departments of Social Services and Human Services}\label{subsec:departments-of-social-services-and-human-services}

The Department of Social Services is responsible for student income support policy. Through Centrelink, the Department of Human Services administers payment of student income support, including the Start-up Loan (\Cref{subsec:student-income-support}).

%
\section{Higher education interest groups}\label{sec:higher-education-interest-groups}

There are higher education interest groups representing universities, private higher education providers, higher education staff, and students.

%
\subsection{University interest groups}\label{subsec:university-interest-groups}

The oldest university interest group is Universities Australia, formerly known as the Australian Vice-Chancellors' Committee (AVCC). All 37 public universities, along with Bond University and the University of Notre Dame, are members of Universities Australia.

In the 1990s, the AVCC struggled to represent the diverging interests of its members, especially on research policy and fees for domestic students. Several new university organisations have been formed since 1999 to give voice to the different perspectives within the university sector. These include the Australian Technology Network which includes all the universities of technology except Swinburne; the Group of Eight, representing the eight most research-intensive universities; Innovative Research Universities, mostly made up of suburban research-intensive universities founded in the 1960s and 1970s; and the Regional Universities Network, which represents six regional universities. About two-thirds of universities are members of one of these groups. Full membership lists of the university interest groups are in Appendix A.

%
\subsection{Non-university higher education provider interest groups}\label{subsec:non-university-higher-education-provider-interest-groups}

The largest private higher education interest group is the Australian Council for Private Education and Training (ACPET), which also represents vocational education providers. The smaller Council of Private Higher Education (CoPHE) represents only higher education providers. Both organisations have lobbied for more equal treatment of public and private higher education provision.

%
\subsection{Staff and student interest groups}\label{subsec:staff-and-student-interest-groups}

The major union representing university staff, the National Tertiary Education Union (NTEU), has more than 27,000 members, although this is under a quarter of university staff.\footnote{NTEU (2017), p. 25. Some non-academic staff are not eligible to join the NTEU. O'Brien (2015) provides a general history of the NTEU.}

The National Union of Students (NUS) is a peak body for university student organisations, although a substantial minority of student unions are not currently involved.\footnote{NUS does not maintain a public list of its affiliate organisations. This information comes from a spreadsheet of delegates to the 2017 annual conference:@edpity and @Conor\_Day99 (2017). Hastings (2003) provides the early history of NUS.} The Council of Australian Postgraduate Associations (CAPA) is another student peak body, representing most campus-based postgraduate organisations. These student groups are consistent advocates of higher education public funding.

The Council of International Students Australia (CISA) represents international students in Australia, including those enrolled in higher education courses. Unlike other higher education interest groups, it is active on state-level issues including public transport concessions, crime affecting international students, and access to public hospitals.

In 2017, TEQSA established a Student Expert Advisory Group, which includes the NUS, CAPA and CISA, along with other smaller student groups.\footnote{TEQSA (2018d)}

%
\chapter{Benefits of higher education for the public and employers}\label{chap:benefits-of-higher-education-for-the-public-and-employers}

This chapter looks at how well the higher education system meets the needs of the country. Is the population becoming more educated? Are employers' skills needs being met? Is university research output meeting expectations? How does the public perceive our higher education sector?

%
\section{Creating a more educated population}\label{sec:creating-a-more-educated-population}

As the enrolment figures in \Cref{chap:higher-education-students} suggest, higher education attainment in Australia has increased over time. In 1982, 645,000 people held a degree; by 2017 that number exceeded 4.8 million.\footnote{ABS (2017b)} \Cref{fig:higher-education-attainment-rate-men-and-women-aged-2534-19822017} shows the share of Australian residents aged 25 to 34 with a bachelor degree or higher qualification. Over the last 35 years, the share of young adults, holding a degree has increased dramatically, especially for women, whose attainment level was less than 10 per cent in the early 1980s but 45 per cent in 2017. Men lag well behind on 34 per cent, although if upper-level vocational qualifications are included attainment levels are more equal, with men at 69 per cent and women at 72 per cent.\footnote{Grattan calculations from ABS (2017d). Including diploma and certificates III and IV for upper-level vocational qualifications.}

While Australia's population has become more educated, this is not solely due to its higher education system. Australia's skilled migration program has also contributed significantly, with 24 per cent of bachelor-degree or higher qualifications completed overseas. The Australian-born population is less educated than migrants, with 38 per cent of women born in Australia and 25 per cent of men aged 25--34 holding a bachelor degree or above in 2017.\footnote{Grattan calculations from ibid.} These attainment rates will increase as students from the 2009 to 2015 enrolment boom reach the 25--34 year old age rage (\Cref{sec:domestic-students}). This will increase male education levels, but as women remain a substantial majority of university students (\Cref{subsec:other-student-characteristics}), the gender gap is unlikely to narrow significantly.

    \begin{figure} %% original Figure 36
    \caption{Higher education attainment rate, men and women aged 25--34, 1982--2017}\label{fig:higher-education-attainment-rate-men-and-women-aged-2534-19822017}
    \units{Per cent}
    \includegraphics[page= 36, width=1\columnwidth]{atlas/mycharts.pdf}
    \notewithsource{Bachelor degree or above. The fluctuations observed in some years are due to problems with the statistical sample.}{ABS (2017b)}
    \end{figure}


%
\section{Meeting skills needs }\label{sec:meeting-skills-needs}

In many occupations employers require or prefer staff with university qualifications. The latest ABS occupational list has 385 managerial and professional occupations rated as needing a university qualification or equivalent experience.\footnote{ABS (2013). At the six-digit level.} The higher education system and migration are used to supply the labour force with relevant skills.

The main available measure of skills shortages is an employer survey conducted by the Department of Employment. An occupation is classified as in skills shortage if employers cannot fill vacancies, or struggle to fill them, at current pay and condition levels, in reasonably accessible locations. This is not necessarily an absolute skills shortage; appropriately-skilled people may exist but choose other work.

Fifty-one high-skill managerial or professional occupations have had reported skills shortages at some time since 1986. Over the decade to 2017, 13 occupations, mostly in the health professions or linked to the mining industry, experienced shortages for 5 years or more. In 2008, a peak of 40 professional or managerial occupations reported shortages, but few do so now. Only five occupations experienced shortages in 2017 (\Cref{fig:number-of-managerial-and-professional-occupations-experiencing-skills-shortages-19862017}). \footnote{Department of Jobs and Small Business (2018)}

The main skills supply weakness is that health workforce needs have often grown more quickly than the supply of health professionals. But in most professional and managerial occupations, and at most times, graduate labour supply has been sufficient.

Higher education policy does not usually directly target skills needs. Before 2012 new Commonwealth-supported places were sometimes allocated in response to employer complaints about skills shortages. But mostly it is up to universities to respond to labour market demand. Under block grant funding, used until 2011 and again from 2018 (\Cref{sec:distributing-student-places}), universities lacked a strong incentive to meet demand. With little or no funding for taking additional students, they could meet skills needs only by moving student places from other courses. The demand driven system of 2012--2017 gave universities more capacity to meet skills needs. Since it was introduced, the largest increase in enrolment share has been in health-related courses (\Cref{subsec:courses-taken-by-domestic-students}). More detailed analysis of specific courses shows that universities responded with increased student places in areas of skills shortage.\footnote{Kemp and Norton (2014), \Cref{chap:the-student-experience-and-academic-integrity}; Norton (2017c)} This may not continue without demand driven funding.

Although employers sometimes cannot hire all the graduates they need, they are generally satisfied with those they do appoint. In a survey of the direct supervisors of graduates, 84 per cent were satisfied overall.\footnote{Social Research Centre/Department of Education and Training (2017), p ii}

    \begin{figure} %% original Figure 37
    \caption{Number of managerial and professional occupations experiencing skills shortages, 1986--2017}\label{fig:number-of-managerial-and-professional-occupations-experiencing-skills-shortages-19862017}
    \units{}
    \includegraphics[page= 37, width=1\columnwidth]{atlas/mycharts.pdf}
    \source{Department of Jobs and Small Business (2018)}
    \end{figure}
%
\section{Research performance}\label{sec:research-performance}

As \Cref{sec:research-outputs} shows, the quantity of research outputs, especially publications, from Australian universities has increased over time. A measure of research productivity is the average number of annual academic publications per academic. This more than doubled to 1.5 a year between 1997 and 2014, although growth is less rapid if the increasing numbers of research-only staff are considered.\footnote{This finding uses a different publications count to that reported in \Cref{sec:research-outputs}. The productivity analysis includes a weighted count of books, book chapters, journal articles and conference proceedings. In the case of multiple authors, research outputs are apportioned between them (so the work of foreign authors would not be counted, and there is no double counting of multi-author articles). Academic staff is all academics with a teaching and research or research only appointment: calculated from Department of Education and Training (2015f) and Department of Education and Training (2015e). Assuming that teaching and research staff spend 40 per cent of their time on research and research staff spend all their time on research, research output per full-time equivalent staff member increased from 1.3 to 2.5 publications a year.}

Publication numbers do not measure research quality or significance, but the best Australian research publications are well regarded internationally. Australian academics are over-represented as authors of the top one per cent of academic publications, as measured by how often these publications are cited by other academics. In 2016, 7 per cent of these most-cited publications had an Australian author, up from 3 per cent in 2000.\footnote{Department of Industry (2017a)}

In recent years, international university rankings have attracted attention. One, the Shanghai Jiao Tong Academic Ranking of World Universities, focuses exclusively on research performance. Indicators include papers published in certain high-prestige journals, numbers of high-citation researchers, and winners of Nobel Prizes and Fields Medals (a prestigious mathematics award). The most recent ranks for Australian universities are in Table 6. Six are in the top 100 universities in the world, up from two in the first year of the Shanghai Jiao Tong ranking, 2003. Twenty-three Australian universities are in the top 500 universities in the Shanghai Jiao Tong ranking.

\begin{table} \caption{Top eight Australian universities, Shanghai Jiao Tong university rankings, 2018}

%%\begin{longtable}[]{@{}ll@{}}
%%\toprule
%%University of Melbourne & 38\tabularnewline
%%\midrule
%%\endhead
%%University of Queensland & 55\tabularnewline
%%University of Sydney & 68\tabularnewline
%%Australian National University & 69\tabularnewline
%%Monash University & 91\tabularnewline
%%University of Western Australia & 93\tabularnewline
%%University of Adelaide & 101-150\tabularnewline
%%University of New South Wales & 101-150\tabularnewline
%%\bottomrule
%%\end{longtable}

\end{table}

Source: ARWU (2018)

More detailed analysis of research performance by university and discipline is available from the Excellence in Research for Australia (ERA) report from the Australian Research Council. The 2018 ERA is underway, but the results had not been released as of August 2018. Quality indicators included citations, peer review (other academics assessing the quality of work) and the level of grant income. ERA also looked at indicators of research volume and activity, indicators of research application (such as patents) and indicators of recognition (for example, a fellowship in a learned academy or editing a prestigious journal).\footnote{For some of the background to ERA and rankings, see Coaldrake and Stedman (2016), \Cref{chap:higher-education-finance-the-macro-picture}.}

In the ERA assessment, fields of research in each university that met a minimum threshold of outputs are rated from one to five. Ratings one and two indicate that research performance in that field is `below world standard'. Rating three indicates average performance at world standard. Rating four is above world standard, and rating five is well above world standard. Table 7 shows the results. On this measure, most research-active disciplines in Australian universities are at least at world standard. The proportion of research disciplines rated as below world standard dropped from 22 per cent in the 2012 ERA to 11 per cent in the 2015 ERA. The results suggest that universities find ways to minimise the number of below world standard areas.\footnote{See Henman (2015) for a discussion on universities `gaming' ERA. The ARC states that this is not a large problem: ARC (2015a).}

\begin{table} \caption{Excellence in Research for Australia, 2015}

%%\begin{longtable}[]{@{}lll@{}}
%%\toprule
%%\textbf{Rating} & \textbf{Units of evaluation} & \textbf{Percentage}\tabularnewline
%%\midrule
%%\endhead
%%1+2 (low) & 198 & 11\%\tabularnewline
%%3 & 470 & 26\%\tabularnewline
%%4 & 544 & 31\%\tabularnewline
%%5 (high) & 563 & 32\%\tabularnewline
%%Total & 1,775 & 100\%\tabularnewline
%%\bottomrule
%%\end{longtable}

\end{table}

ERA can also be used to identify disciplinary areas of national strength and weakness. Reflecting the large investment in health research (\Cref{sec:research-topics-and-types}), more than half of medical and health science disciplines were rated as well above world standard. Nearly half of the smaller earth sciences field were also well above world standard. In education and in commerce more than a third of research disciplines were rated as below world standard.

%
\section{Research impact }\label{sec:research-impact}

Policymakers have long believed that while Australian university research does reasonably well on quality measures, its social and economic impact should increase.\footnote{Department of Education/Department of Industry (2014); Larkins (2011), \Cref{chap:higher-education-finance-the-micro-picture}; Dawkins (1987), p 65} This belief is a major reason why university research activity is now much more likely to be `applied' -- aimed at specific objectives -- than it was 25 years ago (\Cref{sec:research-topics-and-types}). It is why universities work with external organisations much more than before. Between 2000 and 2017, Australian private sector research funding increased by 90 per cent, to \$730.4 billion.\footnote{Department of Education and Training (2017d). HERDC Category 3 -- domestic. In \$2018.} It is why in the 2018 ERA the ARC will also evaluate the engagement of university researchers with the community and its broader impact.\footnote{ARC (2017a)}

One measure of impact is businesses using universities as sources of innovation. An ABS business survey found that 7600 businesses sourced ideas or information for innovation from a university in 2016-17, although this was fewer than in recent years. Health-related businesses were the most likely to source ideas from universities, reflecting the strong emphasis on medical research in Australian universities (\Cref{sec:research-topics-and-types}). A much larger number of business, 78,000, sourced innovation ideas from `websites, journals, research papers or publications'.\footnote{ABS (2018b) and predecessor publications.} Once published or publicly released, research can have impact without any direct connection between its users and universities.

%
\section{Public perceptions}\label{sec:public-perceptions}

Various social surveys have asked Australians about their confidence in social institutions, including universities. Universities enjoy high levels of public confidence. In 2016, 74 per cent of respondents who expressed a view said that they had either a `great deal' of confidence in universities (12 per cent), or `quite of lot of confidence' (62 per cent) (\Cref{fig:public-confidence-in-universities-20012016}). Although that is the lowest figure since 2005, it was the third highest of the fourteen institutions mentioned in the 2016 survey.


    \begin{figure} %% original Figure 38
    \caption{Public confidence in universities, 2001--2016}\label{fig:public-confidence-in-universities-20012016}
    \units{Percentage of public with a `great deal' or `quite a lot' of confidence}
    \includegraphics[page= 38, width=1\columnwidth]{atlas/mycharts.pdf}
    \notewithsources{{Percentages are of poll respondents who offered a view, omitting those who did not answer the question or gave a `don't know' response.}}{{} {Bean et al. (2003Gibson et al. (2004Wilson et al. (2006McAllister (2008McAllister et al. (2011McAllister and Pietsch (2012McAllister (2016Cameron and McAllister (2018); ); ); ); ); ); ); )}}
    \end{figure}



%
\chapter{Benefits for graduates }\label{chap:benefits-for-graduates}

Special chapter on census data to come.

%
\chapter{Glossary}\label{chap:glossary}

ABS Australian Bureau of Statistics

ACER Australian Council for Educational Research

ACPET Australian Council for Private Education and Training

Applied research Research undertaken primarily to acquire new knowledge with a specific application in view.

AQF Australian Qualifications Framework

ANZSCO Australian and New Zealand Standard Classification of Occupations

ARC Australian Research Council

ARWU Academic Ranking of World Universities

ASCED Australian Standard Classification of Education

ATAR Australian Tertiary Admission Rank

ATN Australian Technology Network

ATO Australian Taxation Office

Attrition A student leaving without completing a course. Usually in reference to a commencing student not returning the following year.

Census date The date when domestic students become liable for student contributions or fees

CGS Commonwealth Grant Scheme

Commonwealth contribution The Federal Government's tuition subsidy

COPHE Council of Private Higher Education

Coursework Courses that do not have a major research component

CPI Consumer Price Index

Doubtful debt HELP debt not expected to be repaid

EFTSL Equivalent full-time student load

ERA Excellence in Research for Australia

Experimental development research Research using existing knowledge gained from research or practical experience, which is directed to producing new materials, products, devices, policies, behaviours or outlooks.

FEE-HELP HELP for full-fee students

FTE Full-time equivalent

\begin{quote}
Funding cluster A group of disciplines with the same Commonwealth contribution

GCA Graduate Careers Australia

Graduate premium Extra income of a graduate over another educational level, usually Year 12

Group of Eight Coalition of Australia's `sandstone' universities

HERDC Higher Education Research Data Collection

HECS Higher Education Contribution Scheme

HECS-HELP HELP for Commonwealth-supported students

HELP Higher Education Loan Program

HEP Higher Education Provider

HILDA Household, Income and Labour Dynamics in Australia Survey

IRU Innovative Research Universities

IT Information technology

Load Subjects taken, expressed in full-time student units.

NHMRC National Health and Medical Research Council

NUHEP Non-university higher education provider

OS-HELP HELP to finance overseas study

OUA Open Universities Australia

Pathway college Institution specialising in diploma level courses aimed at facilitating entry to university courses.

Place A student place is equivalent to the study load of a full-time student

Pure basic research Research to acquire new knowledge without looking for long term benefits other than advancing knowledge.

RUN Regional Universities Network

SA-HELP HELP for the student amenities fee

SES Socio-economic status

Strategic basic research Research in specified areas in the expectation of practical discoveries.

Student contribution The amount paid by a student in a Commonwealth-supported place

TAFE Technical and further education

TEQSA Tertiary Education Quality and Standards Agency
\end{quote}

\appendix
\chapter{Appendix A -- Higher education providers offering HELP loans}\label{chap:appendix-a-higher-education-providers-offering-help-loans}

%%\begin{longtable}[]{@{}lll@{}}
%%\toprule
%%\textbf{Universities} & \textbf{NUHEPs offering FEE-HELP}\tabularnewline
%%\midrule
%%\endhead
%%\begin{minipage}[t]{0.30\columnwidth}\raggedright
%%\textbf{Group of Eight}\strut
%%\end{minipage} & \begin{minipage}[t]{0.30\columnwidth}\raggedright
%%\begin{quote}
%%\textbf{Regional Universities Network}
%%\end{quote}\strut
%%\end{minipage} & \begin{minipage}[t]{0.30\columnwidth}\raggedright
%%\begin{quote}
%%Academy of Information Technology
%%\end{quote}\strut
%%\end{minipage}\tabularnewline
%%\begin{minipage}[t]{0.30\columnwidth}\raggedright
%%\begin{quote}
%%Australian National University\^{}
%%\end{quote}\strut
%%\end{minipage} & \begin{minipage}[t]{0.30\columnwidth}\raggedright
%%\begin{quote}
%%Central Queensland University*
%%\end{quote}\strut
%%\end{minipage} & \begin{minipage}[t]{0.30\columnwidth}\raggedright
%%\begin{quote}
%%Academy of Music and Performing Arts
%%\end{quote}\strut
%%\end{minipage}\tabularnewline
%%\begin{minipage}[t]{0.30\columnwidth}\raggedright
%%\begin{quote}
%%Monash University\^{}
%%\end{quote}\strut
%%\end{minipage} & \begin{minipage}[t]{0.30\columnwidth}\raggedright
%%\begin{quote}
%%Southern Cross University*
%%\end{quote}\strut
%%\end{minipage} & \begin{minipage}[t]{0.30\columnwidth}\raggedright
%%\begin{quote}
%%Adelaide Central School of Art
%%\end{quote}\strut
%%\end{minipage}\tabularnewline
%%\begin{minipage}[t]{0.30\columnwidth}\raggedright
%%\begin{quote}
%%The University of Adelaide\^{}
%%\end{quote}\strut
%%\end{minipage} & \begin{minipage}[t]{0.30\columnwidth}\raggedright
%%\begin{quote}
%%Federation University Australia*
%%\end{quote}\strut
%%\end{minipage} & \begin{minipage}[t]{0.30\columnwidth}\raggedright
%%\begin{quote}
%%Adelaide College of Divinity
%%\end{quote}\strut
%%\end{minipage}\tabularnewline
%%\begin{minipage}[t]{0.30\columnwidth}\raggedright
%%\begin{quote}
%%The University of New South Wales\^{}
%%\end{quote}\strut
%%\end{minipage} & \begin{minipage}[t]{0.30\columnwidth}\raggedright
%%\begin{quote}
%%The University of New England
%%\end{quote}\strut
%%\end{minipage} & \begin{minipage}[t]{0.30\columnwidth}\raggedright
%%\begin{quote}
%%Australasian College of Health and Wellness
%%\end{quote}\strut
%%\end{minipage}\tabularnewline
%%\begin{minipage}[t]{0.30\columnwidth}\raggedright
%%\begin{quote}
%%The University of Melbourne\^{}
%%\end{quote}\strut
%%\end{minipage} & \begin{minipage}[t]{0.30\columnwidth}\raggedright
%%\begin{quote}
%%University of Southern Queensland*
%%\end{quote}\strut
%%\end{minipage} & \begin{minipage}[t]{0.30\columnwidth}\raggedright
%%\begin{quote}
%%Australian College of Nursing
%%\end{quote}\strut
%%\end{minipage}\tabularnewline
%%\begin{minipage}[t]{0.30\columnwidth}\raggedright
%%\begin{quote}
%%The University of Sydney\^{}
%%\end{quote}\strut
%%\end{minipage} & \begin{minipage}[t]{0.30\columnwidth}\raggedright
%%\begin{quote}
%%University of the Sunshine Coast
%%\end{quote}\strut
%%\end{minipage} & \begin{minipage}[t]{0.30\columnwidth}\raggedright
%%\begin{quote}
%%Australian College of Physical Education
%%\end{quote}\strut
%%\end{minipage}\tabularnewline
%%\begin{minipage}[t]{0.30\columnwidth}\raggedright
%%\begin{quote}
%%The University of Queensland\^{}
%%\end{quote}\strut
%%\end{minipage} & \begin{minipage}[t]{0.30\columnwidth}\raggedright
%%\strut
%%\end{minipage} & \begin{minipage}[t]{0.30\columnwidth}\raggedright
%%\begin{quote}
%%Australian Guild of Music Education
%%\end{quote}\strut
%%\end{minipage}\tabularnewline
%%\begin{minipage}[t]{0.30\columnwidth}\raggedright
%%\begin{quote}
%%The University of Western Australia
%%\end{quote}\strut
%%\end{minipage} & \begin{minipage}[t]{0.30\columnwidth}\raggedright
%%\textbf{Other universities}\strut
%%\end{minipage} & \begin{minipage}[t]{0.30\columnwidth}\raggedright
%%\begin{quote}
%%Australian Institute of Business
%%\end{quote}\strut
%%\end{minipage}\tabularnewline
%%\begin{minipage}[t]{0.30\columnwidth}\raggedright
%%\strut
%%\end{minipage} & \begin{minipage}[t]{0.30\columnwidth}\raggedright
%%\begin{quote}
%%Australian Catholic University*
%%\end{quote}\strut
%%\end{minipage} & \begin{minipage}[t]{0.30\columnwidth}\raggedright
%%\begin{quote}
%%Australian Institute of Management
%%\end{quote}\strut
%%\end{minipage}\tabularnewline
%%\begin{minipage}[t]{0.30\columnwidth}\raggedright
%%\textbf{Australian Technology Network of Universities}\strut
%%\end{minipage} & \begin{minipage}[t]{0.30\columnwidth}\raggedright
%%\begin{quote}
%%Charles Sturt University*
%%\end{quote}\strut
%%\end{minipage} & \begin{minipage}[t]{0.30\columnwidth}\raggedright
%%\begin{quote}
%%Australian Institute of Music
%%\end{quote}\strut
%%\end{minipage}\tabularnewline
%%\begin{minipage}[t]{0.30\columnwidth}\raggedright
%%\begin{quote}
%%Curtin University of Technology
%%\end{quote}\strut
%%\end{minipage} & \begin{minipage}[t]{0.30\columnwidth}\raggedright
%%\begin{quote}
%%Bond University
%%\end{quote}\strut
%%\end{minipage} & \begin{minipage}[t]{0.30\columnwidth}\raggedright
%%\begin{quote}
%%Australian Institute of Professional Counsellors
%%\end{quote}\strut
%%\end{minipage}\tabularnewline
%%\begin{minipage}[t]{0.30\columnwidth}\raggedright
%%\begin{quote}
%%Queensland University of Technology*
%%\end{quote}\strut
%%\end{minipage} & \begin{minipage}[t]{0.30\columnwidth}\raggedright
%%\begin{quote}
%%Deakin University\^{}
%%\end{quote}\strut
%%\end{minipage} & \begin{minipage}[t]{0.30\columnwidth}\raggedright
%%\begin{quote}
%%BBI The Australian Institute of Theological Education
%%\end{quote}\strut
%%\end{minipage}\tabularnewline
%%\begin{minipage}[t]{0.30\columnwidth}\raggedright
%%\begin{quote}
%%RMIT University*
%%\end{quote}\strut
%%\end{minipage} & \begin{minipage}[t]{0.30\columnwidth}\raggedright
%%\begin{quote}
%%Edith Cowan University*
%%\end{quote}\strut
%%\end{minipage} & \begin{minipage}[t]{0.30\columnwidth}\raggedright
%%\begin{quote}
%%Box Hill Institute
%%\end{quote}\strut
%%\end{minipage}\tabularnewline
%%\begin{minipage}[t]{0.30\columnwidth}\raggedright
%%\begin{quote}
%%University of South Australia*
%%\end{quote}\strut
%%\end{minipage} & \begin{minipage}[t]{0.30\columnwidth}\raggedright
%%\begin{quote}
%%Macquarie University\^{}
%%\end{quote}\strut
%%\end{minipage} & \begin{minipage}[t]{0.30\columnwidth}\raggedright
%%\begin{quote}
%%Cairnmillar Institute
%%\end{quote}\strut
%%\end{minipage}\tabularnewline
%%\begin{minipage}[t]{0.30\columnwidth}\raggedright
%%\begin{quote}
%%University of Technology, Sydney*
%%\end{quote}\strut
%%\end{minipage} & \begin{minipage}[t]{0.30\columnwidth}\raggedright
%%\begin{quote}
%%Swinburne University of Technology*\textsuperscript{\^{}}
%%\end{quote}\strut
%%\end{minipage} & \begin{minipage}[t]{0.30\columnwidth}\raggedright
%%\begin{quote}
%%Campion College Australia
%%\end{quote}\strut
%%\end{minipage}\tabularnewline
%%\begin{minipage}[t]{0.30\columnwidth}\raggedright
%%\strut
%%\end{minipage} & \begin{minipage}[t]{0.30\columnwidth}\raggedright
%%\begin{quote}
%%The University of Newcastle\^{}
%%\end{quote}\strut
%%\end{minipage} & \begin{minipage}[t]{0.30\columnwidth}\raggedright
%%\begin{quote}
%%Canberra Institute of Technology
%%\end{quote}\strut
%%\end{minipage}\tabularnewline
%%\begin{minipage}[t]{0.30\columnwidth}\raggedright
%%\textbf{Innovative Research Universities }\strut
%%\end{minipage} & \begin{minipage}[t]{0.30\columnwidth}\raggedright
%%\begin{quote}
%%Torrens University Australia
%%\end{quote}\strut
%%\end{minipage} & \begin{minipage}[t]{0.30\columnwidth}\raggedright
%%\begin{quote}
%%Chisholm Institute
%%\end{quote}\strut
%%\end{minipage}\tabularnewline
%%\begin{minipage}[t]{0.30\columnwidth}\raggedright
%%\begin{quote}
%%Charles Darwin University*
%%\end{quote}\strut
%%\end{minipage} & \begin{minipage}[t]{0.30\columnwidth}\raggedright
%%\begin{quote}
%%University of Canberra*
%%\end{quote}\strut
%%\end{minipage} & \begin{minipage}[t]{0.30\columnwidth}\raggedright
%%\begin{quote}
%%Christian Heritage College
%%\end{quote}\strut
%%\end{minipage}\tabularnewline
%%\begin{minipage}[t]{0.30\columnwidth}\raggedright
%%\begin{quote}
%%Flinders University
%%\end{quote}\strut
%%\end{minipage} & \begin{minipage}[t]{0.30\columnwidth}\raggedright
%%\begin{quote}
%%University of Notre Dame, Australia
%%\end{quote}\strut
%%\end{minipage} & \begin{minipage}[t]{0.30\columnwidth}\raggedright
%%\begin{quote}
%%Collarts
%%\end{quote}\strut
%%\end{minipage}\tabularnewline
%%\begin{minipage}[t]{0.30\columnwidth}\raggedright
%%\begin{quote}
%%Griffith University\^{}
%%\end{quote}\strut
%%\end{minipage} & \begin{minipage}[t]{0.30\columnwidth}\raggedright
%%\begin{quote}
%%University of Tasmania\^{}
%%\end{quote}\strut
%%\end{minipage} & \begin{minipage}[t]{0.30\columnwidth}\raggedright
%%\begin{quote}
%%Curtin College
%%\end{quote}\strut
%%\end{minipage}\tabularnewline
%%\begin{minipage}[t]{0.30\columnwidth}\raggedright
%%\begin{quote}
%%James Cook University\^{}
%%\end{quote}\strut
%%\end{minipage} & \begin{minipage}[t]{0.30\columnwidth}\raggedright
%%\begin{quote}
%%University of Wollongong
%%\end{quote}\strut
%%\end{minipage} & \begin{minipage}[t]{0.30\columnwidth}\raggedright
%%\begin{quote}
%%Deakin College
%%\end{quote}\strut
%%\end{minipage}\tabularnewline
%%\begin{minipage}[t]{0.30\columnwidth}\raggedright
%%\begin{quote}
%%La Trobe University\^{}
%%\end{quote}\strut
%%\end{minipage} & \begin{minipage}[t]{0.30\columnwidth}\raggedright
%%\begin{quote}
%%Victoria University*
%%\end{quote}\strut
%%\end{minipage} & \begin{minipage}[t]{0.30\columnwidth}\raggedright
%%\begin{quote}
%%Eastern College Australia
%%\end{quote}\strut
%%\end{minipage}\tabularnewline
%%\begin{minipage}[t]{0.30\columnwidth}\raggedright
%%\begin{quote}
%%Murdoch University
%%\end{quote}\strut
%%\end{minipage} & \begin{minipage}[t]{0.30\columnwidth}\raggedright
%%\strut
%%\end{minipage} & \begin{minipage}[t]{0.30\columnwidth}\raggedright
%%\begin{quote}
%%Edith Cowan College
%%\end{quote}\strut
%%\end{minipage}\tabularnewline
%%\begin{minipage}[t]{0.30\columnwidth}\raggedright
%%\begin{quote}
%%Western Sydney University*
%%\end{quote}\strut
%%\end{minipage} & \begin{minipage}[t]{0.30\columnwidth}\raggedright
%%\textbf{Specialist university}\strut
%%\end{minipage} & \begin{minipage}[t]{0.30\columnwidth}\raggedright
%%\begin{quote}
%%Endeavour College of Natural Health
%%\end{quote}\strut
%%\end{minipage}\tabularnewline
%%\begin{minipage}[t]{0.30\columnwidth}\raggedright
%%\strut
%%\end{minipage} & \begin{minipage}[t]{0.30\columnwidth}\raggedright
%%University of Divinity\strut
%%\end{minipage} & \begin{minipage}[t]{0.30\columnwidth}\raggedright
%%\begin{quote}
%%Engineering Institute of Technology
%%\end{quote}\strut
%%\end{minipage}\tabularnewline
%%\begin{minipage}[t]{0.30\columnwidth}\raggedright
%%\strut
%%\end{minipage} & \begin{minipage}[t]{0.30\columnwidth}\raggedright
%%\strut
%%\end{minipage} & \begin{minipage}[t]{0.30\columnwidth}\raggedright
%%\begin{quote}
%%Eynesbury Education Group
%%\end{quote}\strut
%%\end{minipage}\tabularnewline
%%\begin{minipage}[t]{0.30\columnwidth}\raggedright
%%\strut
%%\end{minipage} & \begin{minipage}[t]{0.30\columnwidth}\raggedright
%%\textbf{Overseas university}\strut
%%\end{minipage} & \begin{minipage}[t]{0.30\columnwidth}\raggedright
%%\begin{quote}
%%Gestalt Therapy Brisbane
%%\end{quote}\strut
%%\end{minipage}\tabularnewline
%%\begin{minipage}[t]{0.30\columnwidth}\raggedright
%%\strut
%%\end{minipage} & \begin{minipage}[t]{0.30\columnwidth}\raggedright
%%\begin{quote}
%%Carnegie Mellon University
%%\end{quote}\strut
%%\end{minipage} & \begin{minipage}[t]{0.30\columnwidth}\raggedright
%%\begin{quote}
%%Griffith College
%%\end{quote}\strut
%%\end{minipage}\tabularnewline
%%\begin{minipage}[t]{0.30\columnwidth}\raggedright
%%\strut
%%\end{minipage} & \begin{minipage}[t]{0.30\columnwidth}\raggedright
%%\strut
%%\end{minipage} & \begin{minipage}[t]{0.30\columnwidth}\raggedright
%%\begin{quote}
%%Group Colleges Australia
%%\end{quote}\strut
%%\end{minipage}\tabularnewline
%%\bottomrule
%%\end{longtable}

%%\begin{longtable}[]{@{}lll@{}}
%%\toprule
%%\textbf{NUHEPs offering FEE-HELP} \emph{(continued from previous page)}\tabularnewline
%%\midrule
%%\endhead
%%\begin{minipage}[t]{0.30\columnwidth}\raggedright
%%\begin{quote}
%%Health Education and Training Institute
%%\end{quote}\strut
%%\end{minipage} & \begin{minipage}[t]{0.30\columnwidth}\raggedright
%%\begin{quote}
%%Macleay College
%%\end{quote}\strut
%%\end{minipage} & \begin{minipage}[t]{0.30\columnwidth}\raggedright
%%\begin{quote}
%%South Metropolitan TAFE
%%\end{quote}\strut
%%\end{minipage}\tabularnewline
%%\begin{minipage}[t]{0.30\columnwidth}\raggedright
%%\begin{quote}
%%Holmes Institute
%%\end{quote}\strut
%%\end{minipage} & \begin{minipage}[t]{0.30\columnwidth}\raggedright
%%\begin{quote}
%%Marcus Oldham College
%%\end{quote}\strut
%%\end{minipage} & \begin{minipage}[t]{0.30\columnwidth}\raggedright
%%\begin{quote}
%%Stott's Colleges
%%\end{quote}\strut
%%\end{minipage}\tabularnewline
%%\begin{minipage}[t]{0.30\columnwidth}\raggedright
%%\begin{quote}
%%Holmesglen Institute
%%\end{quote}\strut
%%\end{minipage} & \begin{minipage}[t]{0.30\columnwidth}\raggedright
%%\begin{quote}
%%Melbourne Institute of Technology
%%\end{quote}\strut
%%\end{minipage} & \begin{minipage}[t]{0.30\columnwidth}\raggedright
%%\begin{quote}
%%Study Group Australia
%%\end{quote}\strut
%%\end{minipage}\tabularnewline
%%\begin{minipage}[t]{0.30\columnwidth}\raggedright
%%\begin{quote}
%%IKON Institute of Australia
%%\end{quote}\strut
%%\end{minipage} & \begin{minipage}[t]{0.30\columnwidth}\raggedright
%%\begin{quote}
%%Melbourne Polytechnic
%%\end{quote}\strut
%%\end{minipage} & \begin{minipage}[t]{0.30\columnwidth}\raggedright
%%\begin{quote}
%%Sydney Institute of Business and Technology
%%\end{quote}\strut
%%\end{minipage}\tabularnewline
%%\begin{minipage}[t]{0.30\columnwidth}\raggedright
%%\begin{quote}
%%International College of Hotel Management
%%\end{quote}\strut
%%\end{minipage} & \begin{minipage}[t]{0.30\columnwidth}\raggedright
%%\begin{quote}
%%MIECAT
%%\end{quote}\strut
%%\end{minipage} & \begin{minipage}[t]{0.30\columnwidth}\raggedright
%%\begin{quote}
%%Sydney Institute of Traditional Chinese Medicine
%%\end{quote}\strut
%%\end{minipage}\tabularnewline
%%\begin{minipage}[t]{0.30\columnwidth}\raggedright
%%\begin{quote}
%%International College of Management
%%\end{quote}\strut
%%\end{minipage} & \begin{minipage}[t]{0.30\columnwidth}\raggedright
%%\begin{quote}
%%Monash College
%%\end{quote}\strut
%%\end{minipage} & \begin{minipage}[t]{0.30\columnwidth}\raggedright
%%\begin{quote}
%%Tabor Adelaide
%%\end{quote}\strut
%%\end{minipage}\tabularnewline
%%\begin{minipage}[t]{0.30\columnwidth}\raggedright
%%\begin{quote}
%%ISN Psychology
%%\end{quote}\strut
%%\end{minipage} & \begin{minipage}[t]{0.30\columnwidth}\raggedright
%%\begin{quote}
%%Morling College
%%\end{quote}\strut
%%\end{minipage} & \begin{minipage}[t]{0.30\columnwidth}\raggedright
%%\begin{quote}
%%Tabor College NSW
%%\end{quote}\strut
%%\end{minipage}\tabularnewline
%%\begin{minipage}[t]{0.30\columnwidth}\raggedright
%%\begin{quote}
%%Jazz Music Institute
%%\end{quote}\strut
%%\end{minipage} & \begin{minipage}[t]{0.30\columnwidth}\raggedright
%%\begin{quote}
%%Nan Tien Institute
%%\end{quote}\strut
%%\end{minipage} & \begin{minipage}[t]{0.30\columnwidth}\raggedright
%%\begin{quote}
%%TAFE NSW Higher Education
%%\end{quote}\strut
%%\end{minipage}\tabularnewline
%%\begin{minipage}[t]{0.30\columnwidth}\raggedright
%%\begin{quote}
%%JMC Academy
%%\end{quote}\strut
%%\end{minipage} & \begin{minipage}[t]{0.30\columnwidth}\raggedright
%%\begin{quote}
%%National Art School
%%\end{quote}\strut
%%\end{minipage} & \begin{minipage}[t]{0.30\columnwidth}\raggedright
%%\begin{quote}
%%TAFE Queensland
%%\end{quote}\strut
%%\end{minipage}\tabularnewline
%%\begin{minipage}[t]{0.30\columnwidth}\raggedright
%%\begin{quote}
%%John Paul II Institute for Marriage and Family
%%\end{quote}\strut
%%\end{minipage} & \begin{minipage}[t]{0.30\columnwidth}\raggedright
%%\begin{quote}
%%North Metropolitan TAFE
%%\end{quote}\strut
%%\end{minipage} & \begin{minipage}[t]{0.30\columnwidth}\raggedright
%%\begin{quote}
%%TAFE South Australia
%%\end{quote}\strut
%%\end{minipage}\tabularnewline
%%\begin{minipage}[t]{0.30\columnwidth}\raggedright
%%\begin{quote}
%%Kaplan Business School
%%\end{quote}\strut
%%\end{minipage} & \begin{minipage}[t]{0.30\columnwidth}\raggedright
%%\begin{quote}
%%Paramount College of Natural Health
%%\end{quote}\strut
%%\end{minipage} & \begin{minipage}[t]{0.30\columnwidth}\raggedright
%%\begin{quote}
%%Think Education
%%\end{quote}\strut
%%\end{minipage}\tabularnewline
%%\begin{minipage}[t]{0.30\columnwidth}\raggedright
%%\begin{quote}
%%Kaplan Higher Education
%%\end{quote}\strut
%%\end{minipage} & \begin{minipage}[t]{0.30\columnwidth}\raggedright
%%\begin{quote}
%%Perth Bible College
%%\end{quote}\strut
%%\end{minipage} & \begin{minipage}[t]{0.30\columnwidth}\raggedright
%%\begin{quote}
%%Turning Point Alcohol and Drug Centre
%%\end{quote}\strut
%%\end{minipage}\tabularnewline
%%\begin{minipage}[t]{0.30\columnwidth}\raggedright
%%\begin{quote}
%%Kent Institute
%%\end{quote}\strut
%%\end{minipage} & \begin{minipage}[t]{0.30\columnwidth}\raggedright
%%\begin{quote}
%%Photography Studies College
%%\end{quote}\strut
%%\end{minipage} & \begin{minipage}[t]{0.30\columnwidth}\raggedright
%%\begin{quote}
%%UOW College Australia
%%\end{quote}\strut
%%\end{minipage}\tabularnewline
%%\begin{minipage}[t]{0.30\columnwidth}\raggedright
%%\begin{quote}
%%King's Own Institute
%%\end{quote}\strut
%%\end{minipage} & \begin{minipage}[t]{0.30\columnwidth}\raggedright
%%\begin{quote}
%%Raffles College of Design and Commerce
%%\end{quote}\strut
%%\end{minipage} & \begin{minipage}[t]{0.30\columnwidth}\raggedright
%%\begin{quote}
%%UTS Insearch
%%\end{quote}\strut
%%\end{minipage}\tabularnewline
%%\begin{minipage}[t]{0.30\columnwidth}\raggedright
%%\begin{quote}
%%La Trobe College
%%\end{quote}\strut
%%\end{minipage} & \begin{minipage}[t]{0.30\columnwidth}\raggedright
%%\begin{quote}
%%Russo Business School
%%\end{quote}\strut
%%\end{minipage} & \begin{minipage}[t]{0.30\columnwidth}\raggedright
%%\begin{quote}
%%Victorian Institute of Technology
%%\end{quote}\strut
%%\end{minipage}\tabularnewline
%%\begin{minipage}[t]{0.30\columnwidth}\raggedright
%%\begin{quote}
%%LCI Melbourne
%%\end{quote}\strut
%%\end{minipage} & \begin{minipage}[t]{0.30\columnwidth}\raggedright
%%\begin{quote}
%%S P Jain School of Global Management
%%\end{quote}\strut
%%\end{minipage} & \begin{minipage}[t]{0.30\columnwidth}\raggedright
%%\begin{quote}
%%Wentworth Institute
%%\end{quote}\strut
%%\end{minipage}\tabularnewline
%%\begin{minipage}[t]{0.30\columnwidth}\raggedright
%%\begin{quote}
%%Le Cordon Bleu Australia
%%\end{quote}\strut
%%\end{minipage} & \begin{minipage}[t]{0.30\columnwidth}\raggedright
%%\begin{quote}
%%SAE Institute
%%\end{quote}\strut
%%\end{minipage} & \begin{minipage}[t]{0.30\columnwidth}\raggedright
%%\begin{quote}
%%Whitehouse Institute of Design
%%\end{quote}\strut
%%\end{minipage}\tabularnewline
%%\begin{minipage}[t]{0.30\columnwidth}\raggedright
%%\begin{quote}
%%Leo Cussen Centre for Law
%%\end{quote}\strut
%%\end{minipage} & \begin{minipage}[t]{0.30\columnwidth}\raggedright
%%\begin{quote}
%%South Australian Institute of Business and Technology
%%\end{quote}\strut
%%\end{minipage} & \begin{minipage}[t]{0.30\columnwidth}\raggedright
%%\begin{quote}
%%William Angliss Institute
%%\end{quote}\strut
%%\end{minipage}\tabularnewline
%%\bottomrule
%%\end{longtable}

%%\begin{longtable}[]{@{}lll@{}}
%%\toprule
%%\textbf{Self-accrediting NUHEPs offering FEE-HELP}\tabularnewline
%%\midrule
%%\endhead
%%\begin{minipage}[t]{0.30\columnwidth}\raggedright
%%\begin{quote}
%%Alphacrucis College
%%\end{quote}\strut
%%\end{minipage} & \begin{minipage}[t]{0.30\columnwidth}\raggedright
%%\begin{quote}
%%Avondale College of Higher Education
%%\end{quote}\strut
%%\end{minipage} & \begin{minipage}[t]{0.30\columnwidth}\raggedright
%%\begin{quote}
%%Moore College
%%\end{quote}\strut
%%\end{minipage}\tabularnewline
%%\begin{minipage}[t]{0.30\columnwidth}\raggedright
%%\begin{quote}
%%Australian College of Applied Psychology
%%\end{quote}\strut
%%\end{minipage} & \begin{minipage}[t]{0.30\columnwidth}\raggedright
%%\begin{quote}
%%Batchelor Institute of Indigenous Tertiary Education
%%\end{quote}\strut
%%\end{minipage} & \begin{minipage}[t]{0.30\columnwidth}\raggedright
%%\begin{quote}
%%National Institute of Dramatic Art
%%\end{quote}\strut
%%\end{minipage}\tabularnewline
%%\begin{minipage}[t]{0.30\columnwidth}\raggedright
%%\begin{quote}
%%Australian College of Theology
%%\end{quote}\strut
%%\end{minipage} & \begin{minipage}[t]{0.30\columnwidth}\raggedright
%%\begin{quote}
%%College of Law
%%\end{quote}\strut
%%\end{minipage} & \begin{minipage}[t]{0.30\columnwidth}\raggedright
%%\begin{quote}
%%Sydney College of Divinity
%%\end{quote}\strut
%%\end{minipage}\tabularnewline
%%\begin{minipage}[t]{0.30\columnwidth}\raggedright
%%\begin{quote}
%%Australian Film, Television and Radio School
%%\end{quote}\strut
%%\end{minipage} & \begin{minipage}[t]{0.30\columnwidth}\raggedright
%%\begin{quote}
%%Excelsia College
%%\end{quote}\strut
%%\end{minipage} & \begin{minipage}[t]{0.30\columnwidth}\raggedright
%%\begin{quote}
%%Top Education Institute
%%\end{quote}\strut
%%\end{minipage}\tabularnewline
%%\bottomrule
%%\end{longtable}

Note: Trading names used. * Established or given university status as a result of the John Dawkins education reforms. \^{} Amalgamated with other providers during the John Dawkins education reforms. University name changes: Charles Darwin University was the Northern Territory University until 2004. Federation University Australia was the University of Ballarat until 2014. Western Sydney University was the University of Western Sydney until 2016. The University of the Sunshine Coast was established in 1998.

Source: TEQSA (2018b)

%
\chapter{Appendix B -- Higher education providers not offering HELP loans}\label{chap:appendix-b-higher-education-providers-not-offering-help-loans}

%%\begin{longtable}[]{@{}lll@{}}
%%\toprule
%%\textbf{NUHEPs not offering FEE-HELP}\tabularnewline
%%\midrule
%%\endhead
%%\begin{minipage}[t]{0.30\columnwidth}\raggedright
%%\begin{quote}
%%Australasian College of Dermatologists
%%\end{quote}\strut
%%\end{minipage} & \begin{minipage}[t]{0.30\columnwidth}\raggedright
%%\begin{quote}
%%Governance Institute of Australia
%%\end{quote}\strut
%%\end{minipage} & \begin{minipage}[t]{0.30\columnwidth}\raggedright
%%\begin{quote}
%%Montessori World Educational Institute
%%\end{quote}\strut
%%\end{minipage}\tabularnewline
%%\begin{minipage}[t]{0.30\columnwidth}\raggedright
%%\begin{quote}
%%Academies Australasia Polytechnic
%%\end{quote}\strut
%%\end{minipage} & \begin{minipage}[t]{0.30\columnwidth}\raggedright
%%\begin{quote}
%%Higher Education Leadership Institute
%%\end{quote}\strut
%%\end{minipage} & \begin{minipage}[t]{0.30\columnwidth}\raggedright
%%\begin{quote}
%%National Institute of Organisation Dynamics Australia
%%\end{quote}\strut
%%\end{minipage}\tabularnewline
%%\begin{minipage}[t]{0.30\columnwidth}\raggedright
%%\begin{quote}
%%ACER Professional Learning
%%\end{quote}\strut
%%\end{minipage} & \begin{minipage}[t]{0.30\columnwidth}\raggedright
%%\begin{quote}
%%Institute for Emotionally Focused Therapy
%%\end{quote}\strut
%%\end{minipage} & \begin{minipage}[t]{0.30\columnwidth}\raggedright
%%\begin{quote}
%%Newcastle International College
%%\end{quote}\strut
%%\end{minipage}\tabularnewline
%%\begin{minipage}[t]{0.30\columnwidth}\raggedright
%%\begin{quote}
%%Asia Pacific International College
%%\end{quote}\strut
%%\end{minipage} & \begin{minipage}[t]{0.30\columnwidth}\raggedright
%%\begin{quote}
%%Institute of Chartered Accountants in Australia
%%\end{quote}\strut
%%\end{minipage} & \begin{minipage}[t]{0.30\columnwidth}\raggedright
%%\begin{quote}
%%Ozford Institute of Higher Education
%%\end{quote}\strut
%%\end{minipage}\tabularnewline
%%\begin{minipage}[t]{0.30\columnwidth}\raggedright
%%\begin{quote}
%%Australian Institute of Higher Education
%%\end{quote}\strut
%%\end{minipage} & \begin{minipage}[t]{0.30\columnwidth}\raggedright
%%\begin{quote}
%%Institute of Health \& Management
%%\end{quote}\strut
%%\end{minipage} & \begin{minipage}[t]{0.30\columnwidth}\raggedright
%%\begin{quote}
%%Polytechnic Institute
%%\end{quote}\strut
%%\end{minipage}\tabularnewline
%%\begin{minipage}[t]{0.30\columnwidth}\raggedright
%%\begin{quote}
%%Australian Institute of Police Management
%%\end{quote}\strut
%%\end{minipage} & \begin{minipage}[t]{0.30\columnwidth}\raggedright
%%\begin{quote}
%%Institute of Internal Auditors
%%\end{quote}\strut
%%\end{minipage} & \begin{minipage}[t]{0.30\columnwidth}\raggedright
%%\begin{quote}
%%Sheridan College
%%\end{quote}\strut
%%\end{minipage}\tabularnewline
%%\begin{minipage}[t]{0.30\columnwidth}\raggedright
%%\begin{quote}
%%Bureau of Meteorology Training Centre
%%\end{quote}\strut
%%\end{minipage} & \begin{minipage}[t]{0.30\columnwidth}\raggedright
%%\begin{quote}
%%Institute of International Studies
%%\end{quote}\strut
%%\end{minipage} & \begin{minipage}[t]{0.30\columnwidth}\raggedright
%%\begin{quote}
%%Southern Cross Education Institute
%%\end{quote}\strut
%%\end{minipage}\tabularnewline
%%\begin{minipage}[t]{0.30\columnwidth}\raggedright
%%\begin{quote}
%%Cambridge International College
%%\end{quote}\strut
%%\end{minipage} & \begin{minipage}[t]{0.30\columnwidth}\raggedright
%%\begin{quote}
%%International Institute of Business and Technology
%%\end{quote}\strut
%%\end{minipage} & \begin{minipage}[t]{0.30\columnwidth}\raggedright
%%\begin{quote}
%%Tax Institute
%%\end{quote}\strut
%%\end{minipage}\tabularnewline
%%\begin{minipage}[t]{0.30\columnwidth}\raggedright
%%\begin{quote}
%%Centre for Pavement Engineering Education
%%\end{quote}\strut
%%\end{minipage} & \begin{minipage}[t]{0.30\columnwidth}\raggedright
%%\begin{quote}
%%Kollel Beth HaTalmud Yehuda Fishman Institute
%%\end{quote}\strut
%%\end{minipage} & \begin{minipage}[t]{0.30\columnwidth}\raggedright
%%\begin{quote}
%%Western Sydney University International College
%%\end{quote}\strut
%%\end{minipage}\tabularnewline
%%\begin{minipage}[t]{0.30\columnwidth}\raggedright
%%\begin{quote}
%%Elite Education Institute
%%\end{quote}\strut
%%\end{minipage} & \begin{minipage}[t]{0.30\columnwidth}\raggedright
%%\begin{quote}
%%Mayfield Education
%%\end{quote}\strut
%%\end{minipage} & \begin{minipage}[t]{0.30\columnwidth}\raggedright
%%\strut
%%\end{minipage}\tabularnewline
%%\bottomrule
%%\end{longtable}

Note: Trading names used.

Sources: TEQSA (2018b)

%
\chapter{Appendix C -- Fields of Education}\label{chap:appendix-c-fields-of-education}

%%\begin{longtable}[]{@{}ll@{}}
%%\toprule
%%\textbf{Agriculture} & Agriculture is a 4-digit field, and is a subset of the 2-digit field `Agriculture, Environmental and Related Studies'. Agriculture includes `Agricultural Science', `Animal Husbandry', `Wool Science' and `Agriculture, n.e.c.'.\tabularnewline
%%\midrule
%%\endhead
%%\textbf{Architecture} & Architecture is a 6-digit field, and is a subset of the 2-digit field ``Architecture and Building''.\tabularnewline
%%\textbf{Commerce} & Commerce is the 2-digit field `Management and Commerce'. It includes `Accounting', `Business and Management', `Sales and Marketing', `Tourism', `Office Studies', `Banking Finance and Related Fields', and `Other Management and Commerce.\tabularnewline
%%\textbf{Education} & Education is a 2-digit field. It includes `Teacher Education' Curriculum and Education Studies' and `Other Education'.\tabularnewline
%%\textbf{Engineering} & Engineering is a 2-digit field. It includes, `Manufacturing', `Process and Resources', `Automotive', `Mechanical and Industrial', `Civil', `Geomatic', `Electrical and Electronic', `Aerospace', `Maritime', and `Other Engineering and Related Technologies'.\tabularnewline
%%\textbf{Humanities} & `Humanities' is a category defined in this paper. It is a subset of the 2-digit field `Society and Culture'. It includes `Political Science and Policy Studies', `Studies in Human Society', `Language and Literature', `Philosophy and Religious Studies'.\tabularnewline
%%\textbf{Information Technology} & Information Technology is a 2-digit field. It includes `Computer Science', `Information Systems' and `Other Information Technology'.\tabularnewline
%%\textbf{Law} & Law is a 4-digit field, and is a subset of the 2-digit field `Society and Culture'. Law includes `Business and Commercial Law', `Constitutional Law', `Criminal Law', `Family Law', `International Law', `Taxation Law', `Legal Practice' and `Law, n.e.c.'\tabularnewline
%%\textbf{Mathematics} & Mathematics is the 4-digit field `Mathematical Sciences', and is a subset of the 2-digit field `Natural and Phycial Sciences'. Mathematical Sciences includes `Mathematics', `Statistics' and `Mathematical Sciences n.e.c.'.\tabularnewline
%%\textbf{Medicine} & Medicine is the 4-digit field `Medical Studies', and is a subset of the 2-digit field Health. Medical Studies includes `General Medicine', `Surgery', `Psychiatry', `Obstetrics and Gynaecology', `Paediatrics', `Anaesthesiology', `Pathology', `Radiology', `Internal Medicine', `General Practice', `Medical Studies n.e.c.'.\tabularnewline
%%\textbf{Nursing} & Nursing is a 4-digit field, and is a subset of the 2-digit field `Health'. Nursing includes, `General Nursing', `Midwifery', `Mental Health Nursing', `Community Nursing', `Critical Care Nursing', `Aged Care Nursing', `Palliative Care Nursing', `Mothercraft Nursing and Family and Child Health Nursing', `Nursing, n.e.c.'.\tabularnewline
%%\textbf{Other health} & Other health is a subset of the two-digit field `Health'. It contains all fields not otherwise contained in `Medicine', `Dentistry' and `Nursing'. Other health includes `Pharmacy', `Optical Science', `Veterinary Studies', `Public Health', `Radiography', `Rehabilitation Therapies', `Complementary Therapies', `Other health'.\tabularnewline
%%\textbf{Performing arts} & Performing arts is a 4-digit field, and is a subset of the 2-digit field `Creative Arts'. Performing Arts includes `Music', `Drama and Theatre Studies', `Dance' and `Performing Arts n.e.c.'.\tabularnewline
%%\textbf{Sciences (excl. maths)} & `Sciences (excl. maths) is a category defined in this paper. It is the 2-digit field `Natural and Physical Sciences' with the 4-digit field `Mathematical Sciences' removed'. The category includes `Natural and Physical Science n.f.d.', `Physics and Astronomy', `Chemical Sciences', `Earth Sciences', `Biological Sciences' and `Other Natural and Physical Sciences'.\tabularnewline
%%\bottomrule
%%\end{longtable}

Source: Australian Standard Classification of Education, ABS (2001)

%

\printbibliography

\end{document}
